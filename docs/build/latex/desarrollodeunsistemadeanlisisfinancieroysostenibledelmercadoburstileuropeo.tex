%% Generated by Sphinx.
\def\sphinxdocclass{report}
\documentclass[letterpaper,10pt,spanish]{sphinxmanual}
\ifdefined\pdfpxdimen
   \let\sphinxpxdimen\pdfpxdimen\else\newdimen\sphinxpxdimen
\fi \sphinxpxdimen=.75bp\relax
\ifdefined\pdfimageresolution
    \pdfimageresolution= \numexpr \dimexpr1in\relax/\sphinxpxdimen\relax
\fi
%% let collapsible pdf bookmarks panel have high depth per default
\PassOptionsToPackage{bookmarksdepth=5}{hyperref}

\PassOptionsToPackage{booktabs}{sphinx}
\PassOptionsToPackage{colorrows}{sphinx}

\PassOptionsToPackage{warn}{textcomp}
\usepackage[utf8]{inputenc}
\ifdefined\DeclareUnicodeCharacter
% support both utf8 and utf8x syntaxes
  \ifdefined\DeclareUnicodeCharacterAsOptional
    \def\sphinxDUC#1{\DeclareUnicodeCharacter{"#1}}
  \else
    \let\sphinxDUC\DeclareUnicodeCharacter
  \fi
  \sphinxDUC{00A0}{\nobreakspace}
  \sphinxDUC{2500}{\sphinxunichar{2500}}
  \sphinxDUC{2502}{\sphinxunichar{2502}}
  \sphinxDUC{2514}{\sphinxunichar{2514}}
  \sphinxDUC{251C}{\sphinxunichar{251C}}
  \sphinxDUC{2572}{\textbackslash}
\fi
\usepackage{cmap}
\usepackage[T1]{fontenc}
\usepackage{amsmath,amssymb,amstext}
\usepackage{babel}



\usepackage{tgtermes}
\usepackage{tgheros}
\renewcommand{\ttdefault}{txtt}



\usepackage[Sonny]{fncychap}
\ChNameVar{\Large\normalfont\sffamily}
\ChTitleVar{\Large\normalfont\sffamily}
\usepackage{sphinx}

\fvset{fontsize=auto}
\usepackage{geometry}


% Include hyperref last.
\usepackage{hyperref}
% Fix anchor placement for figures with captions.
\usepackage{hypcap}% it must be loaded after hyperref.
% Set up styles of URL: it should be placed after hyperref.
\urlstyle{same}

\addto\captionsspanish{\renewcommand{\contentsname}{Contents:}}

\usepackage{sphinxmessages}
\setcounter{tocdepth}{1}



\title{Desarrollo de un sistema de análisis financiero y sostenible del mercado bursátil europeo}
\date{18 de septiembre de 2025}
\release{0.1.0}
\author{Julia Escudero, Marco Mendieta, Marco Pacora, Piero Rios, Juan Carlos Romero}
\newcommand{\sphinxlogo}{\vbox{}}
\renewcommand{\releasename}{Versión}
\makeindex
\begin{document}

\ifdefined\shorthandoff
  \ifnum\catcode`\=\string=\active\shorthandoff{=}\fi
  \ifnum\catcode`\"=\active\shorthandoff{"}\fi
\fi

\pagestyle{empty}
\sphinxmaketitle
\pagestyle{plain}
\sphinxtableofcontents
\pagestyle{normal}
\phantomsection\label{\detokenize{index::doc}}


\sphinxstepscope


\chapter{Introducción}
\label{\detokenize{Introduccion:introduccion}}\label{\detokenize{Introduccion::doc}}

\section{Contextualización del problema}
\label{\detokenize{Introduccion:contextualizacion-del-problema}}
\sphinxAtStartPar
En la actualidad, los mercados financieros europeos enfrentan una creciente complejidad derivada de la globalización económica, la volatilidad macroeconómica y la alta interconexión entre sectores y países. Si bien los métodos tradicionales de análisis —fundamental y técnico— han sido ampliamente utilizados para evaluar activos, resultan limitados para capturar el efecto de variables intangibles, como la percepción social, la sostenibilidad empresarial o la reacción inmediata a las noticias y eventos geopolíticos.

\sphinxAtStartPar
En este escenario, el análisis de sentimiento y la aplicación de técnicas de machine learning se consolidan como enfoques innovadores para complementar las metodologías financieras clásicas. Diversos estudios recientes han demostrado que la información procedente de noticias, redes sociales y métricas de sostenibilidad puede anticipar movimientos de mercado y mejorar la precisión de los modelos predictivos (Bollen, Mao \& Zeng, 2011; Nassirtoussi et al., 2014).

\sphinxAtStartPar
La crisis sanitaria global de la COVID\sphinxhyphen{}19 (2020) marcó un punto de inflexión en los mercados financieros internacionales. En cuestión de semanas, la volatilidad alcanzó niveles históricos y los inversores se enfrentaron a un entorno de incertidumbre sin precedentes. En Europa, las bolsas sufrieron caídas abruptas seguidas de una rápida recuperación impulsada por estímulos fiscales y monetarios, lo que evidenció la fragilidad de los métodos tradicionales de análisis financiero ante choques exógenos.

\sphinxAtStartPar
Al mismo tiempo, la digitalización y el auge de las plataformas abiertas de datos financieros han impulsado el desarrollo de herramientas de visualización y automatización que democratizan el acceso a la información. Sin embargo, en Europa, la mayoría de estas soluciones se encuentran fragmentadas, con una fuerte concentración en mercados estadounidenses, lo que genera un vacío en el análisis integral y accesible para el inversor medio europeo.

\sphinxAtStartPar
Por ello, la construcción de visualizaciones interactivas y un asistente conversacional basado en datos financieros, métricas ESG (Environmental, Social and Governance) y análisis de noticias, responde a una necesidad actual: disponer de un sistema abierto, transparente y automatizado que apoye la toma de decisiones de inversión en un entorno caracterizado por la rapidez y la incertidumbre.


\bigskip\hrule\bigskip



\section{Justificación del tema}
\label{\detokenize{Introduccion:justificacion-del-tema}}
\sphinxAtStartPar
La elección de este tema se fundamenta en la necesidad de contar con herramientas más completas para analizar los mercados financieros europeos, donde el análisis fundamental y técnico, aunque útiles, resultan insuficientes por sí solos. La incorporación del análisis de sentimiento y el análisis cuantitativo permite integrar datos financieros, métricas de sostenibilidad y flujos de información en tiempo real, ofreciendo una visión más robusta y actualizada. Además, el desarrollo de visualizaciones interactivas y un bot de consulta en código abierto aporta un valor práctico al democratizar el acceso a información estructurada y apoyar la toma de decisiones de inversión.


\bigskip\hrule\bigskip



\section{Objetivos del trabajo (general y específicos)}
\label{\detokenize{Introduccion:objetivos-del-trabajo-general-y-especificos}}
\sphinxAtStartPar
\sphinxstylestrong{Objetivo General:}
\begin{itemize}
\item {} 
\sphinxAtStartPar
Desarrollar una herramienta de código abierto que integre datos financieros, métricas de sostenibilidad y análisis de noticias para ofrecer asesoría financiera sobre el mercado de valores europeo, mediante la implementación de visualizaciones interactivas y un bot de consulta automatizado. Esta plataforma permitirá visualizar información estructurada y actualizada, así como facilitar el acceso a recomendaciones y análisis basados en machine learning, apoyando la toma de decisiones de inversión en un entorno accesible y dinámico.

\end{itemize}

\sphinxAtStartPar
\sphinxstylestrong{Objetivos específicos:}
\begin{itemize}
\item {} 
\sphinxAtStartPar
Recolectar, depurar y estructurar datos financieros de los principales índices bursátiles europeos para su posterior análisis.

\item {} 
\sphinxAtStartPar
Seleccionar, implementar y justificar los algoritmos de machine learning más adecuados para la construcción y optimización de carteras de inversión.

\item {} 
\sphinxAtStartPar
Evaluar el desempeño de los modelos mediante métricas financieras y de riesgo, garantizando la solidez de los resultados.

\item {} 
\sphinxAtStartPar
Implementar un sistema de actualización automática en tiempo casi real mediante APIs, que permita monitorear de manera continua la evolución de las carteras de inversión.

\item {} 
\sphinxAtStartPar
Desarrollar visualizaciones interactivas que faciliten la interacción, proyecciones y escenarios, apoyando la toma de decisiones de los inversores.

\item {} 
\sphinxAtStartPar
Integrar un asistente conversacional utilizando inteligencia artificial, accesible a través de la plataforma Telegram, que proporcione información en directo a partir de las bases de datos del proyecto.

\end{itemize}

\sphinxstepscope


\chapter{Descripción Problema}
\label{\detokenize{DescripcionProblema:descripcion-problema}}\label{\detokenize{DescripcionProblema::doc}}
\sphinxAtStartPar
En el análisis de los mercados bursátiles, tradicionalmente se distinguen dos enfoques principales: el análisis fundamental y el análisis técnico. El análisis fundamental se apoya en datos financieros como beneficios, pérdidas, márgenes operativos o estructuras de capital para evaluar la solidez y el rendimiento de una empresa. Este enfoque resulta especialmente útil para inversiones a mediano y largo plazo, dado que permite valorar el potencial intrínseco de una compañía. No obstante, presenta limitaciones, ya que puede verse afectado por la manipulación de cifras contables, la opacidad en la presentación de los estados financieros o, incluso, por factores externos como las decisiones de política monetaria (ej. cambios en los tipos de interés).

\sphinxAtStartPar
El análisis técnico, en contraste, prescinde de los fundamentos contables y se centra exclusivamente en el estudio del comportamiento histórico de los precios y volúmenes de negociación. Se basa en la identificación de patrones recurrentes y en el uso de herramientas gráficas —como los gráficos de velas, soportes, resistencias o indicadores chartistas— para anticipar movimientos futuros del mercado. Aunque este enfoque es especialmente útil en horizontes de corto plazo y en la toma de decisiones rápidas basadas en tendencias, también enfrenta limitaciones, dado que depende en gran medida de la validez de los patrones pasados en contextos futuros inciertos.

\sphinxAtStartPar
Sin embargo, en la actualidad los mercados financieros han demostrado que los precios no dependen únicamente de estas dos dimensiones tradicionales. La percepción y comportamiento de los inversores juegan un papel cada vez más decisivo, influenciados por la información que circula en medios de comunicación, redes sociales y foros especializados. Estos flujos de información generan cambios en el sentimiento del mercado, que en ocasiones logran anticipar movimientos alcistas o bajistas incluso antes de que se reflejen en los datos financieros o en los gráficos de precios.

\sphinxAtStartPar
Ante esta realidad, surge la necesidad de incorporar un tercer enfoque: el análisis de sentimiento, basado en técnicas de Natural Language Processing (NLP) y análisis de texto. Esta aproximación permite cuantificar percepciones colectivas y medir su impacto potencial en los mercados, complementando así los enfoques fundamental y técnico. El valor añadido de esta integración radica en su capacidad para identificar correlaciones entre lo que se dice sobre una empresa, sector o temática financiera y el comportamiento posterior de los precios en bolsa.

\sphinxAtStartPar
Finalmente, el análisis cuantitativo aporta un cuarto pilar que fortalece la capacidad predictiva del modelo. Mediante la aplicación de algoritmos matemáticos, estadísticos y de machine learning, este enfoque permite combinar grandes volúmenes de datos financieros y de noticias, detectando patrones complejos que resultan invisibles a través de métodos tradicionales. De esta forma, la integración del análisis fundamental, técnico, de sentimiento y cuantitativo constituye un enfoque integral, que no solo mejora la precisión de las predicciones en los mercados bursátiles, sino que también amplía las herramientas disponibles para la toma de decisiones estratégicas en inversión.


\bigskip\hrule\bigskip



\section{¿Qué es el mercado bursátil?}
\label{\detokenize{DescripcionProblema:que-es-el-mercado-bursatil}}
\sphinxAtStartPar
El mercado bursátil es el espacio donde se negocian instrumentos financieros como acciones y bonos. Su función principal es facilitar la inversión y la financiación de empresas, permitiendo que los precios reflejen las expectativas de los participantes.

\sphinxAtStartPar
Estos precios no solo responden a datos económicos, sino también a factores emocionales y percepciones colectivas influenciadas por noticias, redes sociales y medios especializados. Por eso, el análisis de sentimiento aplicado a textos puede ser clave para anticipar movimientos del mercado, al detectar señales tempranas de optimismo o temor entre los inversores.

\sphinxAtStartPar
El mercado bursátil europeo es el conjunto de bolsas de valores que operan en los países de Europa, donde se negocian instrumentos financieros como acciones, bonos, derivados y fondos cotizados. Este mercado desempeña un papel fundamental en la economía del continente, ya que permite a empresas europeas captar inversión y a los inversores gestionar su capital a través de activos financieros.

\sphinxAtStartPar
Las principales bolsas europeas incluyen:
\begin{itemize}
\item {} 
\sphinxAtStartPar
\sphinxstylestrong{Deutsche Börse} (Alemania),

\item {} 
\sphinxAtStartPar
\sphinxstylestrong{London Stock Exchange (LSE)} (Reino Unido),

\item {} 
\sphinxAtStartPar
\sphinxstylestrong{BME \textendash{} Bolsas y Mercados Españoles} (España),

\item {} 
\sphinxAtStartPar
\sphinxstylestrong{Swiss Exchange (SIX)} (Suiza).

\end{itemize}

\sphinxAtStartPar
Estas bolsas están interconectadas y son reguladas por entidades nacionales e internacionales que garantizan transparencia, legalidad y eficiencia en las transacciones.

\sphinxAtStartPar
El funcionamiento del mercado bursátil europeo se basa en la ley de oferta y demanda, y sus precios se ven influenciados no solo por indicadores económicos y financieros, sino también por factores externos como noticias políticas, eventos geopolíticos y, especialmente, la percepción colectiva de los inversores.

\sphinxAtStartPar
En este contexto, el análisis de sentimiento se ha convertido en una herramienta poderosa para los analistas e inversionistas. Al analizar lo que se comunica en medios financieros, redes sociales y portales especializados en Europa, es posible detectar cambios en el estado de ánimo del mercado (optimismo, temor, incertidumbre) y anticipar posibles movimientos alcistas o bajistas en los precios de las acciones y otros activos.


\bigskip\hrule\bigskip



\section{Hipótesis del estudio}
\label{\detokenize{DescripcionProblema:hipotesis-del-estudio}}\begin{itemize}
\item {} 
\sphinxAtStartPar
\sphinxstylestrong{Hipótesis nula (H\(\sb{\text{0}}\)):}\\
No existe relación entre el uso del asistente financiero automatizado y una mejora en el análisis del mercado europeo o en la toma de decisiones de inversión.

\item {} 
\sphinxAtStartPar
\sphinxstylestrong{Hipótesis alternativa (H\(\sb{\text{1}}\)):}\\
Existe una relación entre el uso del asistente financiero automatizado y una mejora en el análisis del mercado europeo o en la toma de decisiones de inversión.

\end{itemize}


\bigskip\hrule\bigskip



\section{Alcance del proyecto (temporal, geográfico, sectorial)}
\label{\detokenize{DescripcionProblema:alcance-del-proyecto-temporal-geografico-sectorial}}
\sphinxAtStartPar
El presente proyecto tiene un alcance temporal definido entre enero de 2020 y septiembre de 2025, periodo caracterizado por acontecimientos económicos de gran relevancia como la pandemia de COVID\sphinxhyphen{}19, la recuperación posterior, los episodios inflacionarios y la guerra en Ucrania, todos ellos con fuerte impacto en los mercados financieros.

\sphinxAtStartPar
En el ámbito geográfico, el estudio se centra en el mercado de valores europeo, tomando como referencia los principales índices bursátiles de países como España (IBEX 35), Alemania (DAX 40), Francia (CAC 40), Italia (FTSE MIB), Países Bajos (AEX), Reino Unido (FTSE 100), Suecia (OMXS30) y Suiza (SMI).

\sphinxAtStartPar
Desde una perspectiva sectorial, el proyecto se focaliza en la información financiera de empresas cotizadas pertenecientes a dichos índices, complementada con métricas de sostenibilidad (ESG) y análisis de noticias económicas. Además, la investigación aborda los cuatro pilares fundamentales del análisis financiero: análisis de sentimiento, análisis técnico, análisis fundamental y análisis cuantitativo, lo que permite una visión integral de los mercados.

\sphinxAtStartPar
No se incluyen en el alcance productos financieros de renta fija, ETFs u otros instrumentos mixtos y empaquetados, debido a las limitaciones de tiempo y disponibilidad de datos, aunque se reconocen como líneas de extensión futura.

\sphinxAtStartPar
\sphinxstylestrong{Delimitación temporal:}

\sphinxAtStartPar
El análisis se desarrollará sobre el periodo comprendido entre enero de 2020 y septiembre de 2025, un intervalo marcado por acontecimientos económicos de gran relevancia —como la pandemia de COVID\sphinxhyphen{}19, la posterior recuperación económica, los repuntes inflacionarios y la guerra en Ucrania— que generaron abundante información periodística y fuertes fluctuaciones en los mercados financieros. No obstante, este marco temporal también presenta limitaciones en cuanto a la disponibilidad de datos: mientras que la información financiera puede recopilarse de manera consistente desde el inicio del periodo, en el caso de las noticias únicamente se dispone de registros desde enero de 2024 hasta la actualidad, lo que impide contar con un histórico más amplio. Estas restricciones condicionan el alcance del análisis, pero al mismo tiempo permiten centrar el estudio en un horizonte temporal reciente y relevante, asegurando la pertinencia de los resultados en relación con las dinámicas económicas actuales.

\sphinxAtStartPar
\sphinxstylestrong{Delimitación técnicos:}

\sphinxAtStartPar
El proyecto presenta diversas limitaciones derivadas tanto de restricciones técnicas como de recursos. En primer lugar, no ha sido posible descargar toda la información financiera disponible, dado que carecen de una estructura estandarizada para todos los tickers, lo que dificulta la homogeneización de datos. Asimismo, el consumo de tokens para la ejecución de modelos de inferencia estuvo condicionado por el uso de una versión gratuita, limitación que con mayores recursos económicos podría suplirse. En el plano de infraestructura, se realizaron tres migraciones del DataLake debido a la imposibilidad de mantener de forma continua un servicio de almacenamiento en la nube de pago como Snowflake. A nivel operativo, se implementó cloudflared como túnel de acceso para la integración de GitHub Actions con n8n en entorno local, lo que, si bien resolvió la conectividad, añade complejidad al flujo de trabajo. Otro aspecto para considerar es el tiempo de ejecución, que restringe la velocidad y eficiencia de descarga de datos a gran escala. Finalmente, los plazos disponibles limitaron la expansión del proyecto, impidiendo la incorporación de otros productos financieros relevantes como renta fija, ETFs, instrumentos mixtos o empaquetados.


\bigskip\hrule\bigskip



\section{Recursos}
\label{\detokenize{DescripcionProblema:recursos}}\begin{itemize}
\item {} 
\sphinxAtStartPar
\sphinxstylestrong{Comparación de estado de resultados financieros}
\begin{itemize}
\item {} 
\sphinxAtStartPar
Fuentes:
\begin{itemize}
\item {} 
\sphinxAtStartPar
Yahoo Finance

\item {} 
\sphinxAtStartPar
TradingView

\end{itemize}

\end{itemize}

\item {} 
\sphinxAtStartPar
\sphinxstylestrong{Noticias (text mining + scrapping):}
\begin{itemize}
\item {} 
\sphinxAtStartPar
ABC

\item {} 
\sphinxAtStartPar
BBC

\item {} 
\sphinxAtStartPar
El País

\item {} 
\sphinxAtStartPar
Expansión

\item {} 
\sphinxAtStartPar
The times

\item {} 
\sphinxAtStartPar
Bloomberg

\item {} 
\sphinxAtStartPar
El economista

\end{itemize}

\item {} 
\sphinxAtStartPar
\sphinxstylestrong{Escalabilidad del producto}

\item {} 
\sphinxAtStartPar
Ambientes:
\begin{itemize}
\item {} 
\sphinxAtStartPar
Visual Studio

\item {} 
\sphinxAtStartPar
Google Colab

\item {} 
\sphinxAtStartPar
n8n

\item {} 
\sphinxAtStartPar
Cloudflared

\item {} 
\sphinxAtStartPar
Tableau

\end{itemize}

\item {} 
\sphinxAtStartPar
Programas
\begin{itemize}
\item {} 
\sphinxAtStartPar
Docker

\item {} 
\sphinxAtStartPar
Snowflake

\item {} 
\sphinxAtStartPar
Google Drive

\end{itemize}

\item {} 
\sphinxAtStartPar
Lenguajes de programación
\begin{itemize}
\item {} 
\sphinxAtStartPar
SQL

\item {} 
\sphinxAtStartPar
Python

\item {} 
\sphinxAtStartPar
JavaScript

\item {} 
\sphinxAtStartPar
HTML

\end{itemize}

\item {} 
\sphinxAtStartPar
Librerías / Frameworks
\begin{itemize}
\item {} 
\sphinxAtStartPar
Pandas

\item {} 
\sphinxAtStartPar
NumPy

\item {} 
\sphinxAtStartPar
Transformers

\item {} 
\sphinxAtStartPar
Torch (PyTorch)

\item {} 
\sphinxAtStartPar
Snowflake Connector

\item {} 
\sphinxAtStartPar
Matplotlib

\item {} 
\sphinxAtStartPar
Seaborn

\item {} 
\sphinxAtStartPar
Plotly (Graph Objects, Express)

\item {} 
\sphinxAtStartPar
Statsmodels

\end{itemize}

\item {} 
\sphinxAtStartPar
Herramientas
\begin{itemize}
\item {} 
\sphinxAtStartPar
Git / GitHub

\item {} 
\sphinxAtStartPar
OpenAI

\item {} 
\sphinxAtStartPar
Hugging Face

\item {} 
\sphinxAtStartPar
Telegram Bots

\end{itemize}

\end{itemize}

\sphinxstepscope


\chapter{Modelos Teóricos}
\label{\detokenize{MarcoTeorico:modelos-teoricos}}\label{\detokenize{MarcoTeorico::doc}}

\section{Modelo de Markowitz}
\label{\detokenize{MarcoTeorico:modelo-de-markowitz}}\label{\detokenize{MarcoTeorico:id1}}
\sphinxAtStartPar
El \sphinxstylestrong{modelo de Markowitz}, también conocido como \sphinxstylestrong{teoría moderna de carteras (Modern Portfolio Theory, MPT)}, fue desarrollado por \sphinxstylestrong{Harry Markowitz en 1952}. Es un modelo matemático de inversión que busca optimizar la asignación de activos en una cartera, equilibrando \sphinxstylestrong{riesgo y rentabilidad esperada}.

\begin{figure}[htbp]
\centering
\capstart

\noindent\sphinxincludegraphics[width=0.800\linewidth]{{capm}.jpg}
\caption{\sphinxstylestrong{Figura 2.} Representación del modelo CAPM (Capital Asset Pricing Model), que ilustra la relación entre el riesgo sistemático (beta) y la rentabilidad esperada de un activo en comparación con el mercado.*}\label{\detokenize{MarcoTeorico:id2}}\end{figure}


\subsection{Ideas principales}
\label{\detokenize{MarcoTeorico:ideas-principales}}\begin{itemize}
\item {} 
\sphinxAtStartPar
\sphinxstylestrong{Diversificación}: Al combinar activos con diferentes comportamientos, se reduce el riesgo total sin sacrificar necesariamente la rentabilidad.

\item {} 
\sphinxAtStartPar
\sphinxstylestrong{Rentabilidad esperada}: cada activo tiene un rendimiento medio esperado.

\item {} 
\sphinxAtStartPar
\sphinxstylestrong{Riesgo}: se mide con la \sphinxstylestrong{varianza o desviación estándar} de los rendimientos.

\item {} 
\sphinxAtStartPar
\sphinxstylestrong{Correlación}: el modelo tiene en cuenta cómo se mueven los activos entre sí. Activos con correlación baja o negativa reducen el riesgo global.

\item {} 
\sphinxAtStartPar
\sphinxstylestrong{Frontera eficiente}: es el conjunto de carteras óptimas que ofrecen la \sphinxstylestrong{mayor rentabilidad para un nivel de riesgo dado} (o el menor riesgo para una rentabilidad deseada).

\end{itemize}


\subsection{Fórmulas}
\label{\detokenize{MarcoTeorico:formulas}}\begin{enumerate}
\sphinxsetlistlabels{\arabic}{enumi}{enumii}{}{.}%
\item {} 
\sphinxAtStartPar
Rentabilidad esperada de la cartera

\end{enumerate}

\sphinxAtStartPar
La rentabilidad media ponderada de los activos que componen la cartera se define como:

\sphinxAtStartPar
\$\$
E(R\_p) = \textbackslash{}sum\_\{i=1\}\textasciicircum{}n w\_i \textbackslash{}cdot E(R\_i)
\$\$

\sphinxAtStartPar
donde:
\begin{itemize}
\item {} 
\sphinxAtStartPar
Wi \(\rightarrow\) peso del activo (i) en la cartera

\item {} 
\sphinxAtStartPar
E(Ri) \(\rightarrow\) rentabilidad esperada del activo (i)

\end{itemize}
\begin{enumerate}
\sphinxsetlistlabels{\arabic}{enumi}{enumii}{}{.}%
\setcounter{enumi}{1}
\item {} 
\sphinxAtStartPar
Varianza de la cartera (riesgo)

\end{enumerate}

\sphinxAtStartPar
El riesgo total de la cartera se mide a través de su varianza:

\sphinxAtStartPar
\$\$
\textbackslash{}sigma\_p\textasciicircum{}2 = \textbackslash{}sum\_\{i=1\}\textasciicircum{}n \textbackslash{}sum\_\{j=1\}\textasciicircum{}n w\_i w\_j \textbackslash{}sigma\_\{ij\}
\$\$

\sphinxAtStartPar
donde:
\begin{itemize}
\item {} 
\sphinxAtStartPar
σ(ij) \(\rightarrow\) covarianza entre los activos (i) y (j)

\item {} 
\sphinxAtStartPar
Si (i = j), entonces σ(ij) corresponde a la varianza del activo (i).

\end{itemize}


\bigskip\hrule\bigskip



\section{Modelo de Valoración de Activos Financieros}
\label{\detokenize{MarcoTeorico:modelo-de-valoracion-de-activos-financieros}}
\sphinxAtStartPar
El \sphinxstylestrong{Modelo de Valoración de Activos Financieros} (Security Market Line (SML)) es una representación gráfica de la relación entre el riesgo sistemático de un activo y su rentabilidad esperada, según el modelo CAPM. Es un instrumento clave para entender cómo el mercado valora (o debería valorar) la compensación por riesgo que exige un inversor.

\sphinxAtStartPar
Usando estas herramientas debemos tomar en cuenta que:
\begin{enumerate}
\sphinxsetlistlabels{\arabic}{enumi}{enumii}{}{.}%
\item {} 
\sphinxAtStartPar
Beta (β): mide la sensibilidad del activo respecto al mercado.

\sphinxAtStartPar
Si β \textgreater{} 1 \(\rightarrow\) el activo es más volátil que el mercado.

\sphinxAtStartPar
Si β \textless{} 1 \(\rightarrow\) menos volátil.

\sphinxAtStartPar
i β = 1 \(\rightarrow\) sensibilidad igual al mercado.

\item {} 
\sphinxAtStartPar
Interpretaciones prácticas:

\end{enumerate}
\begin{itemize}
\item {} 
\sphinxAtStartPar
Si un activo está \sphinxstylestrong{encima de la SML} \(\rightarrow\) está \sphinxstylestrong{infravalorado} (ofrece más rentabilidad de la que debería para su nivel de riesgo).

\item {} 
\sphinxAtStartPar
Si un activo está \sphinxstylestrong{debajo de la SML} \(\rightarrow\) está \sphinxstylestrong{sobrevalorado} (da menos rentabilidad de la que debería para su riesgo).

\item {} 
\sphinxAtStartPar
La SML sirve para comparar activos de \sphinxstylestrong{distinto riesgo sistemático}, y se aplica tanto a acciones como a carteras.

\end{itemize}
\begin{enumerate}
\sphinxsetlistlabels{\arabic}{enumi}{enumii}{}{.}%
\setcounter{enumi}{2}
\item {} 
\sphinxAtStartPar
Supuestos del modelo CAPM / SML:

\end{enumerate}
\begin{itemize}
\item {} 
\sphinxAtStartPar
Inversores racionales, aversos al riesgo, que maximizan la utilidad esperada.

\item {} 
\sphinxAtStartPar
Mercados eficientes (información disponible, sin costes de transacción, etc.).

\item {} 
\sphinxAtStartPar
Existencia de un activo libre de riesgo al que todos tienen acceso.

\item {} 
\sphinxAtStartPar
Sólo el riesgo sistemático (no diversificable) es recompensado; los riesgos idiosincráticos pueden diversificarse.

\end{itemize}

\sphinxAtStartPar
Distribución normal de retornos (o al menos, que las expectativas y varianzas\sphinxhyphen{}covarianzas son suficientes para describir el riesgo esperado).


\subsection{Descripción del gráfico}
\label{\detokenize{MarcoTeorico:descripcion-del-grafico}}\begin{itemize}
\item {} 
\sphinxAtStartPar
\sphinxstylestrong{Eje X:} Riesgo sistemático de un activo, medido por su \sphinxstylestrong{beta (β)}.

\item {} 
\sphinxAtStartPar
\sphinxstylestrong{Eje Y:} Rentabilidad esperada (\sphinxstylestrong{E{[}R{]}}) del activo.

\end{itemize}

\sphinxAtStartPar
El gráfico muestra la \sphinxstylestrong{Security Market Line (SML)}, que representa la relación entre el \sphinxstylestrong{riesgo sistemático} de un activo, medido por la beta βi , y su \sphinxstylestrong{rentabilidad esperada} E(R). En el eje horizontal se sitúa la beta, mientras que en el eje vertical se encuentra la rentabilidad esperada. La recta parte de la \sphinxstylestrong{tasa libre de riesgo} Rf, en este caso un \sphinxstylestrong{3\%}, y su pendiente corresponde a la \sphinxstylestrong{prima de riesgo del mercado} (Rm \sphinxhyphen{} Rf).

\sphinxAtStartPar
Según el modelo \sphinxstylestrong{CAPM}, un activo con βi = 1 debería tener una rentabilidad esperada igual a la del mercado, que en este ejemplo es del \sphinxstylestrong{10\%}. A medida que la beta aumenta, la rentabilidad exigida por los inversores también crece, ya que el activo está más expuesto a los movimientos del mercado. Por el contrario, un activo con βi \textless{} 1 tendría una rentabilidad inferior, porque asume un menor riesgo sistemático.

\begin{figure}[htbp]
\centering
\capstart

\noindent\sphinxincludegraphics[width=0.800\linewidth]{{SML_Graph}.jpg}
\caption{\sphinxstylestrong{Figura 3.} Representación de la \sphinxstylestrong{Security Market Line (SML)}, que muestra la relación entre la beta de los activos y su rentabilidad esperada según el modelo CAPM.*}\label{\detokenize{MarcoTeorico:id3}}\end{figure}


\bigskip\hrule\bigskip



\subsection{Fórmula}
\label{\detokenize{MarcoTeorico:formula}}\begin{equation*}
\begin{split}E(R_i) = R_f + \beta_i \cdot (R_m - R_f)\end{split}
\end{equation*}
\sphinxAtStartPar
donde:
\begin{itemize}
\item {} 
\sphinxAtStartPar
E(Ri): rentabilidad esperada del activo \sphinxstyleemphasis{i}.

\item {} 
\sphinxAtStartPar
Rf: tasa libre de riesgo (ej. bonos del Estado).

\item {} 
\sphinxAtStartPar
βi: sensibilidad del activo frente al mercado.

\item {} 
\sphinxAtStartPar
Rm: rentabilidad esperada del mercado.

\item {} 
\sphinxAtStartPar
(Rm \sphinxhyphen{} Rf): prima de riesgo del mercado.

\end{itemize}

\sphinxstepscope


\chapter{Datos y Preparación}
\label{\detokenize{DatosPreparacion:datos-y-preparacion}}\label{\detokenize{DatosPreparacion::doc}}
\sphinxAtStartPar
Este trabajo integra dos fuentes complementarias para anticipar el comportamiento del precio de acciones europeas: señales de sentimiento extraídas de titulares de prensa y información cuantitativa de mercado y estados financieros. Los titulares se obtuvieron mediante scraping y de medios generalistas y económicos —BBC, ABC, El Economista, Bloomberg Europa, El País, The Times y Expansión—. Sobre estos textos se aplicó una depuración sistemática (eliminación de duplicados y de ruidos como elementos de navegación, normalización de caracteres/idiomas y control de fechas) para construir indicadores diarios de sentimiento por medio e instrumento.

\sphinxAtStartPar
En paralelo, los datos de mercado (OHLCV por ticker) se descargaron y consolidaron en Snowflake, armonizando formatos de fecha y símbolos, verificando huecos y coherencia (festivos, sesiones sin volumen) y, cuando procede, incorporando información contable resumida por compañía. Finalmente, se integraron ambas fuentes en un único dataset analítico a nivel (ticker, fecha) que incluye rendimientos y rezagos, volumen y métricas de sentimiento agregadas. Este conjunto sirve de base para los modelos predictivos, combinando la dimensión informativa de las noticias con la evidencia numérica del mercado para obtener señales de compra/venta más robustas.

\begin{figure}[htbp]
\centering
\capstart

\noindent\sphinxincludegraphics[width=1.000\linewidth]{{FlujoScraping}.jpeg}
\caption{\sphinxstylestrong{Figura 5.} Flujograma del modelo de scrapping}\label{\detokenize{DatosPreparacion:id1}}\end{figure}


\section{Titulares de noticieros}
\label{\detokenize{DatosPreparacion:titulares-de-noticieros}}
\sphinxAtStartPar
Como primera fase, se llevó a cabo la recopilación de titulares procedentes de noticieros financieros y generalistas de acceso público mediante sus plataformas web y archivos digitales, los cuales fueron posteriormente procesados a través de técnicas de análisis de texto con el fin de filtrar e identificar aquellas noticias con mayor relevancia y potencial de impacto en el comportamiento bursátil de las acciones.

\begin{figure}[htbp]
\centering
\capstart

\noindent\sphinxincludegraphics[width=0.800\linewidth]{{thetimes}.png}
\caption{\sphinxstylestrong{Figura 6.} Representación de noticieros extraidos}\label{\detokenize{DatosPreparacion:id2}}\end{figure}

\sphinxAtStartPar
Se ha creado un \sphinxstylestrong{orquestador} con el objetivo de coordinar el proceso de \sphinxstyleemphasis{scraping} por cada medio de comunicación y la posterior carga de los datos en \sphinxstylestrong{Snowflake}, evitando la ejecución manual de código desde la documentación. Este flujo tiene como finalidad construir una base de datos de noticias relacionadas con las empresas que cotizan en las principales bolsas de valores europeas.   Para la orquestación se ha utilizado la herramienta \sphinxstylestrong{n8n}, una plataforma de automatización en la que se centraliza el código encargado de realizar las solicitudes a las distintas API’s y de aplicar los modelos de análisis de sentimiento.

\begin{figure}[htbp]
\centering
\capstart

\noindent\sphinxincludegraphics[width=1.100\linewidth]{{DigSentimientos}.png}
\caption{\sphinxstylestrong{Figura 7.} Flujo de proceso de scrapping}\label{\detokenize{DatosPreparacion:id3}}\end{figure}

\sphinxAtStartPar
Este modelo genera una probabilidad asociada a cada noticia, lo que permite clasificarla como \sphinxstylestrong{positiva}, \sphinxstylestrong{negativa} o \sphinxstylestrong{neutral}. Aunque el detalle de la aplicación de los modelos se desarrolla en una sección posterior, en este apartado se describe el procedimiento de descarga de titulares desde el scrapping y la construcción del \sphinxstyleemphasis{data lake} en \sphinxstylestrong{Snowflake}.

\begin{figure}[htbp]
\centering
\capstart

\noindent\sphinxincludegraphics[width=1.000\linewidth]{{Snowflake_noticias}.png}
\caption{\sphinxstylestrong{Figura 8.} Base de datos en \sphinxstylestrong{Snowflake}}\label{\detokenize{DatosPreparacion:fig-snowflake-noticias}}\end{figure}


\bigskip\hrule\bigskip



\subsection{Módulos y dependencias}
\label{\detokenize{DatosPreparacion:modulos-y-dependencias}}\begin{itemize}
\item {} 
\sphinxAtStartPar
\sphinxstylestrong{\sphinxcode{\sphinxupquote{config.py}}}
\begin{itemize}
\item {} 
\sphinxAtStartPar
NOTICIEROS: lista de medios (URL, fuente, idioma, tabla destino)

\item {} 
\sphinxAtStartPar
SNOWFLAKE\_CONFIG: credenciales y parámetros de conexión

\item {} 
\sphinxAtStartPar
RETRIES, SLEEP\_BETWEEN\_DIAS: reintentos y pausas

\end{itemize}

\item {} 
\sphinxAtStartPar
\sphinxstylestrong{\sphinxcode{\sphinxupquote{scraper.py}}}
\begin{itemize}
\item {} 
\sphinxAtStartPar
obtener\_snapshot\_url\_directo(url, fecha)

\item {} 
\sphinxAtStartPar
extraer\_titulares(url, fecha, fuente)

\item {} 
\sphinxAtStartPar
log\_error(msg)

\end{itemize}

\item {} 
\sphinxAtStartPar
\sphinxstylestrong{\sphinxcode{\sphinxupquote{snowflake\_utils.py}}}
\begin{itemize}
\item {} 
\sphinxAtStartPar
obtener\_ultima\_fecha\_en\_snowflake(config, tabla)

\item {} 
\sphinxAtStartPar
subir\_a\_snowflake(df, config, tabla)

\end{itemize}

\end{itemize}


\bigskip\hrule\bigskip



\subsection{Flujo general}
\label{\detokenize{DatosPreparacion:flujo-general}}\begin{enumerate}
\sphinxsetlistlabels{\arabic}{enumi}{enumii}{}{.}%
\item {} 
\sphinxAtStartPar
Para cada noticiero en \sphinxcode{\sphinxupquote{config.py}}, se obtiene la última fecha almacenada.

\item {} 
\sphinxAtStartPar
Se recorren los días pendientes hasta el día anterior al actual.

\item {} 
\sphinxAtStartPar
Para cada fecha: se resuelve el snapshot, se extraen titulares y se enriquecen con metadatos.

\item {} 
\sphinxAtStartPar
Se suben los nuevos registros a Snowflake.

\end{enumerate}


\bigskip\hrule\bigskip



\subsection{Preparación del procesamiento por medio}
\label{\detokenize{DatosPreparacion:preparacion-del-procesamiento-por-medio}}
\sphinxAtStartPar
En primer lugar, es necesario verificar la última fecha de carga en \sphinxstylestrong{Snowflake}, tomando como referencia el período comprendido desde \sphinxstylestrong{enero de 2024 hasta el día inmediatamente anterior}. De esta manera se garantiza la continuidad del proceso y se evitan posibles duplicidades. Este enfoque asegura la disponibilidad de comentarios recientes, lo que permite realizar un análisis actualizado y contribuye a mejorar la capacidad de predicción sobre la evolución del valor de las acciones.

\begin{sphinxVerbatim}[commandchars=\\\{\},numbers=left,firstnumber=1,stepnumber=1]
\PYG{k}{def}\PYG{+w}{ }\PYG{n+nf}{obtener\PYGZus{}ultima\PYGZus{}fecha\PYGZus{}en\PYGZus{}snowflake}\PYG{p}{(}\PYG{n}{config}\PYG{p}{,} \PYG{n}{tabla}\PYG{p}{)}\PYG{p}{:}
    \PYG{n}{ctx} \PYG{o}{=} \PYG{n}{snowflake}\PYG{o}{.}\PYG{n}{connector}\PYG{o}{.}\PYG{n}{connect}\PYG{p}{(}
        \PYG{n}{user}\PYG{o}{=}\PYG{n}{config}\PYG{p}{[}\PYG{l+s+s1}{\PYGZsq{}}\PYG{l+s+s1}{user}\PYG{l+s+s1}{\PYGZsq{}}\PYG{p}{]}\PYG{p}{,}
        \PYG{n}{password}\PYG{o}{=}\PYG{n}{config}\PYG{p}{[}\PYG{l+s+s1}{\PYGZsq{}}\PYG{l+s+s1}{password}\PYG{l+s+s1}{\PYGZsq{}}\PYG{p}{]}\PYG{p}{,}
        \PYG{n}{account}\PYG{o}{=}\PYG{n}{config}\PYG{p}{[}\PYG{l+s+s1}{\PYGZsq{}}\PYG{l+s+s1}{account}\PYG{l+s+s1}{\PYGZsq{}}\PYG{p}{]}\PYG{p}{,}
        \PYG{n}{warehouse}\PYG{o}{=}\PYG{n}{config}\PYG{p}{[}\PYG{l+s+s1}{\PYGZsq{}}\PYG{l+s+s1}{warehouse}\PYG{l+s+s1}{\PYGZsq{}}\PYG{p}{]}\PYG{p}{,}
        \PYG{n}{database}\PYG{o}{=}\PYG{n}{config}\PYG{p}{[}\PYG{l+s+s1}{\PYGZsq{}}\PYG{l+s+s1}{database}\PYG{l+s+s1}{\PYGZsq{}}\PYG{p}{]}\PYG{p}{,}
        \PYG{n}{schema}\PYG{o}{=}\PYG{n}{config}\PYG{p}{[}\PYG{l+s+s1}{\PYGZsq{}}\PYG{l+s+s1}{schema}\PYG{l+s+s1}{\PYGZsq{}}\PYG{p}{]}
    \PYG{p}{)}
    \PYG{n}{cs} \PYG{o}{=} \PYG{n}{ctx}\PYG{o}{.}\PYG{n}{cursor}\PYG{p}{(}\PYG{p}{)}
    \PYG{k}{try}\PYG{p}{:}
        \PYG{n}{tabla\PYGZus{}completa} \PYG{o}{=} \PYG{l+s+sa}{f}\PYG{l+s+s2}{\PYGZdq{}}\PYG{l+s+si}{\PYGZob{}}\PYG{n}{config}\PYG{p}{[}\PYG{l+s+s1}{\PYGZsq{}}\PYG{l+s+s1}{database}\PYG{l+s+s1}{\PYGZsq{}}\PYG{p}{]}\PYG{l+s+si}{\PYGZcb{}}\PYG{l+s+s2}{.}\PYG{l+s+si}{\PYGZob{}}\PYG{n}{config}\PYG{p}{[}\PYG{l+s+s1}{\PYGZsq{}}\PYG{l+s+s1}{schema}\PYG{l+s+s1}{\PYGZsq{}}\PYG{p}{]}\PYG{l+s+si}{\PYGZcb{}}\PYG{l+s+s2}{.}\PYG{l+s+si}{\PYGZob{}}\PYG{n}{tabla}\PYG{l+s+si}{\PYGZcb{}}\PYG{l+s+s2}{\PYGZdq{}}
        \PYG{n}{cs}\PYG{o}{.}\PYG{n}{execute}\PYG{p}{(}\PYG{l+s+sa}{f}\PYG{l+s+s2}{\PYGZdq{}}\PYG{l+s+s2}{SELECT MAX(fecha) FROM }\PYG{l+s+si}{\PYGZob{}}\PYG{n}{tabla\PYGZus{}completa}\PYG{l+s+si}{\PYGZcb{}}\PYG{l+s+s2}{\PYGZdq{}}\PYG{p}{)}
        \PYG{n}{resultado} \PYG{o}{=} \PYG{n}{cs}\PYG{o}{.}\PYG{n}{fetchone}\PYG{p}{(}\PYG{p}{)}
        \PYG{k}{if} \PYG{n}{resultado} \PYG{o+ow}{and} \PYG{n}{resultado}\PYG{p}{[}\PYG{l+m+mi}{0}\PYG{p}{]}\PYG{p}{:}
            \PYG{n}{ultima\PYGZus{}fecha} \PYG{o}{=} \PYG{n}{resultado}\PYG{p}{[}\PYG{l+m+mi}{0}\PYG{p}{]}
            \PYG{n+nb}{print}\PYG{p}{(}\PYG{l+s+sa}{f}\PYG{l+s+s2}{\PYGZdq{}}\PYG{l+s+s2}{Última fecha en Snowflake para }\PYG{l+s+si}{\PYGZob{}}\PYG{n}{tabla}\PYG{l+s+si}{\PYGZcb{}}\PYG{l+s+s2}{: }\PYG{l+s+si}{\PYGZob{}}\PYG{n}{ultima\PYGZus{}fecha}\PYG{l+s+si}{\PYGZcb{}}\PYG{l+s+s2}{\PYGZdq{}}\PYG{p}{)}
            \PYG{k}{return} \PYG{n}{ultima\PYGZus{}fecha} \PYG{o}{+} \PYG{n}{timedelta}\PYG{p}{(}\PYG{n}{days}\PYG{o}{=}\PYG{l+m+mi}{1}\PYG{p}{)}
        \PYG{k}{else}\PYG{p}{:}
            \PYG{n+nb}{print}\PYG{p}{(}\PYG{l+s+sa}{f}\PYG{l+s+s2}{\PYGZdq{}}\PYG{l+s+s2}{No se encontraron registros en }\PYG{l+s+si}{\PYGZob{}}\PYG{n}{tabla}\PYG{l+s+si}{\PYGZcb{}}\PYG{l+s+s2}{. Iniciando desde 2024\PYGZhy{}01\PYGZhy{}01.}\PYG{l+s+s2}{\PYGZdq{}}\PYG{p}{)}
            \PYG{k}{return} \PYG{n}{datetime}\PYG{o}{.}\PYG{n}{strptime}\PYG{p}{(}\PYG{l+s+s2}{\PYGZdq{}}\PYG{l+s+s2}{20240101}\PYG{l+s+s2}{\PYGZdq{}}\PYG{p}{,} \PYG{l+s+s2}{\PYGZdq{}}\PYG{l+s+s2}{\PYGZpc{}}\PYG{l+s+s2}{Y}\PYG{l+s+s2}{\PYGZpc{}}\PYG{l+s+s2}{m}\PYG{l+s+si}{\PYGZpc{}d}\PYG{l+s+s2}{\PYGZdq{}}\PYG{p}{)}\PYG{o}{.}\PYG{n}{date}\PYG{p}{(}\PYG{p}{)}
    \PYG{k}{finally}\PYG{p}{:}
        \PYG{n}{cs}\PYG{o}{.}\PYG{n}{close}\PYG{p}{(}\PYG{p}{)}
        \PYG{n}{ctx}\PYG{o}{.}\PYG{n}{close}\PYG{p}{(}\PYG{p}{)}
\end{sphinxVerbatim}


\bigskip\hrule\bigskip


\sphinxAtStartPar
Adicionalmente, es necesario extraer las URL correctas por cada noticiero. Para ello, se ha desarrollado la función \sphinxstylestrong{\sphinxcode{\sphinxupquote{obtener\_snapshot\_url(original\_url, fecha\_str)}}}, la cual consulta la API de \sphinxstylestrong{Wayback Machine} con el objetivo de recuperar la versión archivada más cercana de una página web en una fecha determinada. El procedimiento consiste en construir la URL de la API a partir de la dirección original y la fecha solicitada, realizar una petición GET y procesar la respuesta en formato \sphinxstylestrong{JSON}.\\
Si existe un \sphinxstyleemphasis{snapshot} disponible, la función devuelve la URL más próxima a la fecha indicada en formato https; en caso contrario, retorna None mostrando un mensaje informativo.

\begin{sphinxVerbatim}[commandchars=\\\{\},numbers=left,firstnumber=1,stepnumber=1]
\PYG{k}{def}\PYG{+w}{ }\PYG{n+nf}{obtener\PYGZus{}snapshot\PYGZus{}url}\PYG{p}{(}\PYG{n}{original\PYGZus{}url}\PYG{p}{,} \PYG{n}{fecha\PYGZus{}str}\PYG{p}{)}\PYG{p}{:}
    \PYG{n}{wayback\PYGZus{}api} \PYG{o}{=} \PYG{l+s+sa}{f}\PYG{l+s+s1}{\PYGZsq{}}\PYG{l+s+s1}{https://archive.org/wayback/available?url=}\PYG{l+s+si}{\PYGZob{}}\PYG{n}{original\PYGZus{}url}\PYG{l+s+si}{\PYGZcb{}}\PYG{l+s+s1}{\PYGZam{}timestamp=}\PYG{l+s+si}{\PYGZob{}}\PYG{n}{fecha\PYGZus{}str}\PYG{l+s+si}{\PYGZcb{}}\PYG{l+s+s1}{\PYGZsq{}}
    \PYG{k}{try}\PYG{p}{:}
        \PYG{n}{res} \PYG{o}{=} \PYG{n}{requests}\PYG{o}{.}\PYG{n}{get}\PYG{p}{(}\PYG{n}{wayback\PYGZus{}api}\PYG{p}{,} \PYG{n}{timeout}\PYG{o}{=}\PYG{n}{WAYBACK\PYGZus{}TIMEOUT}\PYG{p}{)}
        \PYG{n}{data} \PYG{o}{=} \PYG{n}{res}\PYG{o}{.}\PYG{n}{json}\PYG{p}{(}\PYG{p}{)}
        \PYG{n}{snapshots} \PYG{o}{=} \PYG{n}{data}\PYG{o}{.}\PYG{n}{get}\PYG{p}{(}\PYG{l+s+s1}{\PYGZsq{}}\PYG{l+s+s1}{archived\PYGZus{}snapshots}\PYG{l+s+s1}{\PYGZsq{}}\PYG{p}{,} \PYG{p}{\PYGZob{}}\PYG{p}{\PYGZcb{}}\PYG{p}{)}
        \PYG{k}{if} \PYG{n}{snapshots} \PYG{o+ow}{and} \PYG{l+s+s1}{\PYGZsq{}}\PYG{l+s+s1}{closest}\PYG{l+s+s1}{\PYGZsq{}} \PYG{o+ow}{in} \PYG{n}{snapshots}\PYG{p}{:}
            \PYG{n}{url\PYGZus{}snapshot} \PYG{o}{=} \PYG{n}{snapshots}\PYG{p}{[}\PYG{l+s+s1}{\PYGZsq{}}\PYG{l+s+s1}{closest}\PYG{l+s+s1}{\PYGZsq{}}\PYG{p}{]}\PYG{p}{[}\PYG{l+s+s1}{\PYGZsq{}}\PYG{l+s+s1}{url}\PYG{l+s+s1}{\PYGZsq{}}\PYG{p}{]}
            \PYG{k}{return} \PYG{n}{url\PYGZus{}snapshot}\PYG{o}{.}\PYG{n}{replace}\PYG{p}{(}\PYG{l+s+s2}{\PYGZdq{}}\PYG{l+s+s2}{http://}\PYG{l+s+s2}{\PYGZdq{}}\PYG{p}{,} \PYG{l+s+s2}{\PYGZdq{}}\PYG{l+s+s2}{https://}\PYG{l+s+s2}{\PYGZdq{}}\PYG{p}{)}
        \PYG{k}{else}\PYG{p}{:}
            \PYG{n+nb}{print}\PYG{p}{(}\PYG{l+s+sa}{f}\PYG{l+s+s2}{\PYGZdq{}}\PYG{l+s+s2}{No snapshot disponible para }\PYG{l+s+si}{\PYGZob{}}\PYG{n}{original\PYGZus{}url}\PYG{l+s+si}{\PYGZcb{}}\PYG{l+s+s2}{ en }\PYG{l+s+si}{\PYGZob{}}\PYG{n}{fecha\PYGZus{}str}\PYG{l+s+si}{\PYGZcb{}}\PYG{l+s+s2}{\PYGZdq{}}\PYG{p}{)}
            \PYG{k}{return} \PYG{k+kc}{None}
    \PYG{k}{except} \PYG{n+ne}{Exception} \PYG{k}{as} \PYG{n}{e}\PYG{p}{:}
        \PYG{n}{log\PYGZus{}error}\PYG{p}{(}\PYG{l+s+sa}{f}\PYG{l+s+s2}{\PYGZdq{}}\PYG{l+s+s2}{Error consultando Wayback API para }\PYG{l+s+si}{\PYGZob{}}\PYG{n}{original\PYGZus{}url}\PYG{l+s+si}{\PYGZcb{}}\PYG{l+s+s2}{ en }\PYG{l+s+si}{\PYGZob{}}\PYG{n}{fecha\PYGZus{}str}\PYG{l+s+si}{\PYGZcb{}}\PYG{l+s+s2}{: }\PYG{l+s+si}{\PYGZob{}}\PYG{n}{e}\PYG{l+s+si}{\PYGZcb{}}\PYG{l+s+s2}{\PYGZdq{}}\PYG{p}{)}
        \PYG{k}{return} \PYG{k+kc}{None}
\end{sphinxVerbatim}


\bigskip\hrule\bigskip



\subsection{Extracción de titulares}
\label{\detokenize{DatosPreparacion:extraccion-de-titulares}}
\sphinxAtStartPar
Una vez verificada la última fecha de carga y obtenidas las URL de las API’s, se procede con la solicitud de la información. Ahora se crea la función está diseñada para obtener los titulares de noticias desde una página web archivada en la Wayback Machine. Para ello, accede al contenido del snapshot, identifica los encabezados relevantes dentro del documento HTML y aplica reglas de filtrado para asegurar que los textos extraídos correspondan efectivamente a titulares. En función del medio, se utilizan criterios específicos de validación, y los resultados se almacenan en una lista estructurada.

\sphinxAtStartPar
Módulo de descarga en API’s

\begin{sphinxVerbatim}[commandchars=\\\{\},numbers=left,firstnumber=1,stepnumber=1]
\PYG{k}{def}\PYG{+w}{ }\PYG{n+nf}{extraer\PYGZus{}titulares}\PYG{p}{(}\PYG{n}{snapshot\PYGZus{}url}\PYG{p}{,} \PYG{n}{fecha\PYGZus{}str}\PYG{p}{,} \PYG{n}{fuente}\PYG{o}{=}\PYG{k+kc}{None}\PYG{p}{)}\PYG{p}{:}
    \PYG{n}{titulares} \PYG{o}{=} \PYG{p}{[}\PYG{p}{]}
    \PYG{k}{try}\PYG{p}{:}
        \PYG{n}{res} \PYG{o}{=} \PYG{n}{requests}\PYG{o}{.}\PYG{n}{get}\PYG{p}{(}\PYG{n}{snapshot\PYGZus{}url}\PYG{p}{,} \PYG{n}{timeout}\PYG{o}{=}\PYG{n}{SNAPSHOT\PYGZus{}TIMEOUT}\PYG{p}{)}
        \PYG{n}{soup} \PYG{o}{=} \PYG{n}{BeautifulSoup}\PYG{p}{(}\PYG{n}{res}\PYG{o}{.}\PYG{n}{content}\PYG{p}{,} \PYG{l+s+s1}{\PYGZsq{}}\PYG{l+s+s1}{html.parser}\PYG{l+s+s1}{\PYGZsq{}}\PYG{p}{)}
        \PYG{n}{encabezados} \PYG{o}{=} \PYG{n}{soup}\PYG{o}{.}\PYG{n}{find\PYGZus{}all}\PYG{p}{(}\PYG{p}{[}\PYG{l+s+s1}{\PYGZsq{}}\PYG{l+s+s1}{h1}\PYG{l+s+s1}{\PYGZsq{}}\PYG{p}{,} \PYG{l+s+s1}{\PYGZsq{}}\PYG{l+s+s1}{h2}\PYG{l+s+s1}{\PYGZsq{}}\PYG{p}{,} \PYG{l+s+s1}{\PYGZsq{}}\PYG{l+s+s1}{h3}\PYG{l+s+s1}{\PYGZsq{}}\PYG{p}{]}\PYG{p}{)}
        \PYG{n+nb}{print}\PYG{p}{(}\PYG{l+s+sa}{f}\PYG{l+s+s2}{\PYGZdq{}}\PYG{l+s+s2}{[}\PYG{l+s+si}{\PYGZob{}}\PYG{n}{fuente}\PYG{l+s+si}{\PYGZcb{}}\PYG{l+s+s2}{] }\PYG{l+s+si}{\PYGZob{}}\PYG{n+nb}{len}\PYG{p}{(}\PYG{n}{encabezados}\PYG{p}{)}\PYG{l+s+si}{\PYGZcb{}}\PYG{l+s+s2}{ encabezados encontrados en }\PYG{l+s+si}{\PYGZob{}}\PYG{n}{snapshot\PYGZus{}url}\PYG{l+s+si}{\PYGZcb{}}\PYG{l+s+s2}{\PYGZdq{}}\PYG{p}{)}

        \PYG{k}{for} \PYG{n}{t} \PYG{o+ow}{in} \PYG{n}{encabezados}\PYG{p}{:}
            \PYG{n}{texto} \PYG{o}{=} \PYG{n}{t}\PYG{o}{.}\PYG{n}{get\PYGZus{}text}\PYG{p}{(}\PYG{n}{strip}\PYG{o}{=}\PYG{k+kc}{True}\PYG{p}{)}
            \PYG{n}{clases} \PYG{o}{=} \PYG{l+s+s2}{\PYGZdq{}}\PYG{l+s+s2}{ }\PYG{l+s+s2}{\PYGZdq{}}\PYG{o}{.}\PYG{n}{join}\PYG{p}{(}\PYG{n}{t}\PYG{o}{.}\PYG{n}{get}\PYG{p}{(}\PYG{l+s+s1}{\PYGZsq{}}\PYG{l+s+s1}{class}\PYG{l+s+s1}{\PYGZsq{}}\PYG{p}{,} \PYG{p}{[}\PYG{p}{]}\PYG{p}{)}\PYG{p}{)} \PYG{k}{if} \PYG{n}{t}\PYG{o}{.}\PYG{n}{get}\PYG{p}{(}\PYG{l+s+s1}{\PYGZsq{}}\PYG{l+s+s1}{class}\PYG{l+s+s1}{\PYGZsq{}}\PYG{p}{)} \PYG{k}{else} \PYG{l+s+s2}{\PYGZdq{}}\PYG{l+s+s2}{\PYGZdq{}}

            \PYG{k}{if} \PYG{n}{fuente} \PYG{o}{==} \PYG{l+s+s2}{\PYGZdq{}}\PYG{l+s+s2}{THE TIMES}\PYG{l+s+s2}{\PYGZdq{}}\PYG{p}{:}
                \PYG{k}{if} \PYG{n+nb}{any}\PYG{p}{(}\PYG{n+nb+bp}{cls} \PYG{o+ow}{in} \PYG{n}{clases} \PYG{k}{for} \PYG{n+nb+bp}{cls} \PYG{o+ow}{in} \PYG{p}{[}
                    \PYG{l+s+s1}{\PYGZsq{}}\PYG{l+s+s1}{responsive\PYGZus{}\PYGZus{}HeadlineContainer\PYGZhy{}sc\PYGZhy{}3t8ix5\PYGZhy{}3}\PYG{l+s+s1}{\PYGZsq{}}\PYG{p}{,}
                    \PYG{l+s+s1}{\PYGZsq{}}\PYG{l+s+s1}{responsive\PYGZus{}\PYGZus{}Heading\PYGZhy{}sc\PYGZhy{}1k9kzho\PYGZhy{}1}\PYG{l+s+s1}{\PYGZsq{}}\PYG{p}{,}
                    \PYG{l+s+s1}{\PYGZsq{}}\PYG{l+s+s1}{responsive\PYGZus{}\PYGZus{}Title\PYGZhy{}sc\PYGZhy{}1ij0d4n\PYGZhy{}5}\PYG{l+s+s1}{\PYGZsq{}}
                \PYG{p}{]}\PYG{p}{)} \PYG{o+ow}{or} \PYG{n+nb}{len}\PYG{p}{(}\PYG{n}{texto}\PYG{o}{.}\PYG{n}{split}\PYG{p}{(}\PYG{p}{)}\PYG{p}{)} \PYG{o}{\PYGZgt{}} \PYG{l+m+mi}{3}\PYG{p}{:}
                    \PYG{n}{titulares}\PYG{o}{.}\PYG{n}{append}\PYG{p}{(}\PYG{p}{\PYGZob{}}
                        \PYG{l+s+s2}{\PYGZdq{}}\PYG{l+s+s2}{fecha}\PYG{l+s+s2}{\PYGZdq{}}\PYG{p}{:} \PYG{n}{fecha\PYGZus{}str}\PYG{p}{,}
                        \PYG{l+s+s2}{\PYGZdq{}}\PYG{l+s+s2}{titular}\PYG{l+s+s2}{\PYGZdq{}}\PYG{p}{:} \PYG{n}{texto}\PYG{p}{,}
                        \PYG{l+s+s2}{\PYGZdq{}}\PYG{l+s+s2}{url\PYGZus{}archivo}\PYG{l+s+s2}{\PYGZdq{}}\PYG{p}{:} \PYG{n}{snapshot\PYGZus{}url}
                    \PYG{p}{\PYGZcb{}}\PYG{p}{)}
            \PYG{k}{else}\PYG{p}{:}
                \PYG{k}{if} \PYG{n}{texto} \PYG{o+ow}{and} \PYG{n+nb}{len}\PYG{p}{(}\PYG{n}{texto}\PYG{o}{.}\PYG{n}{split}\PYG{p}{(}\PYG{p}{)}\PYG{p}{)} \PYG{o}{\PYGZgt{}} \PYG{l+m+mi}{3}\PYG{p}{:}
                    \PYG{n}{titulares}\PYG{o}{.}\PYG{n}{append}\PYG{p}{(}\PYG{p}{\PYGZob{}}
                        \PYG{l+s+s2}{\PYGZdq{}}\PYG{l+s+s2}{fecha}\PYG{l+s+s2}{\PYGZdq{}}\PYG{p}{:} \PYG{n}{fecha\PYGZus{}str}\PYG{p}{,}
                        \PYG{l+s+s2}{\PYGZdq{}}\PYG{l+s+s2}{titular}\PYG{l+s+s2}{\PYGZdq{}}\PYG{p}{:} \PYG{n}{texto}\PYG{p}{,}
                        \PYG{l+s+s2}{\PYGZdq{}}\PYG{l+s+s2}{url\PYGZus{}archivo}\PYG{l+s+s2}{\PYGZdq{}}\PYG{p}{:} \PYG{n}{snapshot\PYGZus{}url}
                    \PYG{p}{\PYGZcb{}}\PYG{p}{)}

    \PYG{k}{except} \PYG{n+ne}{Exception} \PYG{k}{as} \PYG{n}{e}\PYG{p}{:}
        \PYG{n}{log\PYGZus{}error}\PYG{p}{(}\PYG{l+s+sa}{f}\PYG{l+s+s2}{\PYGZdq{}}\PYG{l+s+s2}{[}\PYG{l+s+si}{\PYGZob{}}\PYG{n}{fuente}\PYG{+w}{ }\PYG{o+ow}{or}\PYG{+w}{ }\PYG{l+s+s1}{\PYGZsq{}}\PYG{l+s+s1}{GENERAL}\PYG{l+s+s1}{\PYGZsq{}}\PYG{l+s+si}{\PYGZcb{}}\PYG{l+s+s2}{] Error accediendo a snapshot: }\PYG{l+s+si}{\PYGZob{}}\PYG{n}{e}\PYG{l+s+si}{\PYGZcb{}}\PYG{l+s+s2}{\PYGZdq{}}\PYG{p}{)}

    \PYG{k}{return} \PYG{n}{titulares}
\end{sphinxVerbatim}


\bigskip\hrule\bigskip



\subsection{Carga del DataFrame de titulares de noticieros en el DataLake}
\label{\detokenize{DatosPreparacion:carga-del-dataframe-de-titulares-de-noticieros-en-el-datalake}}
\sphinxAtStartPar
Por último, se valida si el \sphinxstylestrong{DataFrame} contiene información; en caso de estar vacío, no se ejecuta ninguna acción.  En caso contrario, se transforma la columna de fecha al tipo de dato adecuado y se establece la conexión con \sphinxstylestrong{Snowflake} mediante las credenciales configuradas en el entorno.

\sphinxAtStartPar
Si la tabla de destino no existe, esta se crea con un esquema predefinido que incluye los campos: \sphinxstylestrong{fecha}, \sphinxstylestrong{titular de la noticia}, \sphinxstylestrong{URL del snapshot}, \sphinxstylestrong{fuente} e \sphinxstylestrong{idioma}. Posteriormente, se realiza una inserción en bloque de todos los registros del DataFrame, lo que permite almacenar de forma eficiente los titulares extraídos de los medios archivados, garantizando su disponibilidad para procesos posteriores de análisis o visualización.


\begin{savenotes}\sphinxattablestart
\sphinxthistablewithglobalstyle
\centering
\begin{tabulary}{\linewidth}[t]{TTTT}
\sphinxtoprule
\sphinxstyletheadfamily 
\sphinxAtStartPar
ID
&\sphinxstyletheadfamily 
\sphinxAtStartPar
FECHA
&\sphinxstyletheadfamily 
\sphinxAtStartPar
TITULAR
&\sphinxstyletheadfamily 
\sphinxAtStartPar
URL\_ARCHIVO
\\
\sphinxmidrule
\sphinxtableatstartofbodyhook
\sphinxAtStartPar
cfee822f\sphinxhyphen{}4744\sphinxhyphen{}413c\sphinxhyphen{}b175\sphinxhyphen{}06b74a31b269
&
\sphinxAtStartPar
09/15/25
&
\sphinxAtStartPar
Why India’s Supreme Court is “a men’s club”
&
\sphinxAtStartPar
\sphinxhref{https://web.archive.org/web/20250915120000/https://www.bbc.com/news/}{Link}
\\
\sphinxhline
\sphinxAtStartPar
27a933d1\sphinxhyphen{}634f\sphinxhyphen{}442e\sphinxhyphen{}aa1f\sphinxhyphen{}e47bfdad666d
&
\sphinxAtStartPar
09/15/25
&
\sphinxAtStartPar
Pilar 2 y el papel de la OCDE en el panorama fiscal internacional
&
\sphinxAtStartPar
\sphinxhref{https://web.archive.org/web/20250915120000/https://www.expansion.com/}{Link}
\\
\sphinxhline
\sphinxAtStartPar
c1d7c991\sphinxhyphen{}e8cd\sphinxhyphen{}4e90\sphinxhyphen{}955f\sphinxhyphen{}1c5fe8456739
&
\sphinxAtStartPar
09/15/25
&
\sphinxAtStartPar
Así fomenta el empleo el mercado de vehículos de ocasión
&
\sphinxAtStartPar
\sphinxhref{https://web.archive.org/web/20250915120000/https://elpais.com/economia/}{Link}
\\
\sphinxbottomrule
\end{tabulary}
\sphinxtableafterendhook\par
\sphinxattableend\end{savenotes}


\bigskip\hrule\bigskip



\section{Precios por tickers}
\label{\detokenize{DatosPreparacion:precios-por-tickers}}
\sphinxAtStartPar
En el análisis financiero, en primer lugar, se necesitan los tickers sobre los que se realizará el estudio. Para este trabajo se han recopilado los tickers de bolsas de valores europeas, que se muestran a continuación. En total son ocho, cada una con sus respectivos tickers, y el análisis se limitará únicamente a las acciones que forman parte de los índices de dichos mercados.


\begin{savenotes}\sphinxattablestart
\sphinxthistablewithglobalstyle
\centering
\begin{tabulary}{\linewidth}[t]{TTTTT}
\sphinxtoprule
\sphinxstyletheadfamily 
\sphinxAtStartPar
País
&\sphinxstyletheadfamily 
\sphinxAtStartPar
Índice
&\sphinxstyletheadfamily 
\sphinxAtStartPar
Exchange Aceptado
&\sphinxstyletheadfamily 
\sphinxAtStartPar
Sufijo Yahoo
&\sphinxstyletheadfamily 
\sphinxAtStartPar
Número Esperado
\\
\sphinxmidrule
\sphinxtableatstartofbodyhook
\sphinxAtStartPar
España
&
\sphinxAtStartPar
IBEX 35
&
\sphinxAtStartPar
BME
&
\sphinxAtStartPar
.MC
&
\sphinxAtStartPar
35
\\
\sphinxhline
\sphinxAtStartPar
Alemania
&
\sphinxAtStartPar
DAX 40
&
\sphinxAtStartPar
XETR
&
\sphinxAtStartPar
.DE
&
\sphinxAtStartPar
40
\\
\sphinxhline
\sphinxAtStartPar
Francia
&
\sphinxAtStartPar
CAC 40
&
\sphinxAtStartPar
EURONEXT
&
\sphinxAtStartPar
.PA
&
\sphinxAtStartPar
39
\\
\sphinxhline
\sphinxAtStartPar
Italia
&
\sphinxAtStartPar
FTSE MIB
&
\sphinxAtStartPar
MIL
&
\sphinxAtStartPar
.MI
&
\sphinxAtStartPar
40
\\
\sphinxhline
\sphinxAtStartPar
Países Bajos
&
\sphinxAtStartPar
AEX
&
\sphinxAtStartPar
EURONEXT
&
\sphinxAtStartPar
.AS
&
\sphinxAtStartPar
25
\\
\sphinxhline
\sphinxAtStartPar
Reino Unido
&
\sphinxAtStartPar
FTSE 100
&
\sphinxAtStartPar
LSE
&
\sphinxAtStartPar
.L
&
\sphinxAtStartPar
100
\\
\sphinxhline
\sphinxAtStartPar
Suecia
&
\sphinxAtStartPar
OMXS30
&
\sphinxAtStartPar
OMXSTO
&
\sphinxAtStartPar
.ST
&
\sphinxAtStartPar
30
\\
\sphinxhline
\sphinxAtStartPar
Suiza
&
\sphinxAtStartPar
SMI
&
\sphinxAtStartPar
SIX
&
\sphinxAtStartPar
.SW
&
\sphinxAtStartPar
21
\\
\sphinxbottomrule
\end{tabulary}
\sphinxtableafterendhook\par
\sphinxattableend\end{savenotes}

\begin{figure}[htbp]
\centering
\capstart

\noindent\sphinxincludegraphics[width=1.000\linewidth]{{FlujoYahoo}.jpeg}
\caption{\sphinxstylestrong{Figura 9.} Flujograma del modelo de extracción de datos en Yahoo Finance}\label{\detokenize{DatosPreparacion:id4}}\end{figure}


\subsection{Scraping de índices europeos en TradingView}
\label{\detokenize{DatosPreparacion:scraping-de-indices-europeos-en-tradingview}}
\sphinxAtStartPar
En una primera etapa resulta necesario extraer los tickers representativos de las principales bolsas europeas. Para ello se emplea la API de TradingView, ampliamente utilizada en el ámbito del análisis financiero, implementando la función scrape\_country como componente orquestador del proceso. Esta función integra, por un lado, fetch\_html, encargada de recuperar el código HTML de las páginas de TradingView, y por otro, extract\_rows\_precise, que a través de la librería BeautifulSoup identifica los nombres y tickers de las compañías, filtra los resultados según el mercado de interés, elimina duplicados y descarta instrumentos no relevantes como ETFs o futuros. Finalmente, scrape\_country consolida ambos procedimientos para cada índice bursátil, generando un DataFrame estandarizado con el ticker en formato Yahoo Finance, el nombre depurado de la empresa, el país y el símbolo local, lo que permite conformar una base de datos estructurada y preparada para su posterior almacenamiento en Snowflake y para los análisis financieros avanzados.

\begin{sphinxVerbatim}[commandchars=\\\{\},numbers=left,firstnumber=1,stepnumber=1]
\PYG{k}{def}\PYG{+w}{ }\PYG{n+nf}{scrape\PYGZus{}country}\PYG{p}{(}\PYG{n}{spec}\PYG{p}{:} \PYG{n}{Dict}\PYG{p}{)} \PYG{o}{\PYGZhy{}}\PYG{o}{\PYGZgt{}} \PYG{n}{pd}\PYG{o}{.}\PYG{n}{DataFrame}\PYG{p}{:}
    \PYG{n}{html} \PYG{o}{=} \PYG{n}{fetch\PYGZus{}html}\PYG{p}{(}\PYG{n}{spec}\PYG{p}{[}\PYG{l+s+s2}{\PYGZdq{}}\PYG{l+s+s2}{components\PYGZus{}url}\PYG{l+s+s2}{\PYGZdq{}}\PYG{p}{]}\PYG{p}{)}
    \PYG{n}{pairs} \PYG{o}{=} \PYG{n}{extract\PYGZus{}rows\PYGZus{}precise}\PYG{p}{(}\PYG{n}{html}\PYG{p}{,} \PYG{n}{spec}\PYG{p}{[}\PYG{l+s+s2}{\PYGZdq{}}\PYG{l+s+s2}{accept\PYGZus{}exchanges}\PYG{l+s+s2}{\PYGZdq{}}\PYG{p}{]}\PYG{p}{)}
    \PYG{n}{n} \PYG{o}{=} \PYG{n+nb}{len}\PYG{p}{(}\PYG{n}{pairs}\PYG{p}{)}
    \PYG{n}{exp} \PYG{o}{=} \PYG{n}{spec}\PYG{p}{[}\PYG{l+s+s2}{\PYGZdq{}}\PYG{l+s+s2}{expected\PYGZus{}count}\PYG{l+s+s2}{\PYGZdq{}}\PYG{p}{]}
    \PYG{k}{if} \PYG{n}{n} \PYG{o}{!=} \PYG{n}{exp}\PYG{p}{:}
        \PYG{c+c1}{\PYGZsh{} tolera \(\pm\)2 y avisa (TV puede variar 1\textendash{}2 componentes)}
        \PYG{k}{if} \PYG{n+nb}{abs}\PYG{p}{(}\PYG{n}{n} \PYG{o}{\PYGZhy{}} \PYG{n}{exp}\PYG{p}{)} \PYG{o}{\PYGZlt{}}\PYG{o}{=} \PYG{l+m+mi}{2}\PYG{p}{:}
            \PYG{n+nb}{print}\PYG{p}{(}\PYG{l+s+sa}{f}\PYG{l+s+s2}{\PYGZdq{}}\PYG{l+s+s2}{[WARN] }\PYG{l+s+si}{\PYGZob{}}\PYG{n}{spec}\PYG{p}{[}\PYG{l+s+s1}{\PYGZsq{}}\PYG{l+s+s1}{index}\PYG{l+s+s1}{\PYGZsq{}}\PYG{p}{]}\PYG{l+s+si}{\PYGZcb{}}\PYG{l+s+s2}{ (}\PYG{l+s+si}{\PYGZob{}}\PYG{n}{spec}\PYG{p}{[}\PYG{l+s+s1}{\PYGZsq{}}\PYG{l+s+s1}{pais}\PYG{l+s+s1}{\PYGZsq{}}\PYG{p}{]}\PYG{l+s+si}{\PYGZcb{}}\PYG{l+s+s2}{): }\PYG{l+s+si}{\PYGZob{}}\PYG{n}{n}\PYG{l+s+si}{\PYGZcb{}}\PYG{l+s+s2}{ componentes (esperados }\PYG{l+s+si}{\PYGZob{}}\PYG{n}{exp}\PYG{l+s+si}{\PYGZcb{}}\PYG{l+s+s2}{). Continuando...}\PYG{l+s+s2}{\PYGZdq{}}\PYG{p}{)}
        \PYG{k}{else}\PYG{p}{:}
            \PYG{n+nb}{print}\PYG{p}{(}\PYG{l+s+sa}{f}\PYG{l+s+s2}{\PYGZdq{}}\PYG{l+s+s2}{[ERROR] }\PYG{l+s+si}{\PYGZob{}}\PYG{n}{spec}\PYG{p}{[}\PYG{l+s+s1}{\PYGZsq{}}\PYG{l+s+s1}{index}\PYG{l+s+s1}{\PYGZsq{}}\PYG{p}{]}\PYG{l+s+si}{\PYGZcb{}}\PYG{l+s+s2}{ (}\PYG{l+s+si}{\PYGZob{}}\PYG{n}{spec}\PYG{p}{[}\PYG{l+s+s1}{\PYGZsq{}}\PYG{l+s+s1}{pais}\PYG{l+s+s1}{\PYGZsq{}}\PYG{p}{]}\PYG{l+s+si}{\PYGZcb{}}\PYG{l+s+s2}{): }\PYG{l+s+si}{\PYGZob{}}\PYG{n}{n}\PYG{l+s+si}{\PYGZcb{}}\PYG{l+s+s2}{ componentes (esperados }\PYG{l+s+si}{\PYGZob{}}\PYG{n}{exp}\PYG{l+s+si}{\PYGZcb{}}\PYG{l+s+s2}{). Intento de continuar...}\PYG{l+s+s2}{\PYGZdq{}}\PYG{p}{)}
    \PYG{n}{rows} \PYG{o}{=} \PYG{p}{[}\PYG{p}{]}
    \PYG{k}{for} \PYG{n}{exch}\PYG{p}{,} \PYG{n}{sym}\PYG{p}{,} \PYG{n}{name} \PYG{o+ow}{in} \PYG{n}{pairs}\PYG{p}{:}
        \PYG{n}{base\PYGZus{}for\PYGZus{}yahoo} \PYG{o}{=} \PYG{n}{yahoo\PYGZus{}ticker\PYGZus{}from\PYGZus{}local}\PYG{p}{(}\PYG{n}{sym}\PYG{p}{,} \PYG{n}{spec}\PYG{p}{[}\PYG{l+s+s2}{\PYGZdq{}}\PYG{l+s+s2}{pais}\PYG{l+s+s2}{\PYGZdq{}}\PYG{p}{]}\PYG{p}{)}
        \PYG{n}{yahoo} \PYG{o}{=} \PYG{l+s+sa}{f}\PYG{l+s+s2}{\PYGZdq{}}\PYG{l+s+si}{\PYGZob{}}\PYG{n}{base\PYGZus{}for\PYGZus{}yahoo}\PYG{l+s+si}{\PYGZcb{}}\PYG{l+s+si}{\PYGZob{}}\PYG{n}{spec}\PYG{p}{[}\PYG{l+s+s1}{\PYGZsq{}}\PYG{l+s+s1}{yahoo\PYGZus{}suffix}\PYG{l+s+s1}{\PYGZsq{}}\PYG{p}{]}\PYG{l+s+si}{\PYGZcb{}}\PYG{l+s+s2}{\PYGZdq{}}
        \PYG{n}{rows}\PYG{o}{.}\PYG{n}{append}\PYG{p}{(}\PYG{p}{\PYGZob{}}
            \PYG{l+s+s2}{\PYGZdq{}}\PYG{l+s+s2}{TICKER\PYGZus{}YAHOO}\PYG{l+s+s2}{\PYGZdq{}}\PYG{p}{:} \PYG{n}{yahoo}\PYG{p}{,}
            \PYG{l+s+s2}{\PYGZdq{}}\PYG{l+s+s2}{NOMBRE}\PYG{l+s+s2}{\PYGZdq{}}\PYG{p}{:} \PYG{n}{clean\PYGZus{}company\PYGZus{}name}\PYG{p}{(}\PYG{n}{name}\PYG{p}{,} \PYG{n}{sym}\PYG{p}{)}\PYG{p}{,}
            \PYG{l+s+s2}{\PYGZdq{}}\PYG{l+s+s2}{PAIS}\PYG{l+s+s2}{\PYGZdq{}}\PYG{p}{:} \PYG{n}{spec}\PYG{p}{[}\PYG{l+s+s2}{\PYGZdq{}}\PYG{l+s+s2}{pais}\PYG{l+s+s2}{\PYGZdq{}}\PYG{p}{]}\PYG{p}{,}
            \PYG{l+s+s2}{\PYGZdq{}}\PYG{l+s+s2}{TICKET}\PYG{l+s+s2}{\PYGZdq{}}\PYG{p}{:} \PYG{n}{base\PYGZus{}for\PYGZus{}yahoo}
        \PYG{p}{\PYGZcb{}}\PYG{p}{)}
        \PYG{n}{time}\PYG{o}{.}\PYG{n}{sleep}\PYG{p}{(}\PYG{l+m+mf}{0.2}\PYG{p}{)}  \PYG{c+c1}{\PYGZsh{} respeta un poco más a TV}
    \PYG{k}{return} \PYG{n}{pd}\PYG{o}{.}\PYG{n}{DataFrame}\PYG{p}{(}\PYG{n}{rows}\PYG{p}{,} \PYG{n}{columns}\PYG{o}{=}\PYG{p}{[}\PYG{l+s+s2}{\PYGZdq{}}\PYG{l+s+s2}{TICKER\PYGZus{}YAHOO}\PYG{l+s+s2}{\PYGZdq{}}\PYG{p}{,} \PYG{l+s+s2}{\PYGZdq{}}\PYG{l+s+s2}{NOMBRE}\PYG{l+s+s2}{\PYGZdq{}}\PYG{p}{,} \PYG{l+s+s2}{\PYGZdq{}}\PYG{l+s+s2}{PAIS}\PYG{l+s+s2}{\PYGZdq{}}\PYG{p}{,} \PYG{l+s+s2}{\PYGZdq{}}\PYG{l+s+s2}{TICKET}\PYG{l+s+s2}{\PYGZdq{}}\PYG{p}{]}\PYG{p}{)}
\end{sphinxVerbatim}

\sphinxAtStartPar
Ejemplos de DataFrame de stickers:


\begin{savenotes}\sphinxattablestart
\sphinxthistablewithglobalstyle
\centering
\begin{tabulary}{\linewidth}[t]{TTTT}
\sphinxtoprule
\sphinxstyletheadfamily 
\sphinxAtStartPar
NOMBRE\_EMPRESA
&\sphinxstyletheadfamily 
\sphinxAtStartPar
PAIS
&\sphinxstyletheadfamily 
\sphinxAtStartPar
INDEX
&\sphinxstyletheadfamily 
\sphinxAtStartPar
TICKER
\\
\sphinxmidrule
\sphinxtableatstartofbodyhook
\sphinxAtStartPar
ACS, ACTIVIDADES DE CONSTRUCCION Y SERVICIOS, S.A.
&
\sphinxAtStartPar
España
&
\sphinxAtStartPar
IBEX35
&
\sphinxAtStartPar
ACS
\\
\sphinxhline
\sphinxAtStartPar
ACERINOX, S.A.
&
\sphinxAtStartPar
España
&
\sphinxAtStartPar
IBEX35
&
\sphinxAtStartPar
ACX
\\
\sphinxhline
\sphinxAtStartPar
AENA, S.M.E., S.A.
&
\sphinxAtStartPar
España
&
\sphinxAtStartPar
IBEX35
&
\sphinxAtStartPar
AENA
\\
\sphinxbottomrule
\end{tabulary}
\sphinxtableafterendhook\par
\sphinxattableend\end{savenotes}

\begin{figure}[htbp]
\centering
\capstart

\noindent\sphinxincludegraphics[width=1.000\linewidth]{{TV_componentes}.gif}
\caption{\sphinxstylestrong{Figura 10.} Acciones que actualmente pertenecen en al IBEX35. Ejemplo de los componentes que hemos extraídos.}\label{\detokenize{DatosPreparacion:id5}}\end{figure}


\subsection{Descarga de precios por empresa diaria}
\label{\detokenize{DatosPreparacion:descarga-de-precios-por-empresa-diaria}}
\sphinxAtStartPar
Una vez obtenida la lista de tickers, se procede a la consulta de la información histórica en Yahoo Finance mediante la función download\_batch. Dicha función permite recopilar de forma automatizada los precios diarios de apertura, cierre, máximo y mínimo, además del volumen de acciones negociadas, vinculando cada registro con su fecha correspondiente. Posteriormente, los datos son transformados y estandarizados en un formato homogéneo, lo que garantiza su integración sin inconsistencias dentro del flujo de almacenamiento en el DataLake y asegura la disponibilidad de un conjunto de información fiable para el análisis y la modelización posteriores.

\begin{sphinxVerbatim}[commandchars=\\\{\},numbers=left,firstnumber=1,stepnumber=1]
\PYG{k}{def}\PYG{+w}{ }\PYG{n+nf}{download\PYGZus{}batch}\PYG{p}{(}\PYG{n}{tickers}\PYG{p}{:} \PYG{n}{List}\PYG{p}{[}\PYG{n+nb}{str}\PYG{p}{]}\PYG{p}{,} \PYG{n}{start\PYGZus{}date}\PYG{p}{,} \PYG{n}{end\PYGZus{}excl}\PYG{p}{)} \PYG{o}{\PYGZhy{}}\PYG{o}{\PYGZgt{}} \PYG{n}{pd}\PYG{o}{.}\PYG{n}{DataFrame}\PYG{p}{:}
    \PYG{k}{if} \PYG{o+ow}{not} \PYG{n}{tickers}\PYG{p}{:}
        \PYG{k}{return} \PYG{n}{pd}\PYG{o}{.}\PYG{n}{DataFrame}\PYG{p}{(}\PYG{n}{columns}\PYG{o}{=}\PYG{p}{[}\PYG{l+s+s2}{\PYGZdq{}}\PYG{l+s+s2}{TICKER}\PYG{l+s+s2}{\PYGZdq{}}\PYG{p}{,}\PYG{l+s+s2}{\PYGZdq{}}\PYG{l+s+s2}{CLOSE}\PYG{l+s+s2}{\PYGZdq{}}\PYG{p}{,}\PYG{l+s+s2}{\PYGZdq{}}\PYG{l+s+s2}{HIGH}\PYG{l+s+s2}{\PYGZdq{}}\PYG{p}{,}\PYG{l+s+s2}{\PYGZdq{}}\PYG{l+s+s2}{LOW}\PYG{l+s+s2}{\PYGZdq{}}\PYG{p}{,}\PYG{l+s+s2}{\PYGZdq{}}\PYG{l+s+s2}{OPEN}\PYG{l+s+s2}{\PYGZdq{}}\PYG{p}{,}\PYG{l+s+s2}{\PYGZdq{}}\PYG{l+s+s2}{VOLUME}\PYG{l+s+s2}{\PYGZdq{}}\PYG{p}{,}\PYG{l+s+s2}{\PYGZdq{}}\PYG{l+s+s2}{FECHA}\PYG{l+s+s2}{\PYGZdq{}}\PYG{p}{]}\PYG{p}{)}
    \PYG{n}{df} \PYG{o}{=} \PYG{n}{yf}\PYG{o}{.}\PYG{n}{download}\PYG{p}{(}
        \PYG{n}{tickers}\PYG{p}{,} \PYG{n}{start}\PYG{o}{=}\PYG{n}{start\PYGZus{}date}\PYG{p}{,} \PYG{n}{end}\PYG{o}{=}\PYG{n}{end\PYGZus{}excl}\PYG{p}{,} \PYG{n}{interval}\PYG{o}{=}\PYG{l+s+s2}{\PYGZdq{}}\PYG{l+s+s2}{1d}\PYG{l+s+s2}{\PYGZdq{}}\PYG{p}{,}
        \PYG{n}{group\PYGZus{}by}\PYG{o}{=}\PYG{l+s+s2}{\PYGZdq{}}\PYG{l+s+s2}{ticker}\PYG{l+s+s2}{\PYGZdq{}}\PYG{p}{,} \PYG{n}{auto\PYGZus{}adjust}\PYG{o}{=}\PYG{k+kc}{False}\PYG{p}{,} \PYG{n}{progress}\PYG{o}{=}\PYG{k+kc}{False}\PYG{p}{,} \PYG{n}{threads}\PYG{o}{=}\PYG{k+kc}{True}
    \PYG{p}{)}
    \PYG{n}{rows} \PYG{o}{=} \PYG{p}{[}\PYG{p}{]}
    \PYG{k}{for} \PYG{n}{t} \PYG{o+ow}{in} \PYG{n}{tickers}\PYG{p}{:}
        \PYG{k}{try}\PYG{p}{:}
            \PYG{k}{if} \PYG{n+nb}{isinstance}\PYG{p}{(}\PYG{n}{df}\PYG{o}{.}\PYG{n}{columns}\PYG{p}{,} \PYG{n}{pd}\PYG{o}{.}\PYG{n}{MultiIndex}\PYG{p}{)}\PYG{p}{:}
                \PYG{k}{if} \PYG{n}{t} \PYG{o+ow}{not} \PYG{o+ow}{in} \PYG{n}{df}\PYG{o}{.}\PYG{n}{columns}\PYG{o}{.}\PYG{n}{get\PYGZus{}level\PYGZus{}values}\PYG{p}{(}\PYG{l+m+mi}{0}\PYG{p}{)}\PYG{p}{:}
                    \PYG{k}{continue}
                \PYG{n}{dft} \PYG{o}{=} \PYG{n}{df}\PYG{p}{[}\PYG{n}{t}\PYG{p}{]}
            \PYG{k}{else}\PYG{p}{:}
                \PYG{k}{if} \PYG{n+nb}{set}\PYG{p}{(}\PYG{n}{df}\PYG{o}{.}\PYG{n}{columns}\PYG{p}{)} \PYG{o}{\PYGZam{}} \PYG{p}{\PYGZob{}}\PYG{l+s+s2}{\PYGZdq{}}\PYG{l+s+s2}{Open}\PYG{l+s+s2}{\PYGZdq{}}\PYG{p}{,}\PYG{l+s+s2}{\PYGZdq{}}\PYG{l+s+s2}{High}\PYG{l+s+s2}{\PYGZdq{}}\PYG{p}{,}\PYG{l+s+s2}{\PYGZdq{}}\PYG{l+s+s2}{Low}\PYG{l+s+s2}{\PYGZdq{}}\PYG{p}{,}\PYG{l+s+s2}{\PYGZdq{}}\PYG{l+s+s2}{Close}\PYG{l+s+s2}{\PYGZdq{}}\PYG{p}{,}\PYG{l+s+s2}{\PYGZdq{}}\PYG{l+s+s2}{Volume}\PYG{l+s+s2}{\PYGZdq{}}\PYG{p}{\PYGZcb{}}\PYG{p}{:}
                    \PYG{n}{dft} \PYG{o}{=} \PYG{n}{df}
                \PYG{k}{else}\PYG{p}{:}
                    \PYG{k}{continue}
            \PYG{n}{dft} \PYG{o}{=} \PYG{n}{dft}\PYG{o}{.}\PYG{n}{reset\PYGZus{}index}\PYG{p}{(}\PYG{p}{)}\PYG{o}{.}\PYG{n}{rename}\PYG{p}{(}\PYG{n}{columns}\PYG{o}{=}\PYG{p}{\PYGZob{}}
                \PYG{l+s+s2}{\PYGZdq{}}\PYG{l+s+s2}{Date}\PYG{l+s+s2}{\PYGZdq{}}\PYG{p}{:}\PYG{l+s+s2}{\PYGZdq{}}\PYG{l+s+s2}{FECHA}\PYG{l+s+s2}{\PYGZdq{}}\PYG{p}{,}\PYG{l+s+s2}{\PYGZdq{}}\PYG{l+s+s2}{Open}\PYG{l+s+s2}{\PYGZdq{}}\PYG{p}{:}\PYG{l+s+s2}{\PYGZdq{}}\PYG{l+s+s2}{OPEN}\PYG{l+s+s2}{\PYGZdq{}}\PYG{p}{,}\PYG{l+s+s2}{\PYGZdq{}}\PYG{l+s+s2}{High}\PYG{l+s+s2}{\PYGZdq{}}\PYG{p}{:}\PYG{l+s+s2}{\PYGZdq{}}\PYG{l+s+s2}{HIGH}\PYG{l+s+s2}{\PYGZdq{}}\PYG{p}{,}\PYG{l+s+s2}{\PYGZdq{}}\PYG{l+s+s2}{Low}\PYG{l+s+s2}{\PYGZdq{}}\PYG{p}{:}\PYG{l+s+s2}{\PYGZdq{}}\PYG{l+s+s2}{LOW}\PYG{l+s+s2}{\PYGZdq{}}\PYG{p}{,}\PYG{l+s+s2}{\PYGZdq{}}\PYG{l+s+s2}{Close}\PYG{l+s+s2}{\PYGZdq{}}\PYG{p}{:}\PYG{l+s+s2}{\PYGZdq{}}\PYG{l+s+s2}{CLOSE}\PYG{l+s+s2}{\PYGZdq{}}\PYG{p}{,}\PYG{l+s+s2}{\PYGZdq{}}\PYG{l+s+s2}{Volume}\PYG{l+s+s2}{\PYGZdq{}}\PYG{p}{:}\PYG{l+s+s2}{\PYGZdq{}}\PYG{l+s+s2}{VOLUME}\PYG{l+s+s2}{\PYGZdq{}}
            \PYG{p}{\PYGZcb{}}\PYG{p}{)}
            \PYG{n}{dft}\PYG{p}{[}\PYG{l+s+s2}{\PYGZdq{}}\PYG{l+s+s2}{TICKER}\PYG{l+s+s2}{\PYGZdq{}}\PYG{p}{]} \PYG{o}{=} \PYG{n}{t}
            \PYG{n}{rows}\PYG{o}{.}\PYG{n}{append}\PYG{p}{(}\PYG{n}{dft}\PYG{p}{[}\PYG{p}{[}\PYG{l+s+s2}{\PYGZdq{}}\PYG{l+s+s2}{TICKER}\PYG{l+s+s2}{\PYGZdq{}}\PYG{p}{,}\PYG{l+s+s2}{\PYGZdq{}}\PYG{l+s+s2}{CLOSE}\PYG{l+s+s2}{\PYGZdq{}}\PYG{p}{,}\PYG{l+s+s2}{\PYGZdq{}}\PYG{l+s+s2}{HIGH}\PYG{l+s+s2}{\PYGZdq{}}\PYG{p}{,}\PYG{l+s+s2}{\PYGZdq{}}\PYG{l+s+s2}{LOW}\PYG{l+s+s2}{\PYGZdq{}}\PYG{p}{,}\PYG{l+s+s2}{\PYGZdq{}}\PYG{l+s+s2}{OPEN}\PYG{l+s+s2}{\PYGZdq{}}\PYG{p}{,}\PYG{l+s+s2}{\PYGZdq{}}\PYG{l+s+s2}{VOLUME}\PYG{l+s+s2}{\PYGZdq{}}\PYG{p}{,}\PYG{l+s+s2}{\PYGZdq{}}\PYG{l+s+s2}{FECHA}\PYG{l+s+s2}{\PYGZdq{}}\PYG{p}{]}\PYG{p}{]}\PYG{p}{)}
        \PYG{k}{except} \PYG{n+ne}{Exception}\PYG{p}{:}
            \PYG{k}{continue}
    \PYG{k}{if} \PYG{o+ow}{not} \PYG{n}{rows}\PYG{p}{:}
        \PYG{k}{return} \PYG{n}{pd}\PYG{o}{.}\PYG{n}{DataFrame}\PYG{p}{(}\PYG{n}{columns}\PYG{o}{=}\PYG{p}{[}\PYG{l+s+s2}{\PYGZdq{}}\PYG{l+s+s2}{TICKER}\PYG{l+s+s2}{\PYGZdq{}}\PYG{p}{,}\PYG{l+s+s2}{\PYGZdq{}}\PYG{l+s+s2}{CLOSE}\PYG{l+s+s2}{\PYGZdq{}}\PYG{p}{,}\PYG{l+s+s2}{\PYGZdq{}}\PYG{l+s+s2}{HIGH}\PYG{l+s+s2}{\PYGZdq{}}\PYG{p}{,}\PYG{l+s+s2}{\PYGZdq{}}\PYG{l+s+s2}{LOW}\PYG{l+s+s2}{\PYGZdq{}}\PYG{p}{,}\PYG{l+s+s2}{\PYGZdq{}}\PYG{l+s+s2}{OPEN}\PYG{l+s+s2}{\PYGZdq{}}\PYG{p}{,}\PYG{l+s+s2}{\PYGZdq{}}\PYG{l+s+s2}{VOLUME}\PYG{l+s+s2}{\PYGZdq{}}\PYG{p}{,}\PYG{l+s+s2}{\PYGZdq{}}\PYG{l+s+s2}{FECHA}\PYG{l+s+s2}{\PYGZdq{}}\PYG{p}{]}\PYG{p}{)}
    \PYG{n}{out} \PYG{o}{=} \PYG{n}{pd}\PYG{o}{.}\PYG{n}{concat}\PYG{p}{(}\PYG{n}{rows}\PYG{p}{,} \PYG{n}{ignore\PYGZus{}index}\PYG{o}{=}\PYG{k+kc}{True}\PYG{p}{)}\PYG{o}{.}\PYG{n}{dropna}\PYG{p}{(}\PYG{n}{subset}\PYG{o}{=}\PYG{p}{[}\PYG{l+s+s2}{\PYGZdq{}}\PYG{l+s+s2}{CLOSE}\PYG{l+s+s2}{\PYGZdq{}}\PYG{p}{]}\PYG{p}{)}
    \PYG{n}{out}\PYG{p}{[}\PYG{l+s+s2}{\PYGZdq{}}\PYG{l+s+s2}{FECHA}\PYG{l+s+s2}{\PYGZdq{}}\PYG{p}{]} \PYG{o}{=} \PYG{n}{pd}\PYG{o}{.}\PYG{n}{to\PYGZus{}datetime}\PYG{p}{(}\PYG{n}{out}\PYG{p}{[}\PYG{l+s+s2}{\PYGZdq{}}\PYG{l+s+s2}{FECHA}\PYG{l+s+s2}{\PYGZdq{}}\PYG{p}{]}\PYG{p}{)}\PYG{o}{.}\PYG{n}{dt}\PYG{o}{.}\PYG{n}{date}
    \PYG{k}{for} \PYG{n}{col} \PYG{o+ow}{in} \PYG{p}{[}\PYG{l+s+s2}{\PYGZdq{}}\PYG{l+s+s2}{CLOSE}\PYG{l+s+s2}{\PYGZdq{}}\PYG{p}{,}\PYG{l+s+s2}{\PYGZdq{}}\PYG{l+s+s2}{HIGH}\PYG{l+s+s2}{\PYGZdq{}}\PYG{p}{,}\PYG{l+s+s2}{\PYGZdq{}}\PYG{l+s+s2}{LOW}\PYG{l+s+s2}{\PYGZdq{}}\PYG{p}{,}\PYG{l+s+s2}{\PYGZdq{}}\PYG{l+s+s2}{OPEN}\PYG{l+s+s2}{\PYGZdq{}}\PYG{p}{]}\PYG{p}{:}
        \PYG{n}{out}\PYG{p}{[}\PYG{n}{col}\PYG{p}{]} \PYG{o}{=} \PYG{n}{pd}\PYG{o}{.}\PYG{n}{to\PYGZus{}numeric}\PYG{p}{(}\PYG{n}{out}\PYG{p}{[}\PYG{n}{col}\PYG{p}{]}\PYG{p}{,} \PYG{n}{errors}\PYG{o}{=}\PYG{l+s+s2}{\PYGZdq{}}\PYG{l+s+s2}{coerce}\PYG{l+s+s2}{\PYGZdq{}}\PYG{p}{)}
    \PYG{n}{out}\PYG{p}{[}\PYG{l+s+s2}{\PYGZdq{}}\PYG{l+s+s2}{VOLUME}\PYG{l+s+s2}{\PYGZdq{}}\PYG{p}{]} \PYG{o}{=} \PYG{n}{pd}\PYG{o}{.}\PYG{n}{to\PYGZus{}numeric}\PYG{p}{(}\PYG{n}{out}\PYG{p}{[}\PYG{l+s+s2}{\PYGZdq{}}\PYG{l+s+s2}{VOLUME}\PYG{l+s+s2}{\PYGZdq{}}\PYG{p}{]}\PYG{p}{,} \PYG{n}{errors}\PYG{o}{=}\PYG{l+s+s2}{\PYGZdq{}}\PYG{l+s+s2}{coerce}\PYG{l+s+s2}{\PYGZdq{}}\PYG{p}{)}\PYG{o}{.}\PYG{n}{astype}\PYG{p}{(}\PYG{l+s+s2}{\PYGZdq{}}\PYG{l+s+s2}{Int64}\PYG{l+s+s2}{\PYGZdq{}}\PYG{p}{)}
    \PYG{k}{return} \PYG{n}{out}\PYG{o}{.}\PYG{n}{dropna}\PYG{p}{(}\PYG{n}{subset}\PYG{o}{=}\PYG{p}{[}\PYG{l+s+s2}{\PYGZdq{}}\PYG{l+s+s2}{CLOSE}\PYG{l+s+s2}{\PYGZdq{}}\PYG{p}{,}\PYG{l+s+s2}{\PYGZdq{}}\PYG{l+s+s2}{HIGH}\PYG{l+s+s2}{\PYGZdq{}}\PYG{p}{,}\PYG{l+s+s2}{\PYGZdq{}}\PYG{l+s+s2}{LOW}\PYG{l+s+s2}{\PYGZdq{}}\PYG{p}{,}\PYG{l+s+s2}{\PYGZdq{}}\PYG{l+s+s2}{OPEN}\PYG{l+s+s2}{\PYGZdq{}}\PYG{p}{]}\PYG{p}{)}
\end{sphinxVerbatim}

\sphinxAtStartPar
Ejemplos de DataFrame de Precios por accion


\begin{savenotes}\sphinxattablestart
\sphinxthistablewithglobalstyle
\centering
\begin{tabulary}{\linewidth}[t]{TTTTTTT}
\sphinxtoprule
\sphinxstyletheadfamily 
\sphinxAtStartPar
TICKER
&\sphinxstyletheadfamily 
\sphinxAtStartPar
CLOSE
&\sphinxstyletheadfamily 
\sphinxAtStartPar
HIGH
&\sphinxstyletheadfamily 
\sphinxAtStartPar
LOW
&\sphinxstyletheadfamily 
\sphinxAtStartPar
OPEN
&\sphinxstyletheadfamily 
\sphinxAtStartPar
VOLUME
&\sphinxstyletheadfamily 
\sphinxAtStartPar
FECHA
\\
\sphinxmidrule
\sphinxtableatstartofbodyhook
\sphinxAtStartPar
INGA.AS
&
\sphinxAtStartPar
6.59100008
&
\sphinxAtStartPar
6.656000137
&
\sphinxAtStartPar
6.429999828
&
\sphinxAtStartPar
6.5
&
\sphinxAtStartPar
17736691
&
\sphinxAtStartPar
10/23/20
\\
\sphinxhline
\sphinxAtStartPar
GLE.PA
&
\sphinxAtStartPar
23.84499931
&
\sphinxAtStartPar
24.03000069
&
\sphinxAtStartPar
23.65500069
&
\sphinxAtStartPar
23.87000084
&
\sphinxAtStartPar
2111582
&
\sphinxAtStartPar
10/28/24
\\
\sphinxhline
\sphinxAtStartPar
LAND.L
&
\sphinxAtStartPar
686.2000122
&
\sphinxAtStartPar
697.2000122
&
\sphinxAtStartPar
686.2000122
&
\sphinxAtStartPar
690.5999756
&
\sphinxAtStartPar
1574957
&
\sphinxAtStartPar
06/17/21
\\
\sphinxbottomrule
\end{tabulary}
\sphinxtableafterendhook\par
\sphinxattableend\end{savenotes}


\subsection{Carga de los DataFrame de precios por tickers en el DataLake}
\label{\detokenize{DatosPreparacion:carga-de-los-dataframe-de-precios-por-tickers-en-el-datalake}}
\sphinxAtStartPar
Finalmente, se lleva a cabo un proceso de verificación de la última fecha de actualización registrada en el DataLake, tanto para los precios como para los tickers disponibles. A partir de dicha referencia temporal, se descargan únicamente los datos faltantes hasta el día inmediatamente anterior, con el fin de mantener una ingesta incremental que garantice la actualización diaria del repositorio. Este enfoque asegura la coherencia y completitud del histórico financiero, a la vez que proporciona una base de datos confiable, estructurada y permanentemente actualizada para los posteriores análisis y modelado mediante técnicas de machine learning.

\begin{sphinxVerbatim}[commandchars=\\\{\},numbers=left,firstnumber=1,stepnumber=1]
\PYG{k}{def}\PYG{+w}{ }\PYG{n+nf}{merge\PYGZus{}with\PYGZus{}temp}\PYG{p}{(}\PYG{n}{conn}\PYG{p}{,} \PYG{n}{df}\PYG{p}{:} \PYG{n}{pd}\PYG{o}{.}\PYG{n}{DataFrame}\PYG{p}{)}\PYG{p}{:}
\PYG{+w}{    }\PYG{l+s+sd}{\PYGZdq{}\PYGZdq{}\PYGZdq{}Carga df a TMP\PYGZus{}PRICES con write\PYGZus{}pandas y luego MERGE \(\rightarrow\) sin límite de expresiones.\PYGZdq{}\PYGZdq{}\PYGZdq{}}
    \PYG{k}{if} \PYG{n}{df}\PYG{o}{.}\PYG{n}{empty}\PYG{p}{:}
        \PYG{k}{return}
    \PYG{n}{df2} \PYG{o}{=} \PYG{n}{df}\PYG{p}{[}\PYG{p}{[}\PYG{l+s+s2}{\PYGZdq{}}\PYG{l+s+s2}{TICKER}\PYG{l+s+s2}{\PYGZdq{}}\PYG{p}{,}\PYG{l+s+s2}{\PYGZdq{}}\PYG{l+s+s2}{CLOSE}\PYG{l+s+s2}{\PYGZdq{}}\PYG{p}{,}\PYG{l+s+s2}{\PYGZdq{}}\PYG{l+s+s2}{HIGH}\PYG{l+s+s2}{\PYGZdq{}}\PYG{p}{,}\PYG{l+s+s2}{\PYGZdq{}}\PYG{l+s+s2}{LOW}\PYG{l+s+s2}{\PYGZdq{}}\PYG{p}{,}\PYG{l+s+s2}{\PYGZdq{}}\PYG{l+s+s2}{OPEN}\PYG{l+s+s2}{\PYGZdq{}}\PYG{p}{,}\PYG{l+s+s2}{\PYGZdq{}}\PYG{l+s+s2}{VOLUME}\PYG{l+s+s2}{\PYGZdq{}}\PYG{p}{,}\PYG{l+s+s2}{\PYGZdq{}}\PYG{l+s+s2}{FECHA}\PYG{l+s+s2}{\PYGZdq{}}\PYG{p}{]}\PYG{p}{]}\PYG{o}{.}\PYG{n}{copy}\PYG{p}{(}\PYG{p}{)}
    \PYG{k}{with} \PYG{n}{conn}\PYG{o}{.}\PYG{n}{cursor}\PYG{p}{(}\PYG{p}{)} \PYG{k}{as} \PYG{n}{cur}\PYG{p}{:}
        \PYG{n}{cur}\PYG{o}{.}\PYG{n}{execute}\PYG{p}{(}\PYG{l+s+sa}{f}\PYG{l+s+s2}{\PYGZdq{}}\PYG{l+s+s2}{CREATE OR REPLACE TEMP TABLE TMP\PYGZus{}PRICES LIKE }\PYG{l+s+si}{\PYGZob{}}\PYG{n}{PRICES\PYGZus{}TABLE}\PYG{l+s+si}{\PYGZcb{}}\PYG{l+s+s2}{\PYGZdq{}}\PYG{p}{)}
    \PYG{n}{ok}\PYG{p}{,} \PYG{n}{nchunks}\PYG{p}{,} \PYG{n}{nrows}\PYG{p}{,} \PYG{n}{\PYGZus{}} \PYG{o}{=} \PYG{n}{write\PYGZus{}pandas}\PYG{p}{(}\PYG{n}{conn}\PYG{p}{,} \PYG{n}{df2}\PYG{p}{,} \PYG{n}{table\PYGZus{}name}\PYG{o}{=}\PYG{l+s+s2}{\PYGZdq{}}\PYG{l+s+s2}{TMP\PYGZus{}PRICES}\PYG{l+s+s2}{\PYGZdq{}}\PYG{p}{,} \PYG{n}{quote\PYGZus{}identifiers}\PYG{o}{=}\PYG{k+kc}{False}\PYG{p}{)}
    \PYG{k}{if} \PYG{o+ow}{not} \PYG{n}{ok}\PYG{p}{:}
        \PYG{k}{raise} \PYG{n+ne}{RuntimeError}\PYG{p}{(}\PYG{l+s+s2}{\PYGZdq{}}\PYG{l+s+s2}{write\PYGZus{}pandas falló al cargar TMP\PYGZus{}PRICES.}\PYG{l+s+s2}{\PYGZdq{}}\PYG{p}{)}
    \PYG{n}{merge\PYGZus{}sql} \PYG{o}{=} \PYG{l+s+sa}{f}\PYG{l+s+s2}{\PYGZdq{}\PYGZdq{}\PYGZdq{}}
\PYG{l+s+s2}{        MERGE INTO }\PYG{l+s+si}{\PYGZob{}}\PYG{n}{PRICES\PYGZus{}TABLE}\PYG{l+s+si}{\PYGZcb{}}\PYG{l+s+s2}{ t}
\PYG{l+s+s2}{        USING TMP\PYGZus{}PRICES s}
\PYG{l+s+s2}{          ON t.TICKER = s.TICKER AND t.FECHA = s.FECHA}
\PYG{l+s+s2}{        WHEN MATCHED THEN UPDATE SET}
\PYG{l+s+s2}{          t.CLOSE = s.CLOSE, t.HIGH = s.HIGH, t.LOW = s.LOW, t.OPEN = s.OPEN, t.VOLUME = s.VOLUME}
\PYG{l+s+s2}{        WHEN NOT MATCHED THEN}
\PYG{l+s+s2}{          INSERT (TICKER, CLOSE, HIGH, LOW, OPEN, VOLUME, FECHA)}
\PYG{l+s+s2}{          VALUES (s.TICKER, s.CLOSE, s.HIGH, s.LOW, s.OPEN, s.VOLUME, s.FECHA)}
\PYG{l+s+s2}{    }\PYG{l+s+s2}{\PYGZdq{}\PYGZdq{}\PYGZdq{}}
    \PYG{k}{with} \PYG{n}{conn}\PYG{o}{.}\PYG{n}{cursor}\PYG{p}{(}\PYG{p}{)} \PYG{k}{as} \PYG{n}{cur}\PYG{p}{:}
        \PYG{n}{cur}\PYG{o}{.}\PYG{n}{execute}\PYG{p}{(}\PYG{n}{merge\PYGZus{}sql}\PYG{p}{)}
        \PYG{n}{conn}\PYG{o}{.}\PYG{n}{commit}\PYG{p}{(}\PYG{p}{)}
\end{sphinxVerbatim}


\section{Estados Financieros}
\label{\detokenize{DatosPreparacion:estados-financieros}}

\subsection{Estados Financieros por tickers}
\label{\detokenize{DatosPreparacion:estados-financieros-por-tickers}}
\sphinxAtStartPar
En nuestras bases de dato necesitamos para el análisis técnico los estados financieros de las empresas analizadas y en esto se reúne la información contable clave de la empresa, organizada por año. Incluye el balance (activos, pasivos y patrimonio), la cuenta de resultados (ingresos, gastos y beneficio neto) y los flujos de efectivo (operativo, de inversión y de financiación), además de indicadores derivados como el margen neto, el ROA, el ROE y el ratio deuda/patrimonio. Esta estructura no solo permite analizar la evolución histórica de la compañía, sino también evaluar su solidez financiera, rentabilidad y nivel de endeudamiento de manera comparativa y objetiva.

\sphinxAtStartPar
Tabla: Ratios descargados:


\begin{savenotes}\sphinxattablestart
\sphinxthistablewithglobalstyle
\centering
\begin{tabulary}{\linewidth}[t]{TTTTTTTTTTTTTTTT}
\sphinxtoprule
\sphinxstyletheadfamily 
\sphinxAtStartPar
TICKER
&\sphinxstyletheadfamily 
\sphinxAtStartPar
YEAR
&\sphinxstyletheadfamily 
\sphinxAtStartPar
ASSETS
&\sphinxstyletheadfamily 
\sphinxAtStartPar
LIABILITIES
&\sphinxstyletheadfamily 
\sphinxAtStartPar
EQUITY
&\sphinxstyletheadfamily 
\sphinxAtStartPar
REVENUE
&\sphinxstyletheadfamily 
\sphinxAtStartPar
EXPENSES
&\sphinxstyletheadfamily 
\sphinxAtStartPar
NET\_INCOME
&\sphinxstyletheadfamily 
\sphinxAtStartPar
OPERATING\_CF
&\sphinxstyletheadfamily 
\sphinxAtStartPar
INVESTING\_CF
&\sphinxstyletheadfamily 
\sphinxAtStartPar
FINANCING\_CF
&\sphinxstyletheadfamily 
\sphinxAtStartPar
FREE\_CF
&\sphinxstyletheadfamily 
\sphinxAtStartPar
NET\_MARGIN
&\sphinxstyletheadfamily 
\sphinxAtStartPar
ROA
&\sphinxstyletheadfamily 
\sphinxAtStartPar
ROE
&\sphinxstyletheadfamily 
\sphinxAtStartPar
DEBT\_EQUITY
\\
\sphinxmidrule
\sphinxtableatstartofbodyhook
\sphinxAtStartPar
A2A.MI
&
\sphinxAtStartPar
2021
&
\sphinxAtStartPar
18008000000
&
\sphinxAtStartPar
13705000000
&
\sphinxAtStartPar
3760000000
&
\sphinxAtStartPar
11352000000
&
\sphinxAtStartPar
10848000000
&
\sphinxAtStartPar
504000000
&
\sphinxAtStartPar
1135000000
&
\sphinxAtStartPar
\sphinxhyphen{}1595000000
&
\sphinxAtStartPar
412000000
&
\sphinxAtStartPar
61000000
&
\sphinxAtStartPar
0.044
&
\sphinxAtStartPar
0.028
&
\sphinxAtStartPar
0.134
&
\sphinxAtStartPar
3.645
\\
\sphinxhline
\sphinxAtStartPar
A2A.MI
&
\sphinxAtStartPar
2022
&
\sphinxAtStartPar
21367000000
&
\sphinxAtStartPar
16900000000
&
\sphinxAtStartPar
3899000000
&
\sphinxAtStartPar
22938000000
&
\sphinxAtStartPar
22537000000
&
\sphinxAtStartPar
401000000
&
\sphinxAtStartPar
1260000000
&
\sphinxAtStartPar
\sphinxhyphen{}1142000000
&
\sphinxAtStartPar
1502000000
&
\sphinxAtStartPar
20000000
&
\sphinxAtStartPar
0.017
&
\sphinxAtStartPar
0.019
&
\sphinxAtStartPar
0.103
&
\sphinxAtStartPar
4.334
\\
\sphinxhline
\sphinxAtStartPar
A2A.MI
&
\sphinxAtStartPar
2023
&
\sphinxAtStartPar
18798000000
&
\sphinxAtStartPar
13996000000
&
\sphinxAtStartPar
4240000000
&
\sphinxAtStartPar
14492000000
&
\sphinxAtStartPar
13833000000
&
\sphinxAtStartPar
659000000
&
\sphinxAtStartPar
1040000000
&
\sphinxAtStartPar
\sphinxhyphen{}1359000000
&
\sphinxAtStartPar
\sphinxhyphen{}636000000
&
\sphinxAtStartPar
\sphinxhyphen{}336000000
&
\sphinxAtStartPar
0.045
&
\sphinxAtStartPar
0.035
&
\sphinxAtStartPar
0.155
&
\sphinxAtStartPar
3.301
\\
\sphinxbottomrule
\end{tabulary}
\sphinxtableafterendhook\par
\sphinxattableend\end{savenotes}

\begin{figure}[htbp]
\centering
\capstart

\noindent\sphinxincludegraphics[width=1.000\linewidth]{{Docker}.jpeg}
\caption{\sphinxstylestrong{Figura 11.} Encendida del control de Docker.}\label{\detokenize{DatosPreparacion:id6}}\end{figure}


\subsection{Ratios Financieros}
\label{\detokenize{DatosPreparacion:ratios-financieros}}
\sphinxAtStartPar
Los principales ratios bursátiles  constituyen herramientas fundamentales para evaluar cómo el mercado valora a una empresa y qué expectativas existen sobre su desempeño. El PER (Price to Earnings Ratio), tanto en su versión histórica (trailing) como en la proyectada (forward), mide cuántas veces los inversores están pagando las utilidades de la compañía; un valor elevado puede reflejar expectativas de crecimiento, mientras que uno reducido puede sugerir infravaloración. El Price to Book (P/B) compara el precio de mercado con el valor contable, permitiendo identificar si la acción cotiza por encima o por debajo de sus activos netos. Por su parte, el EV/EBITDA es un ratio muy utilizado para comparar empresas dentro de un mismo sector, ya que relaciona el valor total de la compañía (incluyendo deuda) con su capacidad operativa de generar beneficios.

\sphinxAtStartPar
El script desarrollado en este proyecto automatiza la obtención de estos indicadores bursátiles desde Yahoo Finance y los almacena en Snowflake de forma estructurada y actualizada. Este procedimiento permite disponer de un repositorio centralizado de ratios de valoración, rentabilidad y dividendos, listo para ser utilizado en análisis posteriores. Al integrarse con los otros módulos implementados, se conforma una base de datos integral que combina perspectivas de mercado, solidez fundamental y sostenibilidad, lo que proporciona una visión completa para la toma de decisiones financieras y de inversión.

\begin{sphinxVerbatim}[commandchars=\\\{\},numbers=left,firstnumber=1,stepnumber=1]
\PYG{k}{def}\PYG{+w}{ }\PYG{n+nf}{fetch\PYGZus{}snapshot\PYGZus{}one}\PYG{p}{(}\PYG{n}{ticker}\PYG{p}{:} \PYG{n+nb}{str}\PYG{p}{,} \PYG{n}{retries}\PYG{p}{:} \PYG{n+nb}{int} \PYG{o}{=} \PYG{l+m+mi}{3}\PYG{p}{,} \PYG{n}{pause}\PYG{p}{:} \PYG{n+nb}{float} \PYG{o}{=} \PYG{l+m+mf}{0.6}\PYG{p}{)} \PYG{o}{\PYGZhy{}}\PYG{o}{\PYGZgt{}} \PYG{n+nb}{dict}\PYG{p}{:}
    \PYG{k}{for} \PYG{n}{i} \PYG{o+ow}{in} \PYG{n+nb}{range}\PYG{p}{(}\PYG{n}{retries}\PYG{p}{)}\PYG{p}{:}
        \PYG{k}{try}\PYG{p}{:}
            \PYG{n}{tk} \PYG{o}{=} \PYG{n}{yf}\PYG{o}{.}\PYG{n}{Ticker}\PYG{p}{(}\PYG{n}{ticker}\PYG{p}{)}
            \PYG{c+c1}{\PYGZsh{} prefer get\PYGZus{}info() (más estable en versiones recientes); fallback a .info}
            \PYG{n}{info} \PYG{o}{=} \PYG{p}{\PYGZob{}}\PYG{p}{\PYGZcb{}}
            \PYG{k}{try}\PYG{p}{:}
                \PYG{n}{info} \PYG{o}{=} \PYG{n}{tk}\PYG{o}{.}\PYG{n}{get\PYGZus{}info}\PYG{p}{(}\PYG{p}{)} \PYG{o+ow}{or} \PYG{p}{\PYGZob{}}\PYG{p}{\PYGZcb{}}
            \PYG{k}{except} \PYG{n+ne}{Exception}\PYG{p}{:}
                \PYG{n}{info} \PYG{o}{=} \PYG{n}{tk}\PYG{o}{.}\PYG{n}{info} \PYG{o+ow}{or} \PYG{p}{\PYGZob{}}\PYG{p}{\PYGZcb{}}

            \PYG{n}{pe\PYGZus{}trailing}   \PYG{o}{=} \PYG{n}{info}\PYG{o}{.}\PYG{n}{get}\PYG{p}{(}\PYG{l+s+s2}{\PYGZdq{}}\PYG{l+s+s2}{trailingPE}\PYG{l+s+s2}{\PYGZdq{}}\PYG{p}{)}
            \PYG{n}{pe\PYGZus{}forward}    \PYG{o}{=} \PYG{n}{info}\PYG{o}{.}\PYG{n}{get}\PYG{p}{(}\PYG{l+s+s2}{\PYGZdq{}}\PYG{l+s+s2}{forwardPE}\PYG{l+s+s2}{\PYGZdq{}}\PYG{p}{)}
            \PYG{n}{price\PYGZus{}to\PYGZus{}book} \PYG{o}{=} \PYG{n}{info}\PYG{o}{.}\PYG{n}{get}\PYG{p}{(}\PYG{l+s+s2}{\PYGZdq{}}\PYG{l+s+s2}{priceToBook}\PYG{l+s+s2}{\PYGZdq{}}\PYG{p}{)}
            \PYG{n}{ev\PYGZus{}to\PYGZus{}ebitda}  \PYG{o}{=} \PYG{n}{info}\PYG{o}{.}\PYG{n}{get}\PYG{p}{(}\PYG{l+s+s2}{\PYGZdq{}}\PYG{l+s+s2}{enterpriseToEbitda}\PYG{l+s+s2}{\PYGZdq{}}\PYG{p}{)}
            \PYG{n}{dividend\PYGZus{}yld}  \PYG{o}{=} \PYG{n}{info}\PYG{o}{.}\PYG{n}{get}\PYG{p}{(}\PYG{l+s+s2}{\PYGZdq{}}\PYG{l+s+s2}{dividendYield}\PYG{l+s+s2}{\PYGZdq{}}\PYG{p}{)}
            \PYG{n}{payout\PYGZus{}ratio}  \PYG{o}{=} \PYG{n}{info}\PYG{o}{.}\PYG{n}{get}\PYG{p}{(}\PYG{l+s+s2}{\PYGZdq{}}\PYG{l+s+s2}{payoutRatio}\PYG{l+s+s2}{\PYGZdq{}}\PYG{p}{)}
            \PYG{n}{market\PYGZus{}cap}    \PYG{o}{=} \PYG{n}{info}\PYG{o}{.}\PYG{n}{get}\PYG{p}{(}\PYG{l+s+s2}{\PYGZdq{}}\PYG{l+s+s2}{marketCap}\PYG{l+s+s2}{\PYGZdq{}}\PYG{p}{)}
            \PYG{n}{enterprise\PYGZus{}v}  \PYG{o}{=} \PYG{n}{info}\PYG{o}{.}\PYG{n}{get}\PYG{p}{(}\PYG{l+s+s2}{\PYGZdq{}}\PYG{l+s+s2}{enterpriseValue}\PYG{l+s+s2}{\PYGZdq{}}\PYG{p}{)}
            \PYG{n}{shares\PYGZus{}out}    \PYG{o}{=} \PYG{n}{info}\PYG{o}{.}\PYG{n}{get}\PYG{p}{(}\PYG{l+s+s2}{\PYGZdq{}}\PYG{l+s+s2}{sharesOutstanding}\PYG{l+s+s2}{\PYGZdq{}}\PYG{p}{)}

            \PYG{c+c1}{\PYGZsh{} Si falta EV/EBITDA, calcula con enterpriseValue y EBITDA reciente}
            \PYG{k}{if} \PYG{n}{ev\PYGZus{}to\PYGZus{}ebitda} \PYG{o+ow}{in} \PYG{p}{(}\PYG{k+kc}{None}\PYG{p}{,} \PYG{l+m+mi}{0}\PYG{p}{,} \PYG{n+nb}{float}\PYG{p}{(}\PYG{l+s+s2}{\PYGZdq{}}\PYG{l+s+s2}{inf}\PYG{l+s+s2}{\PYGZdq{}}\PYG{p}{)}\PYG{p}{)}\PYG{p}{:}
                \PYG{n}{ebitda} \PYG{o}{=} \PYG{n}{info}\PYG{o}{.}\PYG{n}{get}\PYG{p}{(}\PYG{l+s+s2}{\PYGZdq{}}\PYG{l+s+s2}{ebitda}\PYG{l+s+s2}{\PYGZdq{}}\PYG{p}{)}
                \PYG{k}{if} \PYG{n}{ebitda} \PYG{o+ow}{is} \PYG{k+kc}{None}\PYG{p}{:}
                    \PYG{n}{ebitda} \PYG{o}{=} \PYG{n}{\PYGZus{}latest\PYGZus{}ebitda\PYGZus{}from\PYGZus{}financials}\PYG{p}{(}\PYG{n}{tk}\PYG{p}{)}
                \PYG{k}{if} \PYG{n}{enterprise\PYGZus{}v} \PYG{o+ow}{is} \PYG{o+ow}{not} \PYG{k+kc}{None} \PYG{o+ow}{and} \PYG{n}{ebitda} \PYG{o+ow}{not} \PYG{o+ow}{in} \PYG{p}{(}\PYG{k+kc}{None}\PYG{p}{,} \PYG{l+m+mi}{0}\PYG{p}{)}\PYG{p}{:}
                    \PYG{n}{ev\PYGZus{}to\PYGZus{}ebitda} \PYG{o}{=} \PYG{n+nb}{float}\PYG{p}{(}\PYG{n}{enterprise\PYGZus{}v}\PYG{p}{)} \PYG{o}{/} \PYG{n+nb}{float}\PYG{p}{(}\PYG{n}{ebitda}\PYG{p}{)}

            \PYG{n}{row} \PYG{o}{=} \PYG{p}{\PYGZob{}}
                \PYG{l+s+s2}{\PYGZdq{}}\PYG{l+s+s2}{TICKER}\PYG{l+s+s2}{\PYGZdq{}}\PYG{p}{:} \PYG{n}{ticker}\PYG{p}{,}
                \PYG{l+s+s2}{\PYGZdq{}}\PYG{l+s+s2}{PE\PYGZus{}TRAILING}\PYG{l+s+s2}{\PYGZdq{}}\PYG{p}{:} \PYG{n}{pe\PYGZus{}trailing}\PYG{p}{,}
                \PYG{l+s+s2}{\PYGZdq{}}\PYG{l+s+s2}{PE\PYGZus{}FORWARD}\PYG{l+s+s2}{\PYGZdq{}}\PYG{p}{:} \PYG{n}{pe\PYGZus{}forward}\PYG{p}{,}
                \PYG{l+s+s2}{\PYGZdq{}}\PYG{l+s+s2}{PRICE\PYGZus{}TO\PYGZus{}BOOK}\PYG{l+s+s2}{\PYGZdq{}}\PYG{p}{:} \PYG{n}{price\PYGZus{}to\PYGZus{}book}\PYG{p}{,}
                \PYG{l+s+s2}{\PYGZdq{}}\PYG{l+s+s2}{EV\PYGZus{}TO\PYGZus{}EBITDA}\PYG{l+s+s2}{\PYGZdq{}}\PYG{p}{:} \PYG{n}{ev\PYGZus{}to\PYGZus{}ebitda}\PYG{p}{,}
                \PYG{l+s+s2}{\PYGZdq{}}\PYG{l+s+s2}{DIVIDEND\PYGZus{}YIELD}\PYG{l+s+s2}{\PYGZdq{}}\PYG{p}{:} \PYG{n}{dividend\PYGZus{}yld}\PYG{p}{,}
                \PYG{l+s+s2}{\PYGZdq{}}\PYG{l+s+s2}{PAYOUT\PYGZus{}RATIO}\PYG{l+s+s2}{\PYGZdq{}}\PYG{p}{:} \PYG{n}{payout\PYGZus{}ratio}\PYG{p}{,}
                \PYG{l+s+s2}{\PYGZdq{}}\PYG{l+s+s2}{MARKET\PYGZus{}CAP}\PYG{l+s+s2}{\PYGZdq{}}\PYG{p}{:} \PYG{n}{market\PYGZus{}cap}\PYG{p}{,}
                \PYG{l+s+s2}{\PYGZdq{}}\PYG{l+s+s2}{ENTERPRISE\PYGZus{}VALUE}\PYG{l+s+s2}{\PYGZdq{}}\PYG{p}{:} \PYG{n}{enterprise\PYGZus{}v}\PYG{p}{,}
                \PYG{l+s+s2}{\PYGZdq{}}\PYG{l+s+s2}{SHARES\PYGZus{}OUTSTANDING}\PYG{l+s+s2}{\PYGZdq{}}\PYG{p}{:} \PYG{n}{shares\PYGZus{}out}\PYG{p}{,}
            \PYG{p}{\PYGZcb{}}
            \PYG{k}{return} \PYG{n}{row}
        \PYG{k}{except} \PYG{n+ne}{Exception}\PYG{p}{:}
            \PYG{k}{if} \PYG{n}{i} \PYG{o}{\PYGZlt{}} \PYG{n}{retries} \PYG{o}{\PYGZhy{}} \PYG{l+m+mi}{1}\PYG{p}{:}
                \PYG{n}{time}\PYG{o}{.}\PYG{n}{sleep}\PYG{p}{(}\PYG{n}{pause} \PYG{o}{*} \PYG{p}{(}\PYG{n}{i} \PYG{o}{+} \PYG{l+m+mi}{1}\PYG{p}{)}\PYG{p}{)}
            \PYG{k}{else}\PYG{p}{:}
                \PYG{c+c1}{\PYGZsh{} devuelve fila con NULOS si no se pudo obtener}
                \PYG{k}{return} \PYG{p}{\PYGZob{}}
                    \PYG{l+s+s2}{\PYGZdq{}}\PYG{l+s+s2}{TICKER}\PYG{l+s+s2}{\PYGZdq{}}\PYG{p}{:} \PYG{n}{ticker}\PYG{p}{,}
                    \PYG{l+s+s2}{\PYGZdq{}}\PYG{l+s+s2}{PE\PYGZus{}TRAILING}\PYG{l+s+s2}{\PYGZdq{}}\PYG{p}{:} \PYG{k+kc}{None}\PYG{p}{,}
                    \PYG{l+s+s2}{\PYGZdq{}}\PYG{l+s+s2}{PE\PYGZus{}FORWARD}\PYG{l+s+s2}{\PYGZdq{}}\PYG{p}{:} \PYG{k+kc}{None}\PYG{p}{,}
                    \PYG{l+s+s2}{\PYGZdq{}}\PYG{l+s+s2}{PRICE\PYGZus{}TO\PYGZus{}BOOK}\PYG{l+s+s2}{\PYGZdq{}}\PYG{p}{:} \PYG{k+kc}{None}\PYG{p}{,}
                    \PYG{l+s+s2}{\PYGZdq{}}\PYG{l+s+s2}{EV\PYGZus{}TO\PYGZus{}EBITDA}\PYG{l+s+s2}{\PYGZdq{}}\PYG{p}{:} \PYG{k+kc}{None}\PYG{p}{,}
                    \PYG{l+s+s2}{\PYGZdq{}}\PYG{l+s+s2}{DIVIDEND\PYGZus{}YIELD}\PYG{l+s+s2}{\PYGZdq{}}\PYG{p}{:} \PYG{k+kc}{None}\PYG{p}{,}
                    \PYG{l+s+s2}{\PYGZdq{}}\PYG{l+s+s2}{PAYOUT\PYGZus{}RATIO}\PYG{l+s+s2}{\PYGZdq{}}\PYG{p}{:} \PYG{k+kc}{None}\PYG{p}{,}
                    \PYG{l+s+s2}{\PYGZdq{}}\PYG{l+s+s2}{MARKET\PYGZus{}CAP}\PYG{l+s+s2}{\PYGZdq{}}\PYG{p}{:} \PYG{k+kc}{None}\PYG{p}{,}
                    \PYG{l+s+s2}{\PYGZdq{}}\PYG{l+s+s2}{ENTERPRISE\PYGZus{}VALUE}\PYG{l+s+s2}{\PYGZdq{}}\PYG{p}{:} \PYG{k+kc}{None}\PYG{p}{,}
                    \PYG{l+s+s2}{\PYGZdq{}}\PYG{l+s+s2}{SHARES\PYGZus{}OUTSTANDING}\PYG{l+s+s2}{\PYGZdq{}}\PYG{p}{:} \PYG{k+kc}{None}\PYG{p}{,}
                \PYG{p}{\PYGZcb{}}
\end{sphinxVerbatim}


\begin{savenotes}\sphinxattablestart
\sphinxthistablewithglobalstyle
\centering
\begin{tabulary}{\linewidth}[t]{TTTTTTTTTT}
\sphinxtoprule
\sphinxstyletheadfamily 
\sphinxAtStartPar
TICKER
&\sphinxstyletheadfamily 
\sphinxAtStartPar
PE\_TRAILING
&\sphinxstyletheadfamily 
\sphinxAtStartPar
PE\_FORWARD
&\sphinxstyletheadfamily 
\sphinxAtStartPar
PRICE\_TO\_BOOK
&\sphinxstyletheadfamily 
\sphinxAtStartPar
EV\_TO\_EBITDA
&\sphinxstyletheadfamily 
\sphinxAtStartPar
DIVIDEND\_YIELD
&\sphinxstyletheadfamily 
\sphinxAtStartPar
PAYOUT\_RATIO
&\sphinxstyletheadfamily 
\sphinxAtStartPar
MARKET\_CAP
&\sphinxstyletheadfamily 
\sphinxAtStartPar
ENTERPRISE\_VALUE
&\sphinxstyletheadfamily 
\sphinxAtStartPar
SHARES\_OUTSTANDING
\\
\sphinxmidrule
\sphinxtableatstartofbodyhook
\sphinxAtStartPar
A2A.MI
&
\sphinxAtStartPar
8.319231
&
\sphinxAtStartPar
12.016666
&
\sphinxAtStartPar
1.406372
&
\sphinxAtStartPar
6.135
&
\sphinxAtStartPar
4.62
&
\sphinxAtStartPar
0.3868
&
\sphinxAtStartPar
6767140864
&
\sphinxAtStartPar
12491133952
&
\sphinxAtStartPar
3128590080
\\
\sphinxhline
\sphinxAtStartPar
AAF.L
&
\sphinxAtStartPar
32.62857
&
\sphinxAtStartPar
19.033333
&
\sphinxAtStartPar
321.69016
&
\sphinxAtStartPar
6.584
&
\sphinxAtStartPar
2.15
&
\sphinxAtStartPar
0.71650004
&
\sphinxAtStartPar
8328902656
&
\sphinxAtStartPar
14142440448
&
\sphinxAtStartPar
3646629888
\\
\sphinxhline
\sphinxAtStartPar
AAL.L
&
\sphinxAtStartPar

&
\sphinxAtStartPar
15.484849
&
\sphinxAtStartPar
157.49245
&
\sphinxAtStartPar
6.569
&
\sphinxAtStartPar
0.94
&
\sphinxAtStartPar
5.6102004
&
\sphinxAtStartPar
27304775680
&
\sphinxAtStartPar
45547601920
&
\sphinxAtStartPar
1068680000
\\
\sphinxbottomrule
\end{tabulary}
\sphinxtableafterendhook\par
\sphinxattableend\end{savenotes}


\subsection{Análisis de Sostenibilidad ESG}
\label{\detokenize{DatosPreparacion:analisis-de-sostenibilidad-esg}}
\sphinxAtStartPar
ESG son las siglas de Environmental, Social and Governance (Medioambiental, Social y Gobernanza). Es un conjunto de criterios que se usan para evaluar a las empresas más allá de sus resultados financieros.
\begin{enumerate}
\sphinxsetlistlabels{\arabic}{enumi}{enumii}{}{.}%
\item {} 
\sphinxAtStartPar
Environmental (Medioambiental): mide el impacto que tiene la empresa sobre el medio ambiente. Ejemplos: emisiones de CO\(\sb{\text{2}}\), gestión de residuos, eficiencia energética, uso de energías renovables.

\item {} 
\sphinxAtStartPar
Social (Social): evalúa cómo la empresa se relaciona con empleados, clientes, comunidades y sociedad en general. Ejemplos: condiciones laborales, diversidad e inclusión, derechos humanos, impacto en la comunidad.

\item {} 
\sphinxAtStartPar
Governance (Gobernanza): analiza cómo se gestiona y dirige la empresa. Ejemplos: independencia del consejo de administración, ética corporativa, transparencia, políticas contra la corrupción.

\end{enumerate}

\begin{sphinxVerbatim}[commandchars=\\\{\},numbers=left,firstnumber=1,stepnumber=1]
\PYG{k}{def}\PYG{+w}{ }\PYG{n+nf}{fetch\PYGZus{}esg\PYGZus{}one}\PYG{p}{(}\PYG{n}{ticker}\PYG{p}{:} \PYG{n+nb}{str}\PYG{p}{,} \PYG{n}{retries}\PYG{p}{:} \PYG{n+nb}{int} \PYG{o}{=} \PYG{l+m+mi}{2}\PYG{p}{,} \PYG{n}{pause}\PYG{p}{:} \PYG{n+nb}{float} \PYG{o}{=} \PYG{l+m+mf}{0.5}\PYG{p}{)} \PYG{o}{\PYGZhy{}}\PYG{o}{\PYGZgt{}} \PYG{n+nb}{dict}\PYG{p}{:}
    \PYG{k}{for} \PYG{n}{i} \PYG{o+ow}{in} \PYG{n+nb}{range}\PYG{p}{(}\PYG{n}{retries}\PYG{p}{)}\PYG{p}{:}
        \PYG{k}{try}\PYG{p}{:}
            \PYG{n}{tk} \PYG{o}{=} \PYG{n}{yf}\PYG{o}{.}\PYG{n}{Ticker}\PYG{p}{(}\PYG{n}{ticker}\PYG{p}{)}
            \PYG{k}{try}\PYG{p}{:}
                \PYG{n}{sust} \PYG{o}{=} \PYG{n}{tk}\PYG{o}{.}\PYG{n}{sustainability}
            \PYG{k}{except} \PYG{n+ne}{Exception}\PYG{p}{:}
                \PYG{n}{sust} \PYG{o}{=} \PYG{k+kc}{None}

            \PYG{n}{row} \PYG{o}{=} \PYG{p}{\PYGZob{}}\PYG{l+s+s2}{\PYGZdq{}}\PYG{l+s+s2}{TICKER}\PYG{l+s+s2}{\PYGZdq{}}\PYG{p}{:} \PYG{n}{ticker}\PYG{p}{,} \PYG{l+s+s2}{\PYGZdq{}}\PYG{l+s+s2}{HAS\PYGZus{}ESG}\PYG{l+s+s2}{\PYGZdq{}}\PYG{p}{:} \PYG{k+kc}{False}\PYG{p}{,}
                   \PYG{l+s+s2}{\PYGZdq{}}\PYG{l+s+s2}{TOTAL\PYGZus{}ESG}\PYG{l+s+s2}{\PYGZdq{}}\PYG{p}{:} \PYG{k+kc}{None}\PYG{p}{,} \PYG{l+s+s2}{\PYGZdq{}}\PYG{l+s+s2}{ENVIRONMENTAL}\PYG{l+s+s2}{\PYGZdq{}}\PYG{p}{:} \PYG{k+kc}{None}\PYG{p}{,} \PYG{l+s+s2}{\PYGZdq{}}\PYG{l+s+s2}{SOCIAL}\PYG{l+s+s2}{\PYGZdq{}}\PYG{p}{:} \PYG{k+kc}{None}\PYG{p}{,} \PYG{l+s+s2}{\PYGZdq{}}\PYG{l+s+s2}{GOVERNANCE}\PYG{l+s+s2}{\PYGZdq{}}\PYG{p}{:} \PYG{k+kc}{None}\PYG{p}{,} \PYG{l+s+s2}{\PYGZdq{}}\PYG{l+s+s2}{CONTROVERSY}\PYG{l+s+s2}{\PYGZdq{}}\PYG{p}{:} \PYG{k+kc}{None}\PYG{p}{\PYGZcb{}}

            \PYG{k}{if} \PYG{n}{sust} \PYG{o+ow}{is} \PYG{o+ow}{not} \PYG{k+kc}{None} \PYG{o+ow}{and} \PYG{o+ow}{not} \PYG{n}{sust}\PYG{o}{.}\PYG{n}{empty}\PYG{p}{:}
                \PYG{n}{any\PYGZus{}val} \PYG{o}{=} \PYG{k+kc}{False}
                \PYG{k}{for} \PYG{n}{k\PYGZus{}src}\PYG{p}{,} \PYG{n}{k\PYGZus{}dst} \PYG{o+ow}{in} \PYG{n}{ESG\PYGZus{}KEYS}\PYG{o}{.}\PYG{n}{items}\PYG{p}{(}\PYG{p}{)}\PYG{p}{:}
                    \PYG{n}{v} \PYG{o}{=} \PYG{n}{\PYGZus{}get\PYGZus{}from\PYGZus{}any\PYGZus{}shape}\PYG{p}{(}\PYG{n}{sust}\PYG{p}{,} \PYG{n}{k\PYGZus{}src}\PYG{p}{)}
                    \PYG{n}{row}\PYG{p}{[}\PYG{n}{k\PYGZus{}dst}\PYG{p}{]} \PYG{o}{=} \PYG{n+nb}{float}\PYG{p}{(}\PYG{n}{v}\PYG{p}{)} \PYG{k}{if} \PYG{n}{v} \PYG{o+ow}{is} \PYG{o+ow}{not} \PYG{k+kc}{None} \PYG{k}{else} \PYG{k+kc}{None}
                    \PYG{n}{any\PYGZus{}val} \PYG{o}{=} \PYG{n}{any\PYGZus{}val} \PYG{o+ow}{or} \PYG{p}{(}\PYG{n}{v} \PYG{o+ow}{is} \PYG{o+ow}{not} \PYG{k+kc}{None}\PYG{p}{)}
                \PYG{n}{row}\PYG{p}{[}\PYG{l+s+s2}{\PYGZdq{}}\PYG{l+s+s2}{HAS\PYGZus{}ESG}\PYG{l+s+s2}{\PYGZdq{}}\PYG{p}{]} \PYG{o}{=} \PYG{n+nb}{bool}\PYG{p}{(}\PYG{n}{any\PYGZus{}val}\PYG{p}{)}
            \PYG{k}{return} \PYG{n}{row}
        \PYG{k}{except} \PYG{n+ne}{Exception}\PYG{p}{:}
            \PYG{k}{if} \PYG{n}{i} \PYG{o}{\PYGZlt{}} \PYG{n}{retries} \PYG{o}{\PYGZhy{}} \PYG{l+m+mi}{1}\PYG{p}{:}
                \PYG{n}{time}\PYG{o}{.}\PYG{n}{sleep}\PYG{p}{(}\PYG{n}{pause} \PYG{o}{*} \PYG{p}{(}\PYG{n}{i} \PYG{o}{+} \PYG{l+m+mi}{1}\PYG{p}{)}\PYG{p}{)}
            \PYG{k}{else}\PYG{p}{:}
                \PYG{k}{return} \PYG{p}{\PYGZob{}}\PYG{l+s+s2}{\PYGZdq{}}\PYG{l+s+s2}{TICKER}\PYG{l+s+s2}{\PYGZdq{}}\PYG{p}{:} \PYG{n}{ticker}\PYG{p}{,} \PYG{l+s+s2}{\PYGZdq{}}\PYG{l+s+s2}{HAS\PYGZus{}ESG}\PYG{l+s+s2}{\PYGZdq{}}\PYG{p}{:} \PYG{k+kc}{False}\PYG{p}{,}
                        \PYG{l+s+s2}{\PYGZdq{}}\PYG{l+s+s2}{TOTAL\PYGZus{}ESG}\PYG{l+s+s2}{\PYGZdq{}}\PYG{p}{:} \PYG{k+kc}{None}\PYG{p}{,} \PYG{l+s+s2}{\PYGZdq{}}\PYG{l+s+s2}{ENVIRONMENTAL}\PYG{l+s+s2}{\PYGZdq{}}\PYG{p}{:} \PYG{k+kc}{None}\PYG{p}{,} \PYG{l+s+s2}{\PYGZdq{}}\PYG{l+s+s2}{SOCIAL}\PYG{l+s+s2}{\PYGZdq{}}\PYG{p}{:} \PYG{k+kc}{None}\PYG{p}{,} \PYG{l+s+s2}{\PYGZdq{}}\PYG{l+s+s2}{GOVERNANCE}\PYG{l+s+s2}{\PYGZdq{}}\PYG{p}{:} \PYG{k+kc}{None}\PYG{p}{,} \PYG{l+s+s2}{\PYGZdq{}}\PYG{l+s+s2}{CONTROVERSY}\PYG{l+s+s2}{\PYGZdq{}}\PYG{p}{:} \PYG{k+kc}{None}\PYG{p}{\PYGZcb{}}
\end{sphinxVerbatim}


\begin{savenotes}\sphinxattablestart
\sphinxthistablewithglobalstyle
\centering
\begin{tabulary}{\linewidth}[t]{TTTTTTT}
\sphinxtoprule
\sphinxstyletheadfamily 
\sphinxAtStartPar
TICKER
&\sphinxstyletheadfamily 
\sphinxAtStartPar
HAS\_ESG
&\sphinxstyletheadfamily 
\sphinxAtStartPar
TOTAL\_ESG
&\sphinxstyletheadfamily 
\sphinxAtStartPar
ENVIRONMENTAL
&\sphinxstyletheadfamily 
\sphinxAtStartPar
SOCIAL
&\sphinxstyletheadfamily 
\sphinxAtStartPar
GOVERNANCE
&\sphinxstyletheadfamily 
\sphinxAtStartPar
CONTROVERSY
\\
\sphinxmidrule
\sphinxtableatstartofbodyhook
\sphinxAtStartPar
A2A.MI
&
\sphinxAtStartPar
TRUE
&
\sphinxAtStartPar
20.10
&
\sphinxAtStartPar
10.59
&
\sphinxAtStartPar
4.90
&
\sphinxAtStartPar
4.61
&
\sphinxAtStartPar
1
\\
\sphinxhline
\sphinxAtStartPar
AAF.L
&
\sphinxAtStartPar
TRUE
&
\sphinxAtStartPar
22.69
&
\sphinxAtStartPar
6.52
&
\sphinxAtStartPar
11.82
&
\sphinxAtStartPar
4.35
&
\sphinxAtStartPar
2
\\
\sphinxhline
\sphinxAtStartPar
AAL.L
&
\sphinxAtStartPar
TRUE
&
\sphinxAtStartPar
25.79
&
\sphinxAtStartPar
14.22
&
\sphinxAtStartPar
8.96
&
\sphinxAtStartPar
2.62
&
\sphinxAtStartPar
3
\\
\sphinxbottomrule
\end{tabulary}
\sphinxtableafterendhook\par
\sphinxattableend\end{savenotes}

\sphinxstepscope


\chapter{Automatización Financiera}
\label{\detokenize{AutomatizacionFinanciera:automatizacion-financiera}}\label{\detokenize{AutomatizacionFinanciera::doc}}

\section{Modelos de análisis de sentimiento}
\label{\detokenize{AutomatizacionFinanciera:modelos-de-analisis-de-sentimiento}}
\sphinxAtStartPar
Se ha decidido utilizar un modelo predictivo para estimar la probabilidad de que un titular sea \sphinxstyleemphasis{negativo, positivo o neutral}. El modelo seleccionado es \sphinxstylestrong{\sphinxcode{\sphinxupquote{distilroberta\sphinxhyphen{}finetuned\sphinxhyphen{}financial\sphinxhyphen{}news\sphinxhyphen{}sentiment\sphinxhyphen{}analysis}}}, especializado en el ámbito financiero en idioma inglés, elegido por su buen rendimiento en clasificación.


\subsection{Modelo en inglés: \sphinxstyleemphasis{mrm8488/distilroberta\sphinxhyphen{}finetuned\sphinxhyphen{}financial\sphinxhyphen{}news\sphinxhyphen{}sentiment\sphinxhyphen{}analysis}}
\label{\detokenize{AutomatizacionFinanciera:modelo-en-ingles-mrm8488-distilroberta-finetuned-financial-news-sentiment-analysis}}
\sphinxAtStartPar
Este modelo corresponde a una versión ajustada de \sphinxstylestrong{DistilRoBERTa\sphinxhyphen{}base}, entrenada específicamente en el conjunto de datos \sphinxstyleemphasis{Financial PhraseBank}, que contiene 4.840 frases procedentes de noticias financieras en inglés. Dicho corpus fue anotado por entre 5 y 8 expertos, categorizando cada enunciado según su polaridad de sentimiento (positivo, negativo o neutral).

\sphinxAtStartPar
A nivel arquitectónico, DistilRoBERTa es una versión comprimida de RoBERTa\sphinxhyphen{}base que conserva su rendimiento con menor complejidad computacional. Consta de \sphinxstylestrong{6 capas, 768 dimensiones y 12 cabezales de atención}, sumando un total aproximado de \sphinxstylestrong{82 millones de parámetros} (frente a los 125 millones de RoBERTa\sphinxhyphen{}base). Esta reducción permite que el modelo sea, en promedio, \sphinxstylestrong{el doble de rápido} en comparación con RoBERTa\sphinxhyphen{}base, manteniendo un desempeño competitivo.

\sphinxAtStartPar
En cuanto a resultados, el modelo alcanzó un \sphinxstylestrong{accuracy del 98,23 \%} y una pérdida de validación de \sphinxstylestrong{0,1116} en el conjunto de evaluación, lo que demuestra un desempeño sobresaliente en la clasificación de sentimiento en el ámbito financiero.

\sphinxAtStartPar
La principal ventaja de este modelo es su \sphinxstylestrong{alta especialización en finanzas}, lo que le permite detectar matices en titulares y frases económicas con gran precisión. No obstante, presenta la limitación de estar restringido exclusivamente al inglés.

\sphinxAtStartPar
\sphinxhref{https://huggingface.co/mrm8488/distilroberta-finetuned-financial-news-sentiment-analysis}{Ver HuggingFace \sphinxhyphen{} Modelo Sentimento Inglés }

\sphinxAtStartPar
Ahora podemos ver el flujo de trabajo para procesar titulares de noticias financieras: desde su extracción y verificación, pasando por el análisis de sentimiento con un modelo predictivo, hasta la integración de los resultados en un DataLake, garantizando que solo se analicen y almacenen los titulares nuevos.

\begin{figure}[htbp]
\centering
\capstart

\noindent\sphinxincludegraphics[width=1.100\linewidth]{{DigSentimientos}.png}
\caption{\sphinxstylestrong{Figura 12.} Flujograma del modelo de scrapping}\label{\detokenize{AutomatizacionFinanciera:id1}}\end{figure}


\subsection{Automatización del flujo de trabajo con n8n}
\label{\detokenize{AutomatizacionFinanciera:automatizacion-del-flujo-de-trabajo-con-n8n}}
\sphinxAtStartPar
En la automatización de la ingesta y el procesamiento de datos se ha empleado \sphinxstylestrong{n8n}, una herramienta de \sphinxstyleemphasis{workflow automation} de código abierto. Su función principal es permitir la integración entre múltiples servicios, APIs y bases de datos mediante la creación de flujos de trabajo visuales que se ejecutan de manera automática en función de determinados eventos o programaciones. A partir del trigger que en nuestro caso de tiempo (00:00 todos los días), los nodos se van ejecutando en orden lógico, permitiendo que la información fluya automáticamente entre servicios sin intervención manual. En conjunto, n8n se convierte en un componente clave para la \sphinxstylestrong{orquestación de datos} en este proyecto, permitiendo que el análisis de sentimiento y financiero se sustente en información

\begin{figure}[htbp]
\centering
\capstart

\noindent\sphinxincludegraphics[width=1.000\linewidth]{{FlujoN8N}.jpeg}
\caption{\sphinxstylestrong{Figura 13.} Flujograma del modelo de scrapping en N8N}\label{\detokenize{AutomatizacionFinanciera:id2}}\end{figure}


\subsection{Flujo de utilizacion de los modelos}
\label{\detokenize{AutomatizacionFinanciera:flujo-de-utilizacion-de-los-modelos}}
\sphinxAtStartPar
Dentro del \sphinxstyleemphasis{datalake} definido en el módulo de \sphinxstylestrong{Datos y Preparación}, se encuentra la tabla unificada denominada \sphinxcode{\sphinxupquote{NOTICIAS\_ANALIZADAS}}. A partir de esta base de datos se extraen los titulares procedentes de los distintos noticieros. Posteriormente, se aplica el modelo de predicción correspondiente, previamente entrenado para el análisis de sentimiento.


\subsection{Generación de las columnas con tipo de títulos}
\label{\detokenize{AutomatizacionFinanciera:generacion-de-las-columnas-con-tipo-de-titulos}}
\sphinxAtStartPar
Una vez aplicado el modelo de análisis de sentimiento, se generan cuatro variables principales: \sphinxcode{\sphinxupquote{SENTIMIENTO\_RESULTADO}}, \sphinxcode{\sphinxupquote{PROBABILIDAD\_POSITIVO}}, \sphinxcode{\sphinxupquote{PROBABILIDAD\_NEGATIVA}} y \sphinxcode{\sphinxupquote{PROBABILIDAD\_NEUTRAL}}. Estas variables permiten determinar de manera objetiva la clasificación final del texto en función de su polaridad. Posteriormente, los resultados se incorporan a la tabla principal de \sphinxstylestrong{Noticias\_Analizadas} y se almacenan nuevamente en Snowflake, lo que garantiza su disponibilidad para futuros análisis, tanto de carácter técnico como de integración con otros indicadores financieros.

\begin{figure}[htbp]
\centering
\capstart

\noindent\sphinxincludegraphics[width=1.000\linewidth]{{Correo_Titulares}.png}
\caption{\sphinxstylestrong{Figura 14.} Notificaciones automáticas de la actualización de la base de datos y aplicación del modelo a los titulares}\label{\detokenize{AutomatizacionFinanciera:id3}}\end{figure}


\section{Análisis  de Frontera de eficiencia}
\label{\detokenize{AutomatizacionFinanciera:analisis-de-frontera-de-eficiencia}}
\sphinxAtStartPar
Para la selección de nuestra cartera emplearemos el modelo de media\sphinxhyphen{}varianza de Markowitz, considerado la base de la Teoría Moderna de Carteras. Este enfoque es ampliamente utilizado en finanzas porque permite encontrar la combinación óptima de activos equilibrando riesgo y rendimiento esperado.
El procedimiento consiste en simular 50.000 carteras aleatorias con diferentes ponderaciones de activos. Posteriormente, se identifican aquellas que cumplen con dos criterios clave:
\begin{itemize}
\item {} 
\sphinxAtStartPar
Cartera de mínima varianza: la que presenta el menor nivel de riesgo posible.

\item {} 
\sphinxAtStartPar
Cartera con ratio de Sharpe máximo: la que ofrece la mejor relación entre rendimiento y riesgo ajustado por la tasa libre de riesgo.

\sphinxAtStartPar
Ver teoría en {\hyperref[\detokenize{MarcoTeorico:modelo-de-markowitz}]{\sphinxcrossref{\DUrole{std}{\DUrole{std-ref}{Modelo de Markowitz}}}}}

\end{itemize}

\sphinxAtStartPar
El primer aspecto para considerar es que los resultados de nuestro modelo se generan siempre con datos actualizados hasta el día anterior en todas las bases de datos. Dichos resultados se almacenan en un datalake implementado en Snowflake, el cual se encuentra en constante actualización. Por ello, la primera fase de nuestro análisis consiste en establecer la conexión con Snowflake y asegurar la correcta creación, configuración y mantenimiento del datalake.

\sphinxAtStartPar
Una vez completada esta etapa, el modelo se vincula con las acciones seleccionadas por el usuario en función de su análisis fundamental. Para cada activo elegido, se verifican las tendencias históricas y se realizan diferentes simulaciones con variaciones en los porcentajes de asignación.

\sphinxAtStartPar
De este modo, el modelo permite seleccionar la combinación más eficiente considerando no solo los criterios de diversificación cuantitativa, sino también el análisis fundamental, el análisis descriptivo de los datos históricos y el análisis de sentimiento obtenido a partir de noticias sectoriales e industriales.

\sphinxAtStartPar
El cálculo de los rendimientos de los activos es una etapa fundamental previa a cualquier análisis de carteras. A partir de los precios históricos, se obtienen las variaciones porcentuales que reflejan cómo evoluciona el valor de cada activo en el tiempo.

\begin{sphinxVerbatim}[commandchars=\\\{\},numbers=left,firstnumber=1,stepnumber=1]
\PYG{n}{rendimientos} \PYG{o}{=} \PYG{p}{(}\PYG{n}{prices} \PYG{o}{\PYGZhy{}} \PYG{n}{prices}\PYG{o}{.}\PYG{n}{shift}\PYG{p}{(}\PYG{l+m+mi}{1}\PYG{p}{)}\PYG{p}{)} \PYG{o}{/} \PYG{n}{prices}\PYG{o}{.}\PYG{n}{shift}\PYG{p}{(}\PYG{l+m+mi}{1}\PYG{p}{)}
\end{sphinxVerbatim}

\sphinxAtStartPar
Una vez obtenidos los datos necesarios para el análisis principal, procedemos a la simulación de 50.000 portafolios aleatorios. Este proceso nos permite explorar un amplio rango de combinaciones posibles entre los activos y, de este modo, identificar aquellas carteras que ofrecen el mejor equilibrio entre riesgo y rendimiento esperado.

\begin{sphinxVerbatim}[commandchars=\\\{\},numbers=left,firstnumber=1,stepnumber=1]
\PYG{k}{for} \PYG{n}{x} \PYG{o+ow}{in} \PYG{n+nb}{range}\PYG{p}{(}\PYG{l+m+mi}{50000}\PYG{p}{)}\PYG{p}{:}
    \PYG{n}{pesos} \PYG{o}{=} \PYG{n}{np}\PYG{o}{.}\PYG{n}{random}\PYG{o}{.}\PYG{n}{random}\PYG{p}{(}\PYG{n}{numero\PYGZus{}activos}\PYG{p}{)}
    \PYG{n}{pesos} \PYG{o}{/}\PYG{o}{=} \PYG{n}{np}\PYG{o}{.}\PYG{n}{sum}\PYG{p}{(}\PYG{n}{pesos}\PYG{p}{)}
    \PYG{n}{rendimiento} \PYG{o}{=} \PYG{n}{np}\PYG{o}{.}\PYG{n}{sum}\PYG{p}{(}\PYG{n}{pesos} \PYG{o}{*} \PYG{n}{rendimientos}\PYG{o}{.}\PYG{n}{mean}\PYG{p}{(}\PYG{p}{)}\PYG{p}{)} \PYG{o}{*} \PYG{l+m+mi}{252}
    \PYG{n}{riesgo} \PYG{o}{=} \PYG{n}{np}\PYG{o}{.}\PYG{n}{sqrt}\PYG{p}{(}\PYG{n}{np}\PYG{o}{.}\PYG{n}{dot}\PYG{p}{(}\PYG{n}{pesos}\PYG{o}{.}\PYG{n}{T}\PYG{p}{,} \PYG{n}{np}\PYG{o}{.}\PYG{n}{dot}\PYG{p}{(}\PYG{n}{rendimientos}\PYG{o}{.}\PYG{n}{cov}\PYG{p}{(}\PYG{p}{)} \PYG{o}{*} \PYG{l+m+mi}{252}\PYG{p}{,} \PYG{n}{pesos}\PYG{p}{)}\PYG{p}{)}\PYG{p}{)}
    \PYG{n}{rendimientos\PYGZus{}portafolio}\PYG{o}{.}\PYG{n}{append}\PYG{p}{(}\PYG{n}{rendimiento}\PYG{p}{)}
    \PYG{n}{riesgo\PYGZus{}portafolio}\PYG{o}{.}\PYG{n}{append}\PYG{p}{(}\PYG{n}{riesgo}\PYG{p}{)}
    \PYG{n}{peso\PYGZus{}portafolio}\PYG{o}{.}\PYG{n}{append}\PYG{p}{(}\PYG{n}{pesos}\PYG{p}{)}
\end{sphinxVerbatim}

\sphinxAtStartPar
Con este procedimiento se han generado 50.000 carteras simuladas, cada una con su respectivo nivel de riesgo y rendimiento esperado. A partir de estos resultados, y considerando una tasa libre de riesgo del 3\%, se calcula el ratio de Sharpe, lo que nos permite identificar la combinación de acciones más eficiente en términos de la relación rentabilidad\textendash{}riesgo. La siguiente tabla muestra algunos ejemplos representativos de estas carteras:


\begin{savenotes}\sphinxattablestart
\sphinxthistablewithglobalstyle
\centering
\begin{tabulary}{\linewidth}[t]{TTTTTTTTTTTTTTTT}
\sphinxtoprule
\sphinxstyletheadfamily 
\sphinxAtStartPar
Rendimientos
&\sphinxstyletheadfamily 
\sphinxAtStartPar
Riesgos
&\sphinxstyletheadfamily 
\sphinxAtStartPar
PesoACS.MC
&\sphinxstyletheadfamily 
\sphinxAtStartPar
PesoAENA.MC
&\sphinxstyletheadfamily 
\sphinxAtStartPar
PesoBBVA.MC
&\sphinxstyletheadfamily 
\sphinxAtStartPar
PesoCABK.MC
&\sphinxstyletheadfamily 
\sphinxAtStartPar
PesoELE.MC
&\sphinxstyletheadfamily 
\sphinxAtStartPar
PesoENG.MC
&\sphinxstyletheadfamily 
\sphinxAtStartPar
PesoFER.MC
&\sphinxstyletheadfamily 
\sphinxAtStartPar
PesoIAG.MC
&\sphinxstyletheadfamily 
\sphinxAtStartPar
PesoIBE.MC
&\sphinxstyletheadfamily 
\sphinxAtStartPar
PesoITX.MC
&\sphinxstyletheadfamily 
\sphinxAtStartPar
PesoMAP.MC
&\sphinxstyletheadfamily 
\sphinxAtStartPar
PesoREP.MC
&\sphinxstyletheadfamily 
\sphinxAtStartPar
PesoSAN.MC
&\sphinxstyletheadfamily 
\sphinxAtStartPar
PesoTEF.MC
\\
\sphinxmidrule
\sphinxtableatstartofbodyhook
\sphinxAtStartPar
0.094535
&
\sphinxAtStartPar
0.236939
&
\sphinxAtStartPar
0.056633
&
\sphinxAtStartPar
0.050586
&
\sphinxAtStartPar
0.146062
&
\sphinxAtStartPar
0.077725
&
\sphinxAtStartPar
0.046691
&
\sphinxAtStartPar
0.035503
&
\sphinxAtStartPar
0.113892
&
\sphinxAtStartPar
0.069035
&
\sphinxAtStartPar
0.062491
&
\sphinxAtStartPar
0.131826
&
\sphinxAtStartPar
0.036659
&
\sphinxAtStartPar
0.013266
&
\sphinxAtStartPar
0.052721
&
\sphinxAtStartPar
0.106909
\\
\sphinxhline
\sphinxAtStartPar
0.088686
&
\sphinxAtStartPar
0.247779
&
\sphinxAtStartPar
0.150650
&
\sphinxAtStartPar
0.047172
&
\sphinxAtStartPar
0.018697
&
\sphinxAtStartPar
0.065942
&
\sphinxAtStartPar
0.009546
&
\sphinxAtStartPar
0.004347
&
\sphinxAtStartPar
0.037379
&
\sphinxAtStartPar
0.134567
&
\sphinxAtStartPar
0.145345
&
\sphinxAtStartPar
0.096339
&
\sphinxAtStartPar
0.017419
&
\sphinxAtStartPar
0.038019
&
\sphinxAtStartPar
0.119666
&
\sphinxAtStartPar
0.114913
\\
\sphinxhline
\sphinxAtStartPar
0.076873
&
\sphinxAtStartPar
0.232206
&
\sphinxAtStartPar
0.130143
&
\sphinxAtStartPar
0.112952
&
\sphinxAtStartPar
0.103699
&
\sphinxAtStartPar
0.046427
&
\sphinxAtStartPar
0.117658
&
\sphinxAtStartPar
0.066646
&
\sphinxAtStartPar
0.027429
&
\sphinxAtStartPar
0.084714
&
\sphinxAtStartPar
0.030029
&
\sphinxAtStartPar
0.094975
&
\sphinxAtStartPar
0.118723
&
\sphinxAtStartPar
0.011767
&
\sphinxAtStartPar
0.007015
&
\sphinxAtStartPar
0.047825
\\
\sphinxbottomrule
\end{tabulary}
\sphinxtableafterendhook\par
\sphinxattableend\end{savenotes}

\begin{sphinxVerbatim}[commandchars=\\\{\},numbers=left,firstnumber=1,stepnumber=1]
\PYG{n}{varianza\PYGZus{}minima} \PYG{o}{=} \PYG{n}{Matriz\PYGZus{}portafolios}\PYG{o}{.}\PYG{n}{iloc}\PYG{p}{[}\PYG{n}{Matriz\PYGZus{}portafolios}\PYG{p}{[}\PYG{l+s+s1}{\PYGZsq{}}\PYG{l+s+s1}{Riesgos}\PYG{l+s+s1}{\PYGZsq{}}\PYG{p}{]}\PYG{o}{.}\PYG{n}{idxmin}\PYG{p}{(}\PYG{p}{)}\PYG{p}{]}

\PYG{n}{risk\PYGZus{}free} \PYG{o}{=} \PYG{l+m+mf}{0.03}
\PYG{n}{portafolio\PYGZus{}optimo} \PYG{o}{=} \PYG{n}{Matriz\PYGZus{}portafolios}\PYG{o}{.}\PYG{n}{iloc}\PYG{p}{[}
    \PYG{p}{(}\PYG{p}{(}\PYG{n}{Matriz\PYGZus{}portafolios}\PYG{p}{[}\PYG{l+s+s1}{\PYGZsq{}}\PYG{l+s+s1}{Rendimientos}\PYG{l+s+s1}{\PYGZsq{}}\PYG{p}{]} \PYG{o}{\PYGZhy{}} \PYG{n}{risk\PYGZus{}free}\PYG{p}{)} \PYG{o}{/} \PYG{n}{Matriz\PYGZus{}portafolios}\PYG{p}{[}\PYG{l+s+s1}{\PYGZsq{}}\PYG{l+s+s1}{Riesgos}\PYG{l+s+s1}{\PYGZsq{}}\PYG{p}{]}\PYG{p}{)}\PYG{o}{.}\PYG{n}{idxmax}\PYG{p}{(}\PYG{p}{)}
\PYG{p}{]}
\end{sphinxVerbatim}


\subsection{Gráfica de la frontera de eficiencia}
\label{\detokenize{AutomatizacionFinanciera:grafica-de-la-frontera-de-eficiencia}}
\sphinxAtStartPar
Este gráfico es un ejemplo con una cartera del IBEX35 el cual representa la frontera eficiente obtenida a partir de la simulación de 50.000 combinaciones de activos. En el eje horizontal tenemos el riesgo del portafolio medido como volatilidad y en el eje vertical tenemos el rendimiento esperado por el portafolio. La nube de puntos son los portafolios creados aleatoriamente que combina cada uno de los pesos de las acciones de cada portafolio.

\sphinxAtStartPar
La estrella verde, abajo a la izquierda, es la cartera con el menor riesgo posible, aunque su rentabilidad es moderada y la estrella azul, arriba a la derecha es la cartera más eficiente en términos de relación riesgo\textendash{}rentabilidad, considerando la tasa libre de riesgo.

\begin{figure}[htbp]
\centering
\capstart

\noindent\sphinxincludegraphics{{FronteraEficiencia_porta}.gif}
\caption{\sphinxstylestrong{Figura 15.} Muestra de frontera de eficiencia para la elección de la cartera más eficiente}\label{\detokenize{AutomatizacionFinanciera:id4}}\end{figure}

\sphinxstepscope


\chapter{Implementación de AsesorIA}
\label{\detokenize{ImplementacionAcesorIA:implementacion-de-asesoria}}\label{\detokenize{ImplementacionAcesorIA::doc}}
\sphinxAtStartPar
Se desarrolló un flujo automatizado en N8N utilizando Telegram BotFather, la herramienta oficial de Telegram para la creación de bots. El objetivo principal del robot es permitir a los usuarios consultar información almacenada en la base de datos Snowflake, empleando consultas SQL generadas y validadas de manera automática con apoyo de modelos de lenguaje de OpenAI.


\section{Flujo del Proceso}
\label{\detokenize{ImplementacionAcesorIA:flujo-del-proceso}}
\sphinxAtStartPar
El proceso comienza cuando el usuario escribe al bot en Telegram. Ese mensaje llega a N8N, donde OpenAI lo interpreta y, si corresponde, genera una consulta SQL. Si la consulta no es válida, el sistema avisa; si lo es, se ejecuta directamente en Snowflake. Los resultados se organizan y se analizan automáticamente para generar un resumen con la interpretación y la propia consulta utilizada. Finalmente, todo se transforma en un formato claro y se devuelve al usuario en Telegram, de manera que el bot actúa como un asistente inteligente en la interpretación de datos.

\sphinxAtStartPar
\sphinxincludegraphics{{FlujoRobot_n8n}.jpeg}

\sphinxAtStartPar
\sphinxincludegraphics{{FlujoRobot}.png}


\section{Beneficios del Proceso}
\label{\detokenize{ImplementacionAcesorIA:beneficios-del-proceso}}\begin{itemize}
\item {} 
\sphinxAtStartPar
Automatización completa de las consultas en Snowflake.

\item {} 
\sphinxAtStartPar
Acceso inmediato a la información a través de Telegram.

\item {} 
\sphinxAtStartPar
Análisis asistido por IA, que interpreta y explica los resultados.

\end{itemize}

\sphinxAtStartPar
El robot implementado integra de manera efectiva N8N, Telegram, Snowflake y OpenAI, logrando un flujo robusto para la consulta y análisis de datos en tiempo real. Este sistema no solo proporciona respuestas automáticas, sino que también añade valor al usuario al interpretar la información, consolidando un asesor virtual de inteligencia artificial (AsesorIA) dentro de un entorno de mensajería instantánea.

\sphinxAtStartPar
\sphinxincludegraphics{{Robot}.jpeg}

\sphinxAtStartPar
\sphinxhref{https://t.me/TFMGRUPO4\_BOT}{Datos del Robot Financiero}


\section{Visualización}
\label{\detokenize{ImplementacionAcesorIA:visualizacion}}
\sphinxAtStartPar
Iniciamos un recorrido por un panel de informacion financiera donde te tendremos la vision global de los mercados bursátiles europeos, combinando indicadores geográficos, sectoriales y financieros.
\begin{itemize}
\item {} 
\sphinxAtStartPar
\sphinxstylestrong{Mapa geográfico}: Muestra la distribución de los mercados por país, resaltando las plazas bursátiles más relevantes de Europa.

\item {} 
\sphinxAtStartPar
\sphinxstylestrong{Recuento por Exchange}: Representa el número de compañías cotizadas en cada bolsa (LSE, GER, MIL, PAR, MCE, STO, AMS, EBS), evidenciando el peso de cada mercado en términos de participación empresarial.

\item {} 
\sphinxAtStartPar
\sphinxstylestrong{Capitalización por país}: Indica la magnitud del \sphinxstyleemphasis{Market Cap} acumulado, donde destacan Reino Unido y Suecia como los países con mayor volumen.

\item {} 
\sphinxAtStartPar
\sphinxstylestrong{Empleados vs. Market Cap}: Gráfico de dispersión que relaciona la cantidad de empleados con la capitalización de mercado, permitiendo identificar empresas de gran tamaño y alto valor en los distintos sectores.

\end{itemize}

\sphinxAtStartPar
En conjunto, esta visión proporciona un panorama comparativo entre países y mercados europeos, destacando tanto la relevancia geográfica como la magnitud económica de las empresas que los componen.

\sphinxAtStartPar
\sphinxhref{https://public.tableau.com/app/profile/julia.escudero.velasco/viz/TFM\_17581412187180/Indice}{Dashboard de Análisis Financiero en Tableau}

\begin{figure}[htbp]
\centering
\capstart

\noindent\sphinxincludegraphics{{visionmercados}.jpeg}
\caption{\sphinxstylestrong{Figura 16.} Visualización de los graficos del mercado}\label{\detokenize{ImplementacionAcesorIA:id1}}\end{figure}

\sphinxAtStartPar
Luego podemos ver un análisis por cada una de las empresas que el cliente quiera verificar
\begin{itemize}
\item {} 
\sphinxAtStartPar
\sphinxstylestrong{Boxplots de precios de cierre (2020\textendash{}2025)}: muestran la dispersión, volatilidad y tendencias anuales de los precios de la acción.

\item {} 
\sphinxAtStartPar
\sphinxstylestrong{Evolución histórica del precio}: refleja la trayectoria del valor en bolsa con tendencia creciente a pesar de episodios de volatilidad.

\end{itemize}

\sphinxAtStartPar
El tablero permite analizar tanto la solidez financiera como el comportamiento bursátil de la empresa, integrando métricas clave y evolución histórica para apoyar la toma de decisiones de inversión.

\begin{figure}[htbp]
\centering
\capstart

\noindent\sphinxincludegraphics{{analisisempresarial}.jpeg}
\caption{\sphinxstylestrong{Figura 17.} Análisis empresarial en gráficos}\label{\detokenize{ImplementacionAcesorIA:id2}}\end{figure}

\sphinxAtStartPar
Seguidamente tendremos el análisis por tickers. Este panel permite analizar y comparar el desempeño de distintas acciones a través de indicadores clave, relacionando riesgo, retorno y evolución temporal.
\begin{itemize}
\item {} 
\sphinxAtStartPar
\sphinxstylestrong{Riesgo vs. Retorno}: gráfico de dispersión que muestra el balance entre la volatilidad y el retorno medio de cada acción.
\begin{itemize}
\item {} 
\sphinxAtStartPar
Las burbujas representan a cada \sphinxstyleemphasis{ticker}, diferenciados por tamaño y color.

\item {} 
\sphinxAtStartPar
Permite identificar qué empresas logran mejores retornos ajustados al riesgo.

\end{itemize}

\item {} 
\sphinxAtStartPar
\sphinxstylestrong{Rendimiento acumulado}: evolución temporal del rendimiento compuesto de cada acción, agrupado por trimestre (2020T3\textendash{}2025T3).
\begin{itemize}
\item {} 
\sphinxAtStartPar
Refleja la dinámica de cada activo y su capacidad de mantener tendencia positiva o negativa en el tiempo.

\end{itemize}

\item {} 
\sphinxAtStartPar
\sphinxstylestrong{KPIs de Tickers}
La tabla inferior presenta indicadores clave de desempeño (\sphinxstyleemphasis{Key Performance Indicators}) para cada acción:

\item {} 
\sphinxAtStartPar
\sphinxstylestrong{Precios promedio}: apertura, cierre, máximo y mínimo.

\item {} 
\sphinxAtStartPar
\sphinxstylestrong{Volatilidad intradía}: medida de la variación dentro de una misma sesión.

\item {} 
\sphinxAtStartPar
\sphinxstylestrong{Volumen}: cantidad total negociada en el periodo.

\item {} 
\sphinxAtStartPar
\sphinxstylestrong{Varianza y desviación estándar}: métricas estadísticas que ayudan a dimensionar la dispersión de los precios.

\end{itemize}

\sphinxAtStartPar
En resumen, este tablero ofrece una perspectiva integral de varios activos, facilitando la comparación de su riesgo, rentabilidad y evolución histórica, lo que lo convierte en una herramienta útil para la toma de decisiones financieras.

\begin{figure}[htbp]
\centering
\capstart

\noindent\sphinxincludegraphics{{exploraciontickers}.jpeg}
\caption{\sphinxstylestrong{Figura 18.} Exploración Tickers}\label{\detokenize{ImplementacionAcesorIA:id3}}\end{figure}

\sphinxAtStartPar
Por último, este panel ofrece una visión global del sentimiento de las noticias financieras. Resume la proporción de titulares positivos, negativos y neutros, muestra ejemplos representativos y refleja qué medios aportan más información.

\begin{figure}[htbp]
\centering
\capstart

\noindent\sphinxincludegraphics{{analisisnoticias}.jpeg}
\caption{\sphinxstylestrong{Figura 19.} análisis de sentimiento por titular de noticieros}\label{\detokenize{ImplementacionAcesorIA:id4}}\end{figure}

\sphinxstepscope


\chapter{Conclusiones}
\label{\detokenize{Conclusiones:conclusiones}}\label{\detokenize{Conclusiones::doc}}
\sphinxAtStartPar
El proyecto desarrollado integra distintas tecnologías de \sphinxstylestrong{procesamiento de lenguaje natural (NLP)}, \sphinxstylestrong{automatización de flujos de trabajo} y \sphinxstylestrong{almacenamiento en la nube}, con el propósito de construir una herramienta capaz de facilitar el análisis financiero a partir de diferentes fuentes de datos. Durante el proceso se abordaron múltiples fases: desde la recolección de información mediante \sphinxstyleemphasis{scraping} de noticias y descarga de datos financieros con APIs, hasta la limpieza, transformación y almacenamiento de dicha información en \sphinxstylestrong{Snowflake}. Además, se implementaron módulos especializados para incorporar métricas \sphinxstylestrong{fundamentales} (estados financieros), \sphinxstylestrong{de mercado} (ratios bursátiles) y \sphinxstylestrong{ESG} (sostenibilidad), conformando así una base de datos integral.

\sphinxAtStartPar
Sobre esta infraestructura, se diseñó un flujo en \sphinxstylestrong{N8N} que permite interactuar con los datos a través de un \sphinxstylestrong{bot en Telegram}, el cual procesa las consultas de los usuarios con modelos de \sphinxstylestrong{OpenAI}, genera sentencias SQL, ejecuta las consultas en Snowflake y devuelve los resultados acompañados de un breve análisis interpretativo. Este enfoque no solo resuelve la parte técnica de la integración de datos, sino que también ofrece un \sphinxstylestrong{asesor virtual de inteligencia artificial (AsesorIA)} que democratiza el acceso a información financiera avanzada.

\sphinxAtStartPar
La experiencia adquirida durante la implementación permitió identificar tanto los \sphinxstylestrong{beneficios} como los \sphinxstylestrong{desafíos} de este tipo de soluciones, destacando aspectos técnicos, económicos y estratégicos que serán determinantes para la continuidad y evolución de la herramienta.

\sphinxAtStartPar
En síntesis, las principales conclusiones alcanzadas son:
\begin{itemize}
\item {} 
\sphinxAtStartPar
La implementación de la herramienta ha supuesto un reto importante en las fases de \sphinxstylestrong{obtención, limpieza y transformación de datos}, lo que evidencia la necesidad de procesos sólidos de ingeniería de datos en proyectos de este tipo.

\item {} 
\sphinxAtStartPar
La herramienta representa un aporte significativo al permitir que \sphinxstylestrong{usuarios sin conocimientos técnicos financieros} puedan iniciarse en el análisis de mercados, reduciendo barreras de entrada y facilitando la democratización del acceso a la información financiera.

\item {} 
\sphinxAtStartPar
Los \sphinxstylestrong{costos asociados a plataformas y servicios} como Hugging Face, OpenAI y Snowflake constituyen un factor crítico a considerar en el desarrollo, escalabilidad y sostenibilidad de la solución.

\item {} 
\sphinxAtStartPar
La \sphinxstylestrong{selección de un modelo NLP} debe realizarse en función de parámetros técnicos como \sphinxstylestrong{precisión y exactitud}, garantizando resultados fiables y adecuados al contexto financiero.

\item {} 
\sphinxAtStartPar
Para la \sphinxstylestrong{elaboración, implementación y desarrollo} de la herramienta es necesario contar con \sphinxstylestrong{conocimientos técnicos previos}, que permitan evaluar las tecnologías, optimizar su uso y asegurar la correcta integración de los diferentes componentes.

\end{itemize}

\sphinxstepscope


\chapter{Bibliografía}
\label{\detokenize{Bibliografia:bibliografia}}\label{\detokenize{Bibliografia::doc}}

\section{A.  Libros}
\label{\detokenize{Bibliografia:a-libros}}\begin{enumerate}
\sphinxsetlistlabels{\arabic}{enumi}{enumii}{}{.}%
\item {} 
\sphinxAtStartPar
Murphy, J. J. (2016). Análisis técnico de los mercados financieros (A. de Gispert Ramis, Trad.). Gestión 2000. ISBN 978‐84‐9875‐428‐5.

\item {} 
\sphinxAtStartPar
Cárpatos, J. L. (2014). Leones contra gacelas: Manual completo del especulador. Grupo Planeta (GBS). ISBN 8423419282

\item {} 
\sphinxAtStartPar
Ramírez Gil, C. M. (2021). Python para finanzas: Curso práctico. RA\sphinxhyphen{}MA Editorial. ISBN 978\sphinxhyphen{}84\sphinxhyphen{}18551\sphinxhyphen{}85\sphinxhyphen{}7

\item {} 
\sphinxAtStartPar
Cárpatos, J. L. (2014). Leones contra gacelas: Manual completo del especulador. Ediciones Deusto. ISBN 978\sphinxhyphen{}84\sphinxhyphen{}234\sphinxhyphen{}1928\sphinxhyphen{}9.

\end{enumerate}


\section{B. Cursos}
\label{\detokenize{Bibliografia:b-cursos}}\begin{enumerate}
\sphinxsetlistlabels{\arabic}{enumi}{enumii}{}{.}%
\item {} 
\sphinxAtStartPar
Garay, F. (2025). n8n Total: Build AI Agents \& Automate Workflows Without Code {[}Curso en línea{]}. Udemy. https://www.udemy.com/course/n8n\sphinxhyphen{}total/?couponCode=MT250915G1
Udemy

\end{enumerate}


\section{C. Paginas Webs}
\label{\detokenize{Bibliografia:c-paginas-webs}}\begin{enumerate}
\sphinxsetlistlabels{\arabic}{enumi}{enumii}{}{.}%
\item {} 
\sphinxAtStartPar
ABC. (2025, mayo 11). Cinco de cada diez pymes necesitaron financiación en 2024. \sphinxstyleemphasis{ABC}. https://www.abc.es/economia/cinco\sphinxhyphen{}diez\sphinxhyphen{}pymes\sphinxhyphen{}necesitaron\sphinxhyphen{}financiacion\sphinxhyphen{}2024\sphinxhyphen{}20250511174204\sphinxhyphen{}nt.html

\item {} 
\sphinxAtStartPar
Bloomberg. (2025, septiembre 18). Across Germany, France, Poland and the UK, Europe is becoming ungovernable. \sphinxstyleemphasis{Bloomberg}. https://www.bloomberg.com/news/articles/2025\sphinxhyphen{}09\sphinxhyphen{}18/across\sphinxhyphen{}germany\sphinxhyphen{}france\sphinxhyphen{}poland\sphinxhyphen{}and\sphinxhyphen{}the\sphinxhyphen{}uk\sphinxhyphen{}europe\sphinxhyphen{}is\sphinxhyphen{}becoming\sphinxhyphen{}ungovernable?srnd=homepage\sphinxhyphen{}europe

\item {} 
\sphinxAtStartPar
Cadena SER. (2025, mayo 7). Encuentro estratégico para reforzar la colaboración entre las Cámaras de Comercio, la Cámara de España y el Gobierno de Aragón. \sphinxstyleemphasis{Cadena SER}. https://cadenaser.com/aragon/2025/05/07/encuentro\sphinxhyphen{}estrategico\sphinxhyphen{}para\sphinxhyphen{}reforzar\sphinxhyphen{}la\sphinxhyphen{}colaboracion\sphinxhyphen{}entre\sphinxhyphen{}las\sphinxhyphen{}camaras\sphinxhyphen{}de\sphinxhyphen{}comercio\sphinxhyphen{}la\sphinxhyphen{}camara\sphinxhyphen{}de\sphinxhyphen{}espana\sphinxhyphen{}y\sphinxhyphen{}el\sphinxhyphen{}gobierno\sphinxhyphen{}de\sphinxhyphen{}aragon\sphinxhyphen{}radio\sphinxhyphen{}huesca/

\item {} 
\sphinxAtStartPar
Cadena SER. (2025, mayo 13). Castilla\sphinxhyphen{}La Mancha ha gestionado ya el 81 \% de los fondos Next Generation y está dispuesta a recibir más. \sphinxstyleemphasis{Cadena SER}. https://cadenaser.com/castillalamancha/2025/05/13/castilla\sphinxhyphen{}la\sphinxhyphen{}mancha\sphinxhyphen{}ha\sphinxhyphen{}gestionado\sphinxhyphen{}ya\sphinxhyphen{}el\sphinxhyphen{}81\sphinxhyphen{}de\sphinxhyphen{}los\sphinxhyphen{}fondos\sphinxhyphen{}next\sphinxhyphen{}generation\sphinxhyphen{}con\sphinxhyphen{}un\sphinxhyphen{}impacto\sphinxhyphen{}sin\sphinxhyphen{}precedentes\sphinxhyphen{}en\sphinxhyphen{}la\sphinxhyphen{}region\sphinxhyphen{}y\sphinxhyphen{}esta\sphinxhyphen{}dispuesta\sphinxhyphen{}a\sphinxhyphen{}recibir\sphinxhyphen{}mas\sphinxhyphen{}ser\sphinxhyphen{}toledo/

\item {} 
\sphinxAtStartPar
Cadena SER. (2025, septiembre 17). Los planes de desarrollo turísticos que la Diputación Provincial ejecuta en la provincia suponen una inversión de más de 70 millones de euros. \sphinxstyleemphasis{Cadena SER}. https://cadenaser.com/andalucia/2025/09/17/los\sphinxhyphen{}planes\sphinxhyphen{}de\sphinxhyphen{}desarrollo\sphinxhyphen{}turisticos\sphinxhyphen{}que\sphinxhyphen{}la\sphinxhyphen{}diputacion\sphinxhyphen{}provincial\sphinxhyphen{}ejecuta\sphinxhyphen{}en\sphinxhyphen{}la\sphinxhyphen{}provincia\sphinxhyphen{}suponen\sphinxhyphen{}una\sphinxhyphen{}inversion\sphinxhyphen{}de\sphinxhyphen{}mas\sphinxhyphen{}de\sphinxhyphen{}70\sphinxhyphen{}millones\sphinxhyphen{}de\sphinxhyphen{}euros\sphinxhyphen{}radio\sphinxhyphen{}jodar/

\item {} 
\sphinxAtStartPar
Cámara de Comercio de España. (2025). La Cámara de España lanza una Guía de Financiación para pymes y emprendedores. \sphinxstyleemphasis{Cámara de Comercio de España}. https://www.camara.es/camara\sphinxhyphen{}espana\sphinxhyphen{}lanza\sphinxhyphen{}guia\sphinxhyphen{}financiacion\sphinxhyphen{}pymes\sphinxhyphen{}emprendedores

\item {} 
\sphinxAtStartPar
CEPYME. (2025). \sphinxstyleemphasis{Situación de las pymes}. Confederación Española de la Pequeña y Mediana Empresa. https://cepyme.es/situacion\sphinxhyphen{}de\sphinxhyphen{}las\sphinxhyphen{}pymes/

\item {} 
\sphinxAtStartPar
Cinco Días. (2024, diciembre 2). Alemania, el corazón industrial de Europa, necesita reanimación urgente. \sphinxstyleemphasis{Cinco Días \sphinxhyphen{} El País}. https://cincodias.elpais.com/economia/2024\sphinxhyphen{}12\sphinxhyphen{}02/alemania\sphinxhyphen{}el\sphinxhyphen{}corazon\sphinxhyphen{}industrial\sphinxhyphen{}de\sphinxhyphen{}europa\sphinxhyphen{}necesita\sphinxhyphen{}reanimacion\sphinxhyphen{}urgente.html

\item {} 
\sphinxAtStartPar
Cinco Días. (2024, diciembre 22). Telefónica confía que el acuerdo UE\sphinxhyphen{}Mercosur podría ser muy beneficioso para las telecos. \sphinxstyleemphasis{Cinco Días \sphinxhyphen{} El País}. https://cincodias.elpais.com/companias/2024\sphinxhyphen{}12\sphinxhyphen{}22/telefonica\sphinxhyphen{}confia\sphinxhyphen{}que\sphinxhyphen{}el\sphinxhyphen{}acuerdo\sphinxhyphen{}ue\sphinxhyphen{}mercosur\sphinxhyphen{}podria\sphinxhyphen{}ser\sphinxhyphen{}muy\sphinxhyphen{}beneficioso\sphinxhyphen{}para\sphinxhyphen{}las\sphinxhyphen{}telecos.html

\item {} 
\sphinxAtStartPar
Cinco Días. (2025, septiembre 12). El Gobierno anuncia 13.600 millones de inversión en las redes de transporte de electricidad hasta el 2030. \sphinxstyleemphasis{Cinco Días \sphinxhyphen{} El País}. https://cincodias.elpais.com/companias/2025\sphinxhyphen{}09\sphinxhyphen{}12/el\sphinxhyphen{}gobierno\sphinxhyphen{}anuncia\sphinxhyphen{}13600\sphinxhyphen{}millones\sphinxhyphen{}de\sphinxhyphen{}inversion\sphinxhyphen{}en\sphinxhyphen{}las\sphinxhyphen{}redes\sphinxhyphen{}de\sphinxhyphen{}transporte\sphinxhyphen{}de\sphinxhyphen{}electricidad\sphinxhyphen{}hasta\sphinxhyphen{}el\sphinxhyphen{}2030.html

\item {} 
\sphinxAtStartPar
Consejo General de Economistas. (2024). \sphinxstyleemphasis{Informe pyme 2024: Gestión del Talento}. Cátedra Pyme. https://catedrapyme.es/informe\sphinxhyphen{}pyme\sphinxhyphen{}2024\sphinxhyphen{}gestion\sphinxhyphen{}del\sphinxhyphen{}talento/

\item {} 
\sphinxAtStartPar
Emprendedores. (2025). El impacto de las pymes: una mirada española. \sphinxstyleemphasis{Emprendedores}. https://emprendedores.es/actualidad/impacto\sphinxhyphen{}pymes\sphinxhyphen{}espana/

\item {} 
\sphinxAtStartPar
Emprendedores. (2025). Financiación pyme: lenta normalización, pero avanza. \sphinxstyleemphasis{Emprendedores}. https://emprendedores.es/actualidad/pymes\sphinxhyphen{}financiacion/

\item {} 
\sphinxAtStartPar
Expansión. (2025, septiembre 18). Crónica de Bolsa. \sphinxstyleemphasis{Expansión}. https://www.expansion.com/mercados/cronica\sphinxhyphen{}bolsa/2025/09/18/68cb9dcce5fdeaf2598b459c.html

\item {} 
\sphinxAtStartPar
García, J. (2025, mayo 8). España, en riesgo de incumplir el plazo de ejecución de los Fondos Europeos. \sphinxstyleemphasis{El País}. https://elpais.com/economia/2025\sphinxhyphen{}05\sphinxhyphen{}08/espana\sphinxhyphen{}en\sphinxhyphen{}riesgo\sphinxhyphen{}latente\sphinxhyphen{}de\sphinxhyphen{}incumplir\sphinxhyphen{}el\sphinxhyphen{}plazo\sphinxhyphen{}de\sphinxhyphen{}ejecucion\sphinxhyphen{}de\sphinxhyphen{}los\sphinxhyphen{}fondos\sphinxhyphen{}europeos.html

\item {} 
\sphinxAtStartPar
Generali \& SDA Bocconi. (2025, mayo 13). Demostrado: invertir en sostenibilidad beneficia a las pequeñas y medianas empresas. \sphinxstyleemphasis{Los40}. https://los40.com/2025/05/13/demostrado\sphinxhyphen{}invertir\sphinxhyphen{}en\sphinxhyphen{}sostenibilidad\sphinxhyphen{}beneficia\sphinxhyphen{}a\sphinxhyphen{}las\sphinxhyphen{}pequenas\sphinxhyphen{}y\sphinxhyphen{}medianas\sphinxhyphen{}empresas/

\item {} 
\sphinxAtStartPar
Gobierno de España. (2024). \sphinxstyleemphasis{Cómo se están transformando las pymes con el Plan de Recuperación (PRTR)}. Plan de Recuperación, Transformación y Resiliencia. https://planderecuperacion.gob.es/noticias/como\sphinxhyphen{}se\sphinxhyphen{}estan\sphinxhyphen{}transformando\sphinxhyphen{}las\sphinxhyphen{}pymes\sphinxhyphen{}con\sphinxhyphen{}el\sphinxhyphen{}plan\sphinxhyphen{}de\sphinxhyphen{}recuperacion\sphinxhyphen{}prtr

\item {} 
\sphinxAtStartPar
Gutiérrez, C. (2025, marzo 17). Callejón sin salida para las pymes: su rentabilidad cae y los costes se disparan. \sphinxstyleemphasis{El Español \sphinxhyphen{} Invertia}. https://www.elespanol.com/invertia/economia/empleo/20250317/callejon\sphinxhyphen{}sin\sphinxhyphen{}salida\sphinxhyphen{}pymes\sphinxhyphen{}rentabilidad\sphinxhyphen{}cae\sphinxhyphen{}costes\sphinxhyphen{}disparan/931157099\_0.html

\item {} 
\sphinxAtStartPar
Hostelería Madrid. (2024). La pyme española pierde competitividad y productividad en 2024. \sphinxstyleemphasis{Hostelería Madrid}. https://www.hosteleriamadrid.com/blog/la\sphinxhyphen{}pyme\sphinxhyphen{}espanola\sphinxhyphen{}pierde\sphinxhyphen{}competitividad\sphinxhyphen{}y\sphinxhyphen{}productividad\sphinxhyphen{}en\sphinxhyphen{}2024/

\item {} 
\sphinxAtStartPar
OpenAI. (2024). \sphinxstyleemphasis{Modelos disponibles: GPT\sphinxhyphen{}4o}. OpenAI. https://platform.openai.com/docs/models/gpt\sphinxhyphen{}4o

\item {} 
\sphinxAtStartPar
OpenAI. (2024). \sphinxstyleemphasis{Sora: Generación de video a partir de texto}. OpenAI. https://openai.com/es\sphinxhyphen{}ES/index/sora/

\item {} 
\sphinxAtStartPar
Reuters. (2025, septiembre 13). US financial firms pledge £17 billion to UK ahead of Trump’s visit. \sphinxstyleemphasis{Reuters}. https://www.reuters.com/business/finance/us\sphinxhyphen{}financial\sphinxhyphen{}firms\sphinxhyphen{}pledge\sphinxhyphen{}17\sphinxhyphen{}billion\sphinxhyphen{}uk\sphinxhyphen{}ahead\sphinxhyphen{}trumps\sphinxhyphen{}visit\sphinxhyphen{}2025\sphinxhyphen{}09\sphinxhyphen{}13/

\item {} 
\sphinxAtStartPar
Reuters. (2025, septiembre 17). Fed easing a mixed blessing for the rest of the world. \sphinxstyleemphasis{Reuters}. https://www.reuters.com/markets/funds/fed\sphinxhyphen{}easing\sphinxhyphen{}mixed\sphinxhyphen{}blessing\sphinxhyphen{}rest\sphinxhyphen{}world\sphinxhyphen{}2025\sphinxhyphen{}09\sphinxhyphen{}17/

\item {} 
\sphinxAtStartPar
Reuters. (2025, septiembre 17). Policy easing fuels optimism for Asian EM equities despite political headwinds. \sphinxstyleemphasis{Reuters}. https://www.reuters.com/world/china/policy\sphinxhyphen{}easing\sphinxhyphen{}fuels\sphinxhyphen{}optimism\sphinxhyphen{}asian\sphinxhyphen{}em\sphinxhyphen{}equities\sphinxhyphen{}despite\sphinxhyphen{}political\sphinxhyphen{}headwinds\sphinxhyphen{}2025\sphinxhyphen{}09\sphinxhyphen{}17/

\item {} 
\sphinxAtStartPar
Revista Pymes. (2024). Aumento de costes y falta de financiación hunden la productividad de las pymes. \sphinxstyleemphasis{Revista Pymes}. https://revistapymes.es/aumento\sphinxhyphen{}de\sphinxhyphen{}costes\sphinxhyphen{}y\sphinxhyphen{}falta\sphinxhyphen{}de\sphinxhyphen{}financiacion\sphinxhyphen{}hunden\sphinxhyphen{}la\sphinxhyphen{}productividad\sphinxhyphen{}de\sphinxhyphen{}las\sphinxhyphen{}pymes/

\item {} 
\sphinxAtStartPar
Rivas, C. (2025, febrero 17). Un estudio de Cepyme alerta de que el nuevo salario mínimo representa el 70\% del sueldo medio en las pequeñas empresas. \sphinxstyleemphasis{El País}. https://elpais.com/economia/2025\sphinxhyphen{}02\sphinxhyphen{}17/un\sphinxhyphen{}estudio\sphinxhyphen{}de\sphinxhyphen{}cepyme\sphinxhyphen{}alerta\sphinxhyphen{}de\sphinxhyphen{}que\sphinxhyphen{}el\sphinxhyphen{}nuevo\sphinxhyphen{}salario\sphinxhyphen{}minimo\sphinxhyphen{}representa\sphinxhyphen{}el\sphinxhyphen{}70\sphinxhyphen{}del\sphinxhyphen{}sueldo\sphinxhyphen{}medio\sphinxhyphen{}en\sphinxhyphen{}las\sphinxhyphen{}pequenas\sphinxhyphen{}empresas.html

\item {} 
\sphinxAtStartPar
Sánchez, R. (2025, mayo 6). Las pymes redujeron su actividad internacional en 2024 y advierten que la situación empeorará con la guerra arancelaria de Trump. \sphinxstyleemphasis{El País}. https://elpais.com/economia/2025\sphinxhyphen{}05\sphinxhyphen{}06/las\sphinxhyphen{}pymes\sphinxhyphen{}redujeron\sphinxhyphen{}su\sphinxhyphen{}actividad\sphinxhyphen{}internacional\sphinxhyphen{}en\sphinxhyphen{}2024\sphinxhyphen{}y\sphinxhyphen{}advierten\sphinxhyphen{}que\sphinxhyphen{}la\sphinxhyphen{}situacion\sphinxhyphen{}empeorara\sphinxhyphen{}con\sphinxhyphen{}la\sphinxhyphen{}guerra\sphinxhyphen{}arancelaria\sphinxhyphen{}de\sphinxhyphen{}trump.html

\item {} 
\sphinxAtStartPar
The Wall Street Journal. (2025, septiembre 17). PGIM and Partners Group join forces to deliver multi\sphinxhyphen{}asset investments. \sphinxstyleemphasis{The Wall Street Journal}. https://www.wsj.com/articles/pgim\sphinxhyphen{}and\sphinxhyphen{}partners\sphinxhyphen{}group\sphinxhyphen{}join\sphinxhyphen{}forces\sphinxhyphen{}to\sphinxhyphen{}deliver\sphinxhyphen{}multi\sphinxhyphen{}asset\sphinxhyphen{}investments\sphinxhyphen{}d54e54dc

\item {} 
\sphinxAtStartPar
Trustnet. (2025, septiembre 18). Retail investors back UK and US but sentiment down for other markets. \sphinxstyleemphasis{Trustnet}. https://www.trustnet.com/news/13458549/retail\sphinxhyphen{}investors\sphinxhyphen{}back\sphinxhyphen{}uk\sphinxhyphen{}and\sphinxhyphen{}us\sphinxhyphen{}but\sphinxhyphen{}sentiment\sphinxhyphen{}down\sphinxhyphen{}for\sphinxhyphen{}other\sphinxhyphen{}markets

\item {} 
\sphinxAtStartPar
Instituto Español de Analistas. (s. f.). Instituto Español de Analistas. https://institutodeanalistas.com/

\end{enumerate}

\sphinxstepscope


\chapter{Anexos}
\label{\detokenize{Anexos:anexos}}\label{\detokenize{Anexos::doc}}

\section{Anexo A: Repositorio del Proyecto}
\label{\detokenize{Anexos:anexo-a-repositorio-del-proyecto}}
\sphinxAtStartPar
El repositorio del proyecto completo se encuentra alojado en GitHub:
\begin{enumerate}
\sphinxsetlistlabels{\arabic}{enumi}{enumii}{}{.}%
\item {} 
\sphinxAtStartPar
\sphinxhref{https://github.com/TFMUCM01/TFM}{Repositorio del Proyecto}

\item {} 
\sphinxAtStartPar
\sphinxhref{https://t.me/TFMGRUPO4\_BOT}{Datos del Robot Financiero}

\item {} 
\sphinxAtStartPar
\sphinxhref{https://ucomplutense-my.sharepoint.com/?login\_hint=mmendi04\%40ucm\%2Ees\&amp;source=waffle}{Sharepoint}

\item {} 
\sphinxAtStartPar
\sphinxhref{https://public.tableau.com/app/profile/julia.escudero.velasco/viz/TFM\_17581412187180/Indice}{Dashboard de Análisis Financiero en Tableau}

\item {} 
\sphinxAtStartPar
\sphinxhref{https://app.snowflake.com/vlnvldd/wj67583/\#/homepage}{Snowflake DataLake}

\end{enumerate}


\section{Anexo B: Cuadernos Jupyter}
\label{\detokenize{Anexos:anexo-b-cuadernos-jupyter}}\begin{enumerate}
\sphinxsetlistlabels{\arabic}{enumi}{enumii}{}{.}%
\item {} 
\sphinxAtStartPar
\sphinxhref{https://github.com/TFMUCM01/TFM/blob/main/Analisis\_Tecnico/Analisis\_Tecnico.ipynb}{Análisis Técnico básico para Mercados Financieros}

\item {} 
\sphinxAtStartPar
\sphinxhref{https://github.com/TFMUCM01/TFM/blob/main/Analisis\_Financiero/Frontera\_Eficiencia.ipynb}{Análisis financiero de frontera de eficiencia}

\item {} 
\sphinxAtStartPar
\sphinxhref{https://github.com/TFMUCM01/TFM/blob/main/Analisis\_Financiero/Security\_Market\_Line.ipynb}{Análisis financiero de Security Market Line}

\item {} 
\sphinxAtStartPar
{[}Análisis de sentimiento{]}

\item {} 
\item {} 
\item {} 
\sphinxAtStartPar
\sphinxhref{https://github.com/TFMUCM01/TFM/blob/main/N8N\_workflows/noticias\_actualizadas.json}{Coneccion de N8N noticias actualizadas}

\item {} 
\sphinxAtStartPar
\sphinxhref{https://github.com/TFMUCM01/TFM/blob/main/N8N\_workflows/noticias\_analizadas\_ingles.json}{Coneccion de N8N noticias analizadas ingles}

\item {} 
\sphinxAtStartPar
\sphinxhref{https://github.com/TFMUCM01/TFM/blob/main/N8N\_workflows/robot\_n8n.json}{Coneccion de N8N Telegram bot}

\item {} 
\sphinxAtStartPar
\sphinxhref{https://github.com/TFMUCM01/TFM/tree/main/Yahoo\_prueba/indices\_diarios.py}{Descarga de precio diario}

\item {} 
\sphinxAtStartPar
\sphinxhref{https://github.com/TFMUCM01/TFM/tree/main/Yahoo\_prueba/tickers\_precios\_global.py}{Descarga de tickets}

\item {} 
\sphinxAtStartPar
\sphinxhref{https://github.com/TFMUCM01/TFM/blob/main/N8N\_workflows/yahoo\_finance\_actualizado.json}{Coneccion de N8N Yahoo Finance}

\item {} 
\sphinxAtStartPar
\sphinxhref{https://github.com/TFMUCM01/TFM/tree/main/Yahoo\_prueba/financieros\_esg\_snapshot.py}{Descarga de indicafores ESG financiero}

\item {} 
\sphinxAtStartPar
\sphinxhref{https://github.com/TFMUCM01/TFM/tree/main/Yahoo\_prueba/financieros\_esg\_snapshot\_hist.py}{Descarga de indicafores ESG financiero historico}

\item {} 
\sphinxAtStartPar
\sphinxhref{https://github.com/TFMUCM01/TFM/tree/main/Yahoo\_prueba/financieros\_resumen\_anual.py}{Descarga de indicafores estados de resultados}

\item {} 
\sphinxAtStartPar
\sphinxhref{https://github.com/TFMUCM01/TFM/tree/main/Yahoo\_prueba/financieros\_snapshot.py}{Descarga de indicafores ratios}

\item {} 
\sphinxAtStartPar
\sphinxhref{https://github.com/TFMUCM01/TFM/tree/main/Yahoo\_prueba/financieros\_snapshot\_hist.py}{Descarga de indicafores ratios historicos}

\item {} 
\sphinxAtStartPar
\sphinxhref{https://github.com/TFMUCM01/TFM/tree/main/Yahoo\_prueba/indices\_diarios.py}{Descarga de precio diario}

\item {} 
\sphinxAtStartPar
\sphinxhref{https://github.com/TFMUCM01/TFM/tree/main/Yahoo\_prueba/tickers\_precios\_global.py}{Descarga de tickets}

\end{enumerate}


\section{Anexo C: Pagina Web}
\label{\detokenize{Anexos:anexo-c-pagina-web}}
\begin{figure}[htbp]
\centering

\noindent\sphinxincludegraphics{{1}.png}
\end{figure}

\begin{figure}[htbp]
\centering

\noindent\sphinxincludegraphics{{2}.png}
\end{figure}

\begin{figure}[htbp]
\centering

\noindent\sphinxincludegraphics{{3}.png}
\end{figure}

\begin{figure}[htbp]
\centering

\noindent\sphinxincludegraphics{{4}.png}
\end{figure}



\renewcommand{\indexname}{Índice}
\printindex
\end{document}